\documentclass{article}
\usepackage[utf8]{inputenc}

\usepackage{mathtools}
\usepackage{amsthm}
\usepackage{amssymb}
\usepackage{dsfont}
\usepackage{array}   % for \newcolumntype macro
\usepackage{calligra}
\usepackage{tikz-cd}
\usepackage{mathrsfs}
\usepackage{quiver}
\usepackage[shortlabels]{enumitem}

\newcolumntype{L}{>{$}l<{$}}
% math-mode version of "l" column type

\newcommand{\Rho}{\mathrm{P}}

\newcommand{\iso}{\simeq}
\newcommand{\after}{\circ}
\newcommand{\catname}[1]{\mathbf{#1}}
\newcommand{\intersect}{\cap}
\newcommand{\union}{\cup}
\newcommand{\Intersect}{\bigcap}
\newcommand{\Union}{\bigcup}
\newcommand{\wa}[1]{\langle #1 \rangle}
\newcommand{\bb}[1]{\mathbb{#1}}
\newcommand{\A}{\mathbb{A}}
\newcommand{\Z}{\mathbb{Z}}
\newcommand{\Q}{\mathbb{Q}}
\newcommand{\R}{\mathbb{R}}
\newcommand{\C}{\mathbb{C}}
\newcommand{\F}{\mathbb{F}}
\newcommand{\p}{\mathfrak{p}}
\newcommand{\q}{\mathfrak{q}}
\newcommand{\ai}{\mathfrak{a}}
\newcommand{\bi}{\mathfrak{b}}
\newcommand{\m}{\mathfrak{m}}
\newcommand{\defeq}{\vcentcolon=}

\DeclareMathOperator{\id}{id}
\DeclareMathOperator{\Hom}{\mathscr{H}\text{\kern -3pt
{\calligra\large om}}\,}
\DeclareMathOperator{\Fsh}{\mathscr{F}}
\DeclareMathOperator{\Gsh}{\mathscr{G}}
\DeclareMathOperator{\Hsh}{\mathscr{H}}
\DeclareMathOperator{\Osh}{\mathscr{O}}
% \DeclareMathOperator{\ker}{ker}
\DeclareMathOperator{\coker}{coker}
\DeclareMathOperator{\im}{im}
\DeclareMathOperator{\Spec}{Spec}
\DeclareMathOperator{\Proj}{Proj}
\DeclareMathOperator{\V}{V}
\DeclareMathOperator{\D}{D}

\newcommand*{\rationalto}[1][]{\mathbin{\tikz [baseline=-0.25ex,-latex, dashed,->,densely dashed,#1] \draw [#1] (0pt,0.5ex) -- (1.3em,0.5ex);}}%


\title{Chapter 8 Problems}
\author{Josiah Bills}
\date{March 2021}

\begin{document}

\maketitle

\section{8.1.A}
Choose an affine cover. Injectivity clearly holds on this cover, and so holds
globally. The complement of the image on an element of the affine cover is
open, so complement of the overall image is open. Hence the image is closed.
\qed

\section{8.1.B}
Quotients are clearly finite over the base ring, and closed embeddings are
affine by definition.

\section{8.1.C}
Composition of affines morphisms is affine. Composition of surjective morphisms
of rings is surjective. \qed

\section{8.1.D}
Consider $A \to B$ surjective. Then $A_f \to B_f$ is also
surjective. The latter part of the affine communication lemma is just gluing
schemes along affine opens.

\section{8.1.E}
Apply 8.1.D. \qed

\section{8.1.F}
Let $f: Y \to X$ be a closed subscheme. There are only two closed
sets in $X$, $X$ itself and
${\m}$. For the latter, we must have a surjective morphism
$k[x]_{(x)} \to l$ where $l$ is a field. The kernel of
this must be the closed point $\m$, which is not
$mathfrak{I}(X)$. Suppose we are given $f: X \to X$ an
automorphism and closed embedding. Then $f^{-1}(\D_x)=f^{-1}({\eta})$, giving us
$f^{\sharp}: k(x) \to B$. If the condition on ideals holds then the kernel is
$k(x)$, meaning that $B$ is the empty set,
contradicting $f$ being an automorphism. \qed

\section{8.1.G}
Let $\phi: B \to B_f$ be the natural map. Then there is an induced map
$\phi': I(B)_f \to I(B_f) x/f \mapsto \phi(x)/f$. The result follows by exactness of localization.

\section{8.1.H}
It is easy to see that $\operatorname{I}(B)$ defines a sheaf on a base. The
restriction maps are the restriction to the ideal of the ring restriction maps.
 The hypothesis ensures this is well defined. Let $\Ish$ be the
induced sheaf. It remains to conclude that $\Osh / \mathfrak{I}$, regarded as
the cokernel in the category of presheaves, is in fact a sheaf. It suffices to
show it is a sheaf on a base. By definition, $(\Osh / \mathfrak{I})(\Spec B)=B/I(B)$. Also,
$(\Osh / \mathfrak{I})(\Spec B_f)=B_f/I(B_f)=(B/I(B))_f$ so that $\Osh / \mathfrak{I}$ agrees with the structure
sheaf of $\Spec B/I(B)$. \qed

\section{8.1.I}
\begin{enumerate}[a.]
    \item Consider the ideal $(s)$ on some affine open cover. Let
          $\phi: A \to A_f$. Then $(\phi(s))=(s)_f$ so that we satisfy the
          conditions of 8.1.H.
    \item It is clear that on an affine open cover $(su)=(s)$. \qed
    \item Using notation as in (a), we have $(\phi((S)))=(S)_f$ and we are done.
\end{enumerate}

\section{8.1.J}
\begin{enumerate}[a.]
    \item We check that if $I_1, I_2 \subseteq A$ are ideals, then $(I_1 \intersect I_2)_f = I_{1,f} \intersect
              I_{2,f}$.
          Let $x/f^n \in (I_1 \intersect I_2)_f$. Then since $x \in I_1 \intersect I_2$,
          $x/f^n \in I_{1,f} \intersect
              I_{2,f}$. The converse holds for similar reasons. Now consider
          $(x_1+x_2)/f^n \in (I_1 + I_2)_f$ where $x_i \in I_i$. It is obvious that
          $(x_1+x_2)/f^n=x_1/f^n+x_2/f^n \in I_{1,f} +
              I_{2,f}$. Thus $(I_1 \intersect I_2)(B)$ and $(I_1 + I_2)(B)$
          define closed embeddings by 8.1.H. \qed
    \item We have $(y-x^2)+(y)=(y,x^2)$, so we have a closed subscheme
          $\Spec A[x, y]/(x^2, y)$, a parabola intersect the x-axis. Now,
          $A[x, y]/(x^2, y) \cong
              A[x]/(x^2)$. The union is given by $(y-x^2) \intersect (y)$. \qed
    \item Trivial
    \item We have $(y^2-x^2)+(y)=(y, x^2)$, we also have $(y^2-x^2)=((y+x)(y-x))=(y+x)(y-x)$. Thus the
          above ideal corresponds to $(\V(y+x)\union\V(y-x))\intersect\V(y)$. Now $\V(y+x)\intersect\V(y)=\V(x,y)$,
          but $(x,y)(x,y) \neq (y, x^2)$. So while they give the same closed set, they don't
          give the same closed subscheme.
\end{enumerate}

\section{8.1.K}
Let $\pi: X \to \Spec A$ be a morphism which induces a homeomorphism. Then
$X$ is affine. Let $B=\Osh_X(X)$. Then we have a
morphism $\phi: X \to \Spec B$. It is clear that $\phi$ is a
homeomorphism, so it remains to show that it is an isomorphism of sheaves. Let
$\Spec C \subseteq X$. Then we have a morphism $\phi': B \to C$. Now,
let $b \in B$ be such that $\phi^{-1}(\D(b)) \subseteq \Spec C$. But it is clear
that $\phi^{-1}(\D(b)) = \D(\phi'(b))$. Now, let $b$ range over
$B$ to get a cover of $\Spec B$ by affines.
By affine locallity of affineness, we are done.

The fact that closed immersions satisfy this property follows from 2.5.E. For
the other direction, we immediately reduce to the case where
$Y$ is affine, say $\Spec A$. We find that
$\pi$ corresponds to a morphism $\pi^{\sharp}: A \to \Osh_X(X)$.
Taking kernels, we find that $\pi=\psi \after \phi$ where
$\psi: \Spec A/I \to \Spec A$ is a closed embedding. Since $\psi$
induces a homeomorphism, by the above lemma, $X$ is affine.
Hence the morphism is affine.

We change our notation now and write $\pi: A \to B$,
$\psi: A \to A/I$, and $\phi: A/I \to B$, with
$\phi$ inducing a homeomorphism of spaces. By 8.1.H, the
closed subscheme structure of $\psi$ is uniquely determined by
the ideal $I=\ker \psi = \ker \pi$. Now, apply 2.4.E to the stalks of
$\phi$. \qed

\section{8.1.L}
Closed embeddings are finite by 8.1.B. Finite morphisms are locally of finite
type by 7.3.P. Open embeddings are locally of finite type by 7.3.Q(a), so the
result follows by 7.3.Q(b). \qed

\section{8.1.M}
The intersection is defined as the pullback given in the hint. (i)
$\iff$ (ii) follows directly from 7.2.B. (i)
$\implies$ (iii) follows from 9.2.B.

\section{8.1.N}
Consider $f=f_2 \after g_2 \after f_1 \after g_1$ where $g_1, g_2$ are closed
embeddings and $f_1, f_2$ are open embeddings. Then by 8.1.M we
have $f=f_2 \after f_3 \after g_3 \after g_1$, where $g_3$ is a closed
embedding and $f_3$ is an open embedding. \qed

\section{Some remarks on $\Proj A$}
We first note that $\Proj A \subseteq \Spec A$ has the subset topology. In
particular, if $\Spec A$ is irreducible then so is
$\Proj A$. Indeed, a nontrivial closed cover
$Z \union X = \Proj A$ yields a nontrivial closed cover
$Z' \union X' = \Spec A$.

\section{8.2.A}
\begin{enumerate}[a.]
    \item It suffices to show that $\V((z-x^3, x^2-y)) \subseteq k[x, y ,z]$ is irreducible, then by 9.5.L
          we will be done. Now, we map $k[x, y ,z] \to k[x]$ by $x \mapsto x, y \mapsto x^2, z \mapsto x^3$.
          Clearly the kernel of this map is $(z-x^3, x^2-y)$, and this map is
          clearly surjective. \qed
    \item The above isomorphism shows that on each open it is isomorphic to
          $\A^1_k$, hence after gluing it is $\P^1_k$.
\end{enumerate}

\section{8.2.B}
Let $\phi: S_{\bullet} \to R_{\bullet}$ be a surjective map of graded rings. Then
$\ker \phi$ is homogenous. Indeed, we get induced morphisms of
$S_0$-modules $\phi_i: S_i \to R_{di}$. These are surjective,
hence their kernel is graded with degree $i$. Then clearly
$\ker \phi = \Union_i \ker \phi_i$. Next, we note that $d=1$. From this
we immediately see that $\phi(S_+)=R_+$, hence we have an induced
morphism $\Proj R_{\bullet} \to \Proj S_{\bullet}$. Finally, note that if $\ker \phi = I$
then $\Proj R_{\bullet} \supseteq \D(f) = \Spec
    S_{\bullet, f, 0}/I_{f, 0}$ by exactness of localization. \qed

\section{8.2.C}
Lemma: Let $S_{\bullet}$ be an arbitrary graded ring and let
$f \in S_{\bullet}$ be homogenous of degree $d$.
Suppose $I \subset S_{\bullet, f, 0}$ is an arbitrary ideal. Let
$\phi: S_{\bullet, f, 0} \to S_{\bullet}$ be the obvious inclusion. Then $\V(I^{\text{eh}})=\V(I^{\text{e}})$
(with notation as in the proof of 4.5.H(a)). Indeed, if $g=\sum_{i=0}^n g_{id}/f^i$
then clearly $f^ng$ is homogenous of degree
$dn$ and hence is contained in $I^{\text{eh}}$. But
then any homogenous prime ideal must contain either $f^n$ or
$g$. Clearly it cannot contain the former, hence it must
contain the latter. Hence it also contains $I^{\text{e}}$. \qed

Note also that $I = I^{\text{ehc}}$. Indeed, Let $x \in I^{\text{eh}}$ and
$x \in \im \phi$. WLOG assume that $x$ is
homogenous. The former relationship says that $x=ay$ where
$y \in I$ and $a$ is homogenous. The latter
implies that $a$ must have degree zero, hence
$a \in I$ so that $x \in I$.

Next we reduce to the case of a graded ring finitely generated in degree one by
applying 6.4.G followed by 6.4.D. Such a ring is a finitely generated
$A$-algebra, yielding a closed embedding into
$\P^n_A$ for some $n$ (by 8.2.B). Since the
composition of two closed embeddings is a closed embedding, we are reduced to
the case of an arbitrary closed embedding into $\P^n_A$.

We cover $\P^n_A$ by affine opens $\D(x_1), \dots \D(x_n)$. Now,
on each $\D(x_i)$ we have a closed subscheme defined by an ideal
$\Ish(\D(x_i))=I_i$. Now, applying the above lemmas we find that we have a
homogenous ideal $(I_i)^{eh} \subset
    A[x_1, \dots, x_n]_{x_i}$ and a corresponding relevant
homogenous ideal of $J_i \subset A[x_1, \dots, x_n]$ which doesn't contain
$x_i$. This ideal corresponds to a closed subset of
$X$. Taking the union of these across the
$\D(x_i)$ we get a open cover of $X$ which
corresponds to the homogenous ideal $\Intersect_i J_i$.

\section{8.2.D}
Injective linear maps of (finite dimensional) vector spaces induce surjective
maps of their duals. This extends to a surjective map of the symmetric algebra,
hence (by 8.2.B) to a closed embedding. \qed

\section{8.2.E}
Let $f \in k[x, y]$ be homogenous of degree $d$. We
show that this has exactly $d$ nontrivial roots. There are
two cases. If $f$ has degree $d$ when
dehomogenized with respect to $x$ (resp.
$y$) then we do not have a factorization
$f=x^ig$, hence $[0, 1]$ is not a root. Because
$[0, 1]$ is the only point not contained in
$\D(x)$, we find that it has exactly $d$
roots. Now, suppose $f$ has degree less than
$d$ when dehomogenized with respect to
$x$. Then we have a factorization $f=x^ig$
where $x \nmid g$. Then $g$ has
$d-i$ roots, and we have a root with multiplicity
$i$ at $[0, 1]$.

We now reduce the general case to the above theorem. Let
$\P^n_k$ be given by $k[x_0, \dots, x_n]$. Given an arbitrary
line we may perform a projective transform so that the line is defined by the
ideal $(x_0, \dots, x_{n-2})$. Now, the degree of $f$ is
preserved by projective transform, and since $f$ does not
contain our line we see that $x_i \nmid f$ for
$i \in [0, n-2]$. Thus $f$ contains at least one
monomial (of degree $d$) in only $x_{n-1}, x_n$.
\qed

\section{8.2.F}
By 3.6.F this morphism is surjective. The same theorem describes the kernel of
this map as the defining equations of 8.2.A. \qed

\section{8.2.G}
I proved this as part of my proof of 8.2.C. \qed

\section{8.2.H}
Let $p \in \D(x_0) \subseteq \Proj S_{\bullet}$ be a closed point inducing a closed subscheme
$\Spec \kappa(p) \to \Proj S_{\bullet}$. Thus we have a surjective morphism
$S_{\bullet, x_0, 0} \to \kappa(p)$ whose kernel is necessarily $p$. We
know that $S_{\bullet, x_0, 0} \cong k[x_1, \dots, x_n]/I$ for some ideal $I$.
Hence, by the Nullstellensatz we know $p=(x_1-a_1, \dots, x_n-a_n)$. This prime ideal
corresponds to the homogenous prime ideal $(x_1-a_1x_0, \dots x_n-a_nx_0) \subset S_{\bullet}$. But this is
indeed a line in $\Spec S_{\bullet}$, as we have $n-1$
linear equations (which are clearly linearly independent) in a space with
$n$ variables. \qed

\section{8.2.I}
Trivial. \qed

\section{8.2.J}
This will follow from 3.6.F(b). Indeed, that exercise shows that we have a
surjection of rings $k[x_0, \dots, x_n] \to S_{d\bullet}$. This gives us the desired closed
embedding by 8.2.C. \qed

\section{8.2.K}
The desired result is equivalent to the following problem: We wish to count the
number of $n+1$-tuples $(a_i)_{i \in [0, n]}$ such that
$\sum_{i=0}^n a_i = d$. This is equivalent to counting the number of ways to
divide up an ordered list of $d$ indistinguishable objects
using $n$ partitions. But this is equivalent to choosing
$d$ items from an ordered list of $n+d$
indistinguishable items. (We think of this as choosing which of the
$n+d$ items will be partitions and which will be objects).
This number is clearly ${n+d \choose d}$. \qed

\section{8.2.L}
We have a surjection $k[y_0, y_1, y_2, y_3, y_4, y_5] \to k[x_0, x_1, x_2]_{2\bullet}$ which, for example, sends
$y_0 \mapsto x_0x_0$. Now, note that the cofactors of each entry in the
lower triangle of the determinant are zero, and are clearly linearly
independent. Pulling back these cofactors along the surjection, we get our
desired 6 quadratics.

\section{8.2.M}
Lemma: suppose we have finitely generated graded rings over a field
$k[x_0, \dots, x_m]/I, k[a_0, \dots, a_n]/J$ and a map which sends the closed point
$[a_0, \dots, a_n] \mapsto [f_1, \dots, f_m]$ where the $f_i$ are homogenous
polynomials, not all zero, in the $a_i$. Then this map is a
map of projective schemes. Indeed, we choose the map which sends
$x_i \mapsto f_i$. This is clearly a map of graded rings, and we may
check that it gives the polynomial map by applying 3.2.P to the distinguished
open sets. \qed

\begin{enumerate}[a.]
    \item By the above lemma, this gives a map of schemes. This is also clearly a closed
          embedding, since the map is surjective. We need only check that the image is a
          line. Indeed, a line in $\P^n_k$ is given by the span of two
          linearly independent vectors. Thus, if we have a function
          $[c, d] \to [f_0, \dots, f_n]$ where the $f_i$ are linear and have
          linearly independent coefficients, then we have a line. Furthermore, all lines
          must be of this form. The function given is clearly one of this form, as is the
          function $[c, d] \mapsto [a_0c, a_0d, b_0c, b_0d]$. These are the two rulings of lines. \qed
    \item Suppose $[w, x, y, z]$ is a point on the quadric surface. WLOG assume
          $w \neq 0$. Then we may take $a_0=w$ and
          $b_0=x$, then set $c=1, d=y$. For the other ruling
          we take $a_0=w$ and $b_0=y$, and set
          $c=1, d=x$. \qed
    \item In part (a) we classified lines in $\P^n_k$. Now, a plane cannot
          be simultaniously parallel to all the planes $\D(x_i)$. Thus, in
          our case we assume WLOG that we can find two points on the line with
          $w=1$. Thus we have $[a, b] \mapsto [a+b, ax+bx', ay+by', az+bz']$. Again, WLOG assume
          that $a \neq 0$, then we may rewrite this point as
          $[1+t, x+tx', y+ty', z+tz']$ where $t=b/a$ is arbitrary. Now, since
          the point lies on the quadric surface, $(1+t)(z+tz')=(x+tx')(y+ty')$. Set
          $t=-1$ and we get $(x-x')(y-y')=0$. The former
          corresponds to the first ruling, the latter corresponds to the second ruling.
          \qed
\end{enumerate}

\section{8.2.N}
Let $R_{\bullet}$ be the ring $k[x_0, x_1, x_2]$ with
$x_2$ given grading of two. Now, $R_{2\bullet}$ is
generated by 4 monomials, $x_0x_0, x_1x_0, x_1x_1, x_2$. Clearly the relation
$(x_1x_0)^2=(x_0x_0)(x_1x_1)$ holds. \qed

\section{8.2.O}
Follows from 8.2.P

\section{8.2.P}
We prove this for an arbitrary graded ring $R_{\bullet}$. Let
$R_+=(\{f_i\}_{i \in I})$. Then we have ring maps $R_{\bullet, f_i, 0} \to R_{\bullet, f_i}$. The
latter are the $\D(f_i)$, and clearly cover
$\Spec R_{\bullet} \setminus \V(R_+)$. The ring maps are also clearly compatible, hence glue
to give the desired map. \qed

\section{8.2.Q}
First note that $R[T]/(T) \cong R$ for any ring $R$.
This proves the first fact. Second note that $\D(T) \cong \Spec R_{\bullet}[T]_{T, 0} = \Spec
    R_{\bullet}[T]/(T-1)=\Spec R_{\bullet}$

\section{Note on definition of scheme theoretic image}
Suppose the image of $\pi: X \to Y$ lies in $Z$.
Then $\pi$ factors through the closed embedding
$\iota: Z \to Y$. On an open cover this corresponds to 9.2.B.

Now, suppose $X_1, X_2, \dots$ is the (not necessarily countable) family
of closed subschemes schemes such that the image of $\pi$
lies in each $X_i$. Then the scheme theoretic image is
defined to be the limit of the following diagram: \[\begin{tikzcd}
        {X_1} & {X_2} & \dots \\
              & Y
        \arrow[from=1-1, to=2-2]
        \arrow[from=1-2, to=2-2]
        \arrow[from=1-3, to=2-2]
    \end{tikzcd}\] which
exists by 8.1.J(a). Now, by definition of a limit, since we have a map
$\pi: X \to X_i \to Y$, we also have a map $X \to \lim X_i$.

\section{8.3.A}
Note that under the hypothesis, the scheme theoretic image is computable
affine-locally. Reducedness is also affine local. Hence it suffices to show
that the kernel of a map to a reduced ring is radical. Indeed, let
$f(a^n)=0$. Then $f(a)^n=0$ so that
$f(a)=0$, by reducedness. Thus, $a \in \ker f$. \qed

\section{8.3.C}
\[\begin{tikzcd}
          &   & Z \\
        V & U & X
        \arrow[from=1-3, to=2-3]
        \arrow[hook, from=2-2, to=2-3]
        \arrow[hook, from=2-1, to=2-2]
    \end{tikzcd}\]
We wish to show, under the hypotheses of 8.3.4, that the pullback of
$U \to X$ along $Z\to X$ is the same as the
pullback of $V \to X$ along $Z \to X$. By 8.3.4
this can be checked affine locally. Choose $\Spec A \subset U$. Then this
statement reduces to the triviality $A/I \otimes_A A/I \cong A/I$. \qed

\section{8.3.E}
\begin{itemize}
    \item[(i) $\iff$ (ii)] We note that $W$ is reduced (since reducedness is stalk
        local), hence by 8.3.4, we may check the image affine locally. We will consider
        $\Spec B \to \Spec A$, where $\Spec B=W$. Then the image is given
        by an ideal $I$ such that if $x \in I$ then
        $x$ is zero in every residue field of
        $A$. But this means that $x$ is in every
        prime ideal, hence $I=\sqrt{(0)}$.
    \item[(iii) $\iff$ (i)] We note that both definitions can be checked on an affine open cover. Then this
        reduces to the statement that $\V(0)=\V(\sqrt{(0)})$.
\end{itemize}
\qed

\section{8.3.F}
For the first, just use the second definition, clearly the image is the entire
space. For the second, again use the second definition and recall that
reducedness is stalk-local. \qed

\section{8.3.G}
This is just the first of the three definitions in 8.3.9.

\section{8.4.A}
Suppose $tx=0$ in $A_{\p}$. Then
$stx=0$ where $s \not \in \p$, so that
$sx=0$. Thus $x=0$ in
$A_{\p}$. \qed

\section{8.4.C}
We see that $\sqrt{(t)}=\sqrt{(t')}$. Let $(a)=\sqrt{(t)}$ and let
$(a')=\sqrt{(t')}$. Then both $a$ and
$a'$ vanish to the same order on $\V(t)=\V(t')$
(since these are identical as subschemes), hence $a^n=t$ and
$a'^n=t'$. Now, it is clear that $a'=sa$ where
$s$ is a unit, hence the result follows. \qed

\section{8.4.D}
Condition (ii) fails because the localization of a non-zero module can be zero.

The property is preserved by localization since being a non-zero divisor is
preserved by localization. Indeed, let $x \in A$ be a non-zero
divisor, and suppose $xm/s=0 \in S^{-1}M$. Then $s'xm=0$ for
some $s' \in S$. But $x$ isn't a zero divisor,
so $s'm=0$. Thus $m=0 \in S^{-1}M$. \qed

\section{8.4.E}
We prove this by induction. The result is clear by hypothesis when
$N=1$. Now suppose it holds for all $n<N$.
Now, by way of contradiction suppose $y$ is a zero-divisor
modulo $(x^N)M$. Then $ym=x^Nm'$. Now,
$x^Nm' \in (x)M$, so $m \in (x)M$ by hypothesis. Also by
hypothesis, $x$ is not a zero divisor, so
$ym''=x^{N-1}m'$ for $m''=xm$. But by the induction
hypothesis, this can only hold if $m'' \in (x^{N-1})M$. \qed

\section{8.4.F}
We begin by considering $x_1, x_2 \in M$ to be a regular sequence. Now,
we may also define maps $m_{x_1}: M \to M, m_{x_2}: M \to
    M$ given by multiplication. Now,
$x_1$ being a zero divisor is equivalent to the statement
that $m_{x_1}$ is injective. We wish to prove that
$m_{x_2}$ is injective, and also that $m'_{x_1}: M/(x_2)M \to M/(x_2)M$ is
injective. For the latter, note that if $x_1a=x_2a'$ then
$a'=x_1a''$ by regularity. Thus since $x_1$ is
not a zero divisor, $a=x_2a''$.

For the former, let $a \in \ker m_{x_2}$. Then $x_2a=0 \in (x_1)M$, so
$a = x_1a' \in (x_1)M = \im m_{x_1}$. Now, $x_1x_2a'=0$ so that
$x_2a'=0$, thus $a' \in \ker m_{x_2}$. Thus we see that
$m_{x_1}$ maps $\ker m_{x_2}$ onto
$\ker m_{x_2}$. It is also injective by definition. Thus,
$\ker m_{x_2}=(x_1)\ker m_{x_2}$, so that Nakayama version 2 applies. Thus
$\ker m_{x_2}=0$.

Finally, to show the general case recall that any permutation can be written as
a composition of transpositions. Thus it suffices to show the case where
$x_1, \dots x_i, x_{i+1}, \dots x_n$ are transposed to give $x_1, \dots x_{i+1}, x_{i},
    \dots x_n$. This
follows by the above argument applied to $M/(x_1, \dots x_{i-1})M$. \qed

\section{8.4.G}
Let $f \after g: Z \to U \to X$ be a locally closed embedding. We follow the hints
by first noting that since $U$ is open in
$X$ any open subset of $U$ is itself
open, hence we may replace $X$ with $U$
and consider only a closed embedding. Now, choose an affine open containing
$\p$ and the same argument lets us reduce to the affine
case. Let us set $Z=\Spec A/I$ and let $I_{\p}=(x_1, \dots, x_n)$ where
the $x_i$ form a regular sequence for
$A_{\p}$.

We note that if $x, y$ is a regular sequence and
$s$ is a unit, then $sx, y$ and
$x, sy$ are both regular sequences. Indeed, multiplication by a
unit doesn't change the ideal generated by a set of elements. Likewise, if
$asx=0$ then clearly $ax=0$, and likewise for
$y$. Thus we may take the $x_i$ to lie in
$A$.

Lemma: Suppose we have any finitely generated module $M$
and $M_{\p}=0$. Then we may choose generators
$x_1, \dots, x_n$ for $M$ as well as annihilators
$s_1, \dots, s_n \in S = A \setminus \p$. If $s=\prod_i s_i$ then clearly
$M_s=0$.

Next we must prove that $I_{\p}=(x_1, \dots,
    x_n)_{\p}$ implies $I=(x_1, \dots, x_n)$
on an open set. We do this by following the hint. Let $I, J \subset M$
where $M$ is an arbitrary module over a Noetherian ring
$A$. Suppose $I_{\p}=J_{\p}$ for a prime
$\p$. Then there exists an $a \in A$ such that
$I_a=J_a$. Indeed, suppose $(I \intersect J)_a=I_a$ and
$(I \intersect J)_b=J_b$. Then clearly $I_{ab}=J_{ab}$. Thus we reduce
to the case where $I \subset J$. Now, apply the above lemma to
$J/I$.

Let $x_1, \dots, x_n$ be set whose images are a regular sequence that
generates the ideal $I_{\p}$. Then $x_1, \dots, x_n$ is a
regular sequence on $I_a$ for some $a$.
Indeed, recall that since $A$ is noetherian, we have that
$\ker m_{x_1}$ (with notation as in 8.4.F) is a finitely generated
ideal. Choose generators $\ker m_{x_1}=(a_1, \dots, a_k)$. In $A_{\p}$ we
have that $a_i=0$ so that $s_ia_i=0$ in
$A$ for some $s_i \in A \setminus \p$. Now, then, we may
simply take the product of the $s_i$, which clearly
annihilates $\ker m_{x_1}$. We may continue likewise for the other
$x_i$ to prove the desired result.

\qed

\section{8.4.H}
\begin{itemize}
    \item[$\implies$] By 8.4.G, for every point $p$ we have an open set
        $U$ on which the closed subscheme is given by a single non
        zero divisor. \qed
    \item[$\impliedby$] Follows immediately from 8.4.D. \qed
\end{itemize}

\end{document}
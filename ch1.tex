\documentclass{article}
\usepackage[utf8]{inputenc}

\usepackage{mathtools}
\usepackage{amsthm}
\usepackage{amssymb}
\usepackage{dsfont}
\usepackage{array}   % for \newcolumntype macro
\usepackage{calligra}
\usepackage{tikz-cd}
\usepackage{mathrsfs}

\newcolumntype{L}{>{$}l<{$}} % math-mode version of "l" column type

\newcommand{\Rho}{\mathrm{P}}

\newcommand{\iso}{\simeq}
\newcommand{\after}{\circ}
\newcommand{\catname}[1]{\mathbf{#1}}
\newcommand{\intersect}{\cap}
\newcommand{\union}{\cup}
\newcommand{\Intersect}{\bigcap}
\newcommand{\Union}{\bigcup}
\newcommand{\wa}[1]{\langle #1 \rangle}
\newcommand{\ds}[1]{\mathds{#1}}
\newcommand{\Z}{\mathds{Z}}
\newcommand{\Q}{\mathds{Q}}
\newcommand{\R}{\mathds{R}}
\newcommand{\C}{\mathds{C}}
\newcommand{\p}{\mathfrak{p}}
\newcommand{\q}{\mathfrak{q}}
\newcommand{\ai}{\mathfrak{a}}
\newcommand{\bi}{\mathfrak{b}}
\newcommand{\m}{\mathfrak{m}}
\newcommand{\defeq}{\vcentcolon=}

\DeclareMathOperator{\Hom}{\mathscr{H}\text{\kern -3pt {\calligra\large om}}\,}
\DeclareMathOperator{\Fsh}{\mathscr{F}}
\DeclareMathOperator{\Gsh}{\mathscr{G}}
\DeclareMathOperator{\Hsh}{\mathscr{H}}
\DeclareMathOperator{\Osh}{\mathscr{O}}
% \DeclareMathOperator{\ker}{ker}
\DeclareMathOperator{\coker}{coker}
\DeclareMathOperator{\im}{im}
\DeclareMathOperator{\Spec}{Spec}
\DeclareMathOperator{\V}{V}
\DeclareMathOperator{\D}{D}


\title{Chapter 1 Problems}
\author{Josiah Bills}
\date{July 2020}

\begin{document}

\maketitle

\section{Transitivity of Localization}
Let $a/b \sim c/d \sim e/f$. Then there exist $s_1, s_2$ such that
$s_1(ad-bc)=0$ and $s_2(cf-ed)=0$. Thus
$s_2(bcf-edb)=s_2(s_1daf-s_1bed)=s_2ds_1(af-eb)$. Since $s_2ds_1 \in S$ the result follows.

\section{1.3.C}
\begin{itemize}
    \item[$\leftarrow$] Suppose $ab=0$, $0 \not = a \in S$ and $b \not = 0$. Then $a(b-0)=0$ so that $b/1=0/1$. \qed
    \item[$\rightarrow$] Suppose $x/1=y/1$. Then $s(x-y)=0$. Since $s$ is not a zero divisor, $x=y$. \qed
\end{itemize}

\section{1.3.D}
Let $\phi: A \to B$ be such that $\phi(S)$ contains only invertible elements.
Define $f : S^{-1}A \to B$ by $f(a/b)=\phi(a)\phi(b)^{-1}$. The map is well defined,
and $f(a/1)=\phi(a)$ as desired.

\end{document}

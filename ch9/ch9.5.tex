\documentclass{article}
\usepackage[utf8]{inputenc}

\usepackage{mathtools}
\usepackage{amsthm}
\usepackage{amssymb}
\usepackage{dsfont}
\usepackage{array}   % for \newcolumntype macro
\usepackage{calligra}
\usepackage{tikz-cd}
\usepackage{mathrsfs}
\usepackage{quiver}
\usepackage[shortlabels]{enumitem}

\newcolumntype{L}{>{$}l<{$}}
% math-mode version of "l" column type

\newcommand{\Rho}{\mathrm{P}}

\newcommand{\iso}{\simeq}
\newcommand{\after}{\circ}
\newcommand{\catname}[1]{\mathbf{#1}}
\newcommand{\intersect}{\cap}
\newcommand{\union}{\cup}
\newcommand{\Intersect}{\bigcap}
\newcommand{\Union}{\bigcup}
\newcommand{\wa}[1]{\langle #1 \rangle}
\newcommand{\bb}[1]{\mathbb{#1}}
\newcommand{\A}{\mathbb{A}}
\newcommand{\Z}{\mathbb{Z}}
\newcommand{\Q}{\mathbb{Q}}
\newcommand{\R}{\mathbb{R}}
\newcommand{\C}{\mathbb{C}}
\newcommand{\F}{\mathbb{F}}
\newcommand{\p}{\mathfrak{p}}
\newcommand{\q}{\mathfrak{q}}
\newcommand{\ai}{\mathfrak{a}}
\newcommand{\bi}{\mathfrak{b}}
\newcommand{\m}{\mathfrak{m}}
\newcommand{\defeq}{\vcentcolon=}

\DeclareMathOperator{\id}{id}
\DeclareMathOperator{\Hom}{\mathscr{H}\text{\kern -3pt
{\calligra\large om}}\,}
\DeclareMathOperator{\Fsh}{\mathscr{F}}
\DeclareMathOperator{\Gsh}{\mathscr{G}}
\DeclareMathOperator{\Hsh}{\mathscr{H}}
\DeclareMathOperator{\Osh}{\mathscr{O}}
% \DeclareMathOperator{\ker}{ker}
\DeclareMathOperator{\coker}{coker}
\DeclareMathOperator{\im}{im}
\DeclareMathOperator{\Spec}{Spec}
\DeclareMathOperator{\Proj}{Proj}
\DeclareMathOperator{\V}{V}
\DeclareMathOperator{\D}{D}

\newcommand*{\rationalto}[1][]{\mathbin{\tikz [baseline=-0.25ex,-latex, dashed,->,densely dashed,#1] \draw [#1] (0pt,0.5ex) -- (1.3em,0.5ex);}}%


\title{Chapter 9.5 Problems}
\author{Josiah Bills}
\date{March 2021}

\begin{document}

\maketitle

\section{9.5.A}
We immediately see that
\begin{align*}
    k(u) \otimes_{k(u^p)} k(u) & \cong k(u^p)[x]/(x^p-u^p) \otimes_{k(u^p)} k(u) \\
                               & \cong k(u)[x]/(x^p-u^p)
\end{align*}
In particular, $x-u$ is nilpotent.

We generalize: let $K$ be a non-trivial purely inseparable
extension of an imperfect field $k$. Then
$\Spec K \to \Spec k$ is reduced, but $\Spec K \otimes_{\Spec k} \Spec K$ isn't. Indeed,
let $x \in K \setminus k$ and let $x^q \in k$. Then
$(x \otimes 1 - 1 \otimes x)^q=x^q \otimes 1 - 1 \otimes x^q=x^q\otimes 1 - x^q
    \otimes 1 = 0$. \qed

\section{9.5.B}
Compose the relevant pullback squares. \qed

\section{9.5.C}
Let $\R \to K$ be an algebraic closure of $\R$.
Then we have
\begin{align*}
    \C \otimes_{\R} K & \cong \R[x]/(x^2+1) \otimes_{\R} K \\
                      & \cong K[x]/(x^2+1)                 \\
                      & \cong K[x]/(x+i) \times K[x]/(x-i)
\end{align*}
\qed

\section{9.5.D}
\begin{enumerate}[a.]
    \item 9.5.A \qed
    \item The example of 9.5.C is disconnected since it is the disjoin union of two
          points. \qed
    \item The example 9.5.C suffices for this also. \qed
\end{enumerate}

\section{9.5.E}

\section{9.5.F}
\begin{enumerate}[a.]
    \item We note that free modules are flat, hence tensor product with
          $E$ is exact, hence preserves injectivity. \qed
    \item We recall that reducedness is affine-local, thus we may reduce to affine
          $X$. We then note that a subring of a reduced ring is reduced
          then apply (a).
    \item By 9.5.L we may reduce to the affine case. Note that a subring of a irreducible
          ring is irreducible then apply (a). \qed
    \item The inverse image of a disconnected space under a surjective continuous map is
          disconnected. Indeed, let $f: X \to Y$ be surjective and continuous
          and let $Y=U \union V$ with $U \intersect V = \emptyset$. By surjectivity
          $f^{-1}(Y)=X$ and $f^{-1}(U) \intersect
              f^{-1}(V)=f^{-1}(U \intersect V) = \emptyset$. To see the result, recall that
          (9.4.D) the map $X_E \to X$ is surjective. \qed
\end{enumerate}

\section{9.5.4}
We first reduce to the case of reduced affine schemes over
$k$. Then we reduce to showing $\pi: A^n_k \to A^m_k$ is
open (for $m < n$). By Chevalley's theorem the image of any open
is constructible. By 7.4.C(b) the image of an open $U$ is
open iff it is closed under generization. Let $p \in U$,
$S=\pi(U)$, $\pi(p)=q$. We show $\Spec k[x_1, \dots, x_n]_p \to \Spec
    k[x_1, \dots, x_m]_q$
is surjective. It suffices to show \[\begin{tikzcd}
        {k[x_1, \dots, x_m]_q}                        & {\kappa(r)}
        \\{k[x_1, \dots,x_n]_p} &
        {k[x_1, \dots,x_n]_p \otimes \kappa(r)}
        \arrow["f", from=1-1, to=1-2]
        \arrow["g"', from=1-1, to=2-1]
        \arrow[from=1-2, to=2-2]
        \arrow[from=2-1, to=2-2]
    \end{tikzcd}\] is non-zero.
Consider $A \otimes_C B$ Indeed, the pure tensor $a \otimes b = 0$
iff $a=0$ or $b=0$ or $a=cx$
where $c \in C$ and $ca=0$. Consider
$1 \otimes 1 \in k[x_1, \dots,x_n]_p \otimes \kappa(r)$. Suppose $a \in x_1, \dots, x_m]_q$ be such that
$f(a)=0$. Then $a \in r$ so that
$g(a) \in p$. Hence there does not exist an $x$
such that $g(a)x \neq 1 \in k[x_1, \dots,x_n]_p$. Next we show that $g$ is
injective. Indeed, the map $k[x_1, \dots, x_m] \to k[x_1, \dots, x_n]$ is injective and localization
is exact, so $k[x_1, \dots, x_m]_q \to S^{-1}k[x_1,
    \dots, x_n]$ is injective. The latter has no zero
divisors, so $S^{-1}k[x_1, \dots, x_n] \to
    k[x_1, \dots, x_n]_p$ is also injective. Thus, if
$g(a)=0$ then $a=0$ so that
$f(a)=0$. \qed

\section{Note on 9.5.6}
The book doesn't show the following critical result: Let $S, A$
be $k$-algebras and let $S' \subseteq S$ be a
sub-$k$-algebra. Let $f \in S'$. Let
$\p \in \Spec A$ be in the image of the map $(\Spec S' \otimes_k A)_f \to \Spec A$.
Consider the following diagram: \[\begin{tikzcd}
        {\operatorname{Spec} (S' \otimes_k A)_f\otimes_A\kappa(\mathfrak{p})}                       & {\operatorname{Spec} (S' \otimes_k A)_f} &
        {\operatorname{Spec} (S \otimes_kA)_f}                                                      &
        {}
        \\{\operatorname{Spec} (S' \otimes_k A)\otimes_A\kappa(\mathfrak{p})} & {\operatorname{Spec} S'\otimes_kA}       &
        {\operatorname{Spec} S\otimes_kA}
        \\{\operatorname{Spec} \kappa(\mathfrak{p})}                          &                                          &
        {\operatorname{Spec} A}
        \arrow[from=2-2, to=3-3]
        \arrow[from=2-3, to=3-3]
        \arrow[from=2-3, to=2-2]
        \arrow[from=1-2, to=2-2]
        \arrow[from=1-3, to=2-3]
        \arrow[from=1-3, to=1-2]
        \arrow[from=3-1, to=3-3]
        \arrow[from=1-1, to=1-2]
        \arrow[from=1-1, to=2-1]
        \arrow[from=2-1, to=2-2]
        \arrow[from=2-1, to=3-1]
    \end{tikzcd}\] Then the image of the
map $(\Spec S' \otimes_k A)_f \to \Spec A$ is the same as the image of $(\Spec S \otimes_k A)_f \to \Spec A$.
This theorem is Stacks 037F. Note first that $S' \otimes_k \kappa(\p) \to S \otimes_k \kappa(\p)$ is
injective since the component maps are injective and $k$ is
a field. Now,
\begin{align*}
    (S' \otimes_k A)_f \otimes_A \kappa(\p) & \cong ((S' \otimes_k A) \otimes_A \kappa(\p))_f \\
                                            & \cong (S' \otimes_k (A \otimes_A \kappa(\p)))_f \\
                                            & \cong (S' \otimes_k \kappa(\p))_f
\end{align*}
and similarly for $S$. Now, since $\p$ is
in the image of $(\Spec S' \otimes_k A)_f \to \Spec A$, we find that $(S' \otimes_k \kappa(\p))_f$ is
non-trivial, hence $S' \otimes_k \kappa(\p)$ is nontrival and
$f$ is non-nilpotent. Thus $S \otimes_k \kappa(\p)$ is
nontrival and $f$ is non-nilpotent, so that it has a point.
\qed

\section{Note on 9.5.7}
The following result is key: Let $k \to K$ be a purely inseparable
extension of fields of characteristic $p$. Consider the
following diagram: \[\begin{tikzcd}
        k & K             \\
        A & {A\otimes_kK}
        \arrow[from=1-2, to=2-2] \arrow[from=2-1, to=2-2] \arrow[from=1-1, to=2-1]
        \arrow[from=1-1, to=1-2]
    \end{tikzcd}\] The prime ideals of
$A\otimes_kK$ inject into $A$. Indeed, let
$\p, \q \subset A\otimes_kK$ and let them inverse image into the prime ideal
$\mathfrak{r}$. Let $f \in \p$. Then we see that there
exists an $n$ such that $f^{p^n} \in A$ (take
$n$ to be the largest power of a coefficient of
$f$ such that the coefficients go to
$k$). Now, $f^{p^n} \in \mathfrak{r}$ which means that
$f^{p^n} \in \q$. But $\q$ is prime so
$f \in \q$.

\section{9.5.H}
Let $X$ be a topological space and let
$p \in X$. Consider the set of connected subsets containing
$p$. Let $\{S_i\}_{i \in I}$ be a chain in this poset and
consider $S=\Union_i S_i$. Let $S=U \union V$. Then they
intersect on some element of the chain, hence they intersect. Hence
$S$ is connected. Hence by Zorn's lemma,
$p$ is contained in a maximal connected set.

Next we show that the closure of a connected subspace is connected. Indeed, let
$S^{\text{C}} \intersect (U \union V) =
    S^{\text{C}}$ where $S$ is an arbitrary subset
and $S^{\text{C}} \intersect U \intersect V = \emptyset$. Then these can't intersect in
$S$ either. Since $S$ is connected, we
must have $S \subseteq U \intersect S^{\text{C}}$ or $S \subseteq V$. WLOG assume it is
$U$. Then $V \subseteq \overline{S^{\text{C}}}$ by definition of closure,
hence $U=S^{\text{C}}$.

\section{9.5.I}
Let $S \subseteq Y$ be a connected component. Then
$S'=\phi^{-1}(S)$ is connected. Indeed, suppose otherwise. Then we have a
disconnection $S'=U \union V$ with $U \intersect V = \emptyset$. Now,
$\phi$ is open, hence $\phi(U), \phi(V)$ are open and
cover $S$. Hence $\phi(U) \intersect \phi(V) \neq \emptyset$. Let
$x \in \phi(U) \intersect \phi(V)$. Then $U$ and
$V$ disconnect $\phi^{-1}(x)$, contradiction.

Now, suppose $S \subseteq X$ a connected component. Then
$\phi(S)$ is connected. Indeed, note that $\phi$
is surjective, hence the inverse image of a disconnected set is disconnected.
Let $x \in \phi(S)$. Then $\phi^{-1}(x) \subseteq S$, since it is
connected. Thus $\phi^{-1}(\phi(S)) = S$, hence $\phi(S)$ is
connected. \qed

\section{9.5.J}
We first note that if $X=U \coprod V$ then $\Osh_X(X)=\Osh_X(U) \times \Osh_X(V)$.
Indeed, any pair of sections over $U$ and
$V$ agree on the overlap and thus glue. Now,
$(1,0)$ will suffice as a nontrivial nilpotent. Conversely, if
$e^2=e$ then $\V(e) \intersect \V(1-e)
    =\V(1)= \emptyset$ and
$\V(e) \union
    \V(1-e)=\V(e^2-e)=\V(0)=X$. $1-e$ is also idempotent (and not
0), hence it isn't a unit, so $\V(1-e)$ is nontrivial. All this
together shows that the $X=\overline{\V(e)} \coprod \overline{\V(1-e)}$. \qed

\section{9.5.L}
\begin{itemize}
    \item[$\implies$] Let $U_1=X$. \qed
    \item[$\impliedby$] Irreducible is equivalent to the statement that every
          pair of non-empty open subsets meets. We are given a cover of non-empty opens,
          which all meet. Let $V, W$ be arbitrary non-empty opens. We
          show they intersect. Indeed, choose $i, j$ such that
          $U_i \intersect V$ and $U_j \intersect W$ are non-empty. Then
          $U_i \intersect U_j \intersect V \intersect W$ is nonempty since $U_i \intersect U_j$ is nonempty.
          \qed
\end{itemize}

\end{document}
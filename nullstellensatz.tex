\documentclass{article}
\usepackage[utf8]{inputenc}

\usepackage{mathtools}
\usepackage{amsthm}
\usepackage{amssymb}
\usepackage{dsfont}
\usepackage{array}   % for \newcolumntype macro
\usepackage{calligra}
\usepackage{tikz-cd}
\usepackage{mathrsfs}

\newcolumntype{L}{>{$}l<{$}} % math-mode version of "l" column type

\newcommand{\Rho}{\mathrm{P}}

\newcommand{\iso}{\simeq}
\newcommand{\after}{\circ}
\newcommand{\catname}[1]{\mathbf{#1}}
\newcommand{\intersect}{\cap}
\newcommand{\union}{\cup}
\newcommand{\Intersect}{\bigcap}
\newcommand{\Union}{\bigcup}
\newcommand{\wa}[1]{\langle #1 \rangle}
\newcommand{\ds}[1]{\mathds{#1}}
\newcommand{\Z}{\mathds{Z}}
\newcommand{\Q}{\mathds{Q}}
\newcommand{\R}{\mathds{R}}
\newcommand{\C}{\mathds{C}}
\newcommand{\p}{\mathfrak{p}}
\newcommand{\q}{\mathfrak{q}}
\newcommand{\ai}{\mathfrak{a}}
\newcommand{\bi}{\mathfrak{b}}
\newcommand{\m}{\mathfrak{m}}
\newcommand{\defeq}{\vcentcolon=}

\DeclareMathOperator{\Hom}{\mathscr{H}\text{\kern -3pt {\calligra\large om}}\,}
\DeclareMathOperator{\Fsh}{\mathscr{F}}
\DeclareMathOperator{\Gsh}{\mathscr{G}}
\DeclareMathOperator{\Hsh}{\mathscr{H}}
\DeclareMathOperator{\Osh}{\mathscr{O}}
% \DeclareMathOperator{\ker}{ker}
\DeclareMathOperator{\coker}{coker}
\DeclareMathOperator{\im}{im}
\DeclareMathOperator{\Spec}{Spec}
\DeclareMathOperator{\V}{V}
\DeclareMathOperator{\D}{D}


\title{Proof of the Nullstellensatz}
\author{Josiah Bills}
\date{March 2021}

\begin{document}

\maketitle

\begin{theorem}
    Let $k$ be a field. Let $R=k[x_1, \dots, x_n]/\m = k[a_1, \dots, a_n]$ be a finitely
    generated $k$-algebra. Suppose $R$ is a
    field. Then each $a_i$ is algebraic over $k$.
\end{theorem}
\begin{proof}
    Recall that the difference between an algebraic element and an integral element
    is that the former satsifies an arbitrary polynomial equation, while the latter
    satsifies a monic polynomial equation.

    Note that $k[a_1, \dots, a_j] \subseteq R$ is still a finitely generated
    $k$-algebra (though not necessarily a field). Note also that
    $R=k(a_1, \dots, a_j)[a_{j+1}, \dots, a_n]$ and that $R$ is finitely generated
    over $k(a_1, \dots, a_j)$. Finally, note that $x_n$ is
    algebraic over $k(x_1, \dots, x_{n-1})$.

    Suppose that $a_2, \dots, a_n$ are algebraic over $k(a_1)$
    and $a_1$ is not algebraic over $k$. Then
    $a_2, \dots a_n$ is algebraic over $k[a_1]$ by clearing
    denominators. Localizing at these denominators, we get that
    $a_2,\dots a_n$ are integral over $A=k[a_1]_r \subseteq R$ for some
    $r \in k[a_1]$. Note that $A$ is clearly finitely
    generated over $k$. Since $R$ is a field
    and the $a_i$ are integral over $A$, we
    must have that $A$ is a field. But clearly
    $A \subseteq k(a_1)$, so $A=k(a_1)$. But the field of fractions
    of a PID with infinitely many primes is not finitely generated (a PID is a UFD,
    so we note that the LCM of the denominators has finitely many primes). This is
    a contradiction, so $a_1$ must be algebraic.

    Suppose $x_i, \dots x_n$ are algebraic over $k(x_1, \dots x_{i-1})=k(x_1, \dots
        x_{i-2})(x_{i-1})$. By
    the above lemma, we must have $x_{i-1}, \dots, x_n$ are algebraic over
    $k(x_1, \dots x_{i-2})$. The result follows by induction. \qed
\end{proof}

\end{document}
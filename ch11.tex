\documentclass{article}
\usepackage[utf8]{inputenc}

\usepackage{mathtools}
\usepackage{amsthm}
\usepackage{amssymb}
\usepackage{dsfont}
\usepackage{array}   % for \newcolumntype macro
\usepackage{calligra}
\usepackage{tikz-cd}
\usepackage{mathrsfs}
\usepackage{quiver}
\usepackage[shortlabels]{enumitem}

\newcolumntype{L}{>{$}l<{$}}
% math-mode version of "l" column type

\newcommand{\Rho}{\mathrm{P}}

\newcommand{\iso}{\simeq}
\newcommand{\after}{\circ}
\newcommand{\catname}[1]{\mathbf{#1}}
\newcommand{\intersect}{\cap}
\newcommand{\union}{\cup}
\newcommand{\Intersect}{\bigcap}
\newcommand{\Union}{\bigcup}
\newcommand{\wa}[1]{\langle #1 \rangle}
\newcommand{\bb}[1]{\mathbb{#1}}
\newcommand{\A}{\mathbb{A}}
\newcommand{\Z}{\mathbb{Z}}
\newcommand{\Q}{\mathbb{Q}}
\newcommand{\R}{\mathbb{R}}
\newcommand{\C}{\mathbb{C}}
\newcommand{\F}{\mathbb{F}}
\newcommand{\p}{\mathfrak{p}}
\newcommand{\q}{\mathfrak{q}}
\newcommand{\ai}{\mathfrak{a}}
\newcommand{\bi}{\mathfrak{b}}
\newcommand{\m}{\mathfrak{m}}
\newcommand{\defeq}{\vcentcolon=}

\DeclareMathOperator{\id}{id}
\DeclareMathOperator{\Hom}{\mathscr{H}\text{\kern -3pt
{\calligra\large om}}\,}
\DeclareMathOperator{\Fsh}{\mathscr{F}}
\DeclareMathOperator{\Gsh}{\mathscr{G}}
\DeclareMathOperator{\Hsh}{\mathscr{H}}
\DeclareMathOperator{\Osh}{\mathscr{O}}
% \DeclareMathOperator{\ker}{ker}
\DeclareMathOperator{\coker}{coker}
\DeclareMathOperator{\im}{im}
\DeclareMathOperator{\Spec}{Spec}
\DeclareMathOperator{\Proj}{Proj}
\DeclareMathOperator{\V}{V}
\DeclareMathOperator{\D}{D}

\newcommand*{\rationalto}[1][]{\mathbin{\tikz [baseline=-0.25ex,-latex, dashed,->,densely dashed,#1] \draw [#1] (0pt,0.5ex) -- (1.3em,0.5ex);}}%


\title{Chapter 11 Problems}
\author{Josiah Bills}
\date{January 2022}

\begin{document}

\maketitle

\section{11.1.A}
Trivial

\section{11.1.B}
\begin{lemma}
    Let $X$ be a topological space. Let $U \subseteq X$ be
    open. Then the set of irreducible closed subsets of $U$ is in
    bijection with the irreducible closed subsets of $X$ meeting
    $U$.
\end{lemma}
\begin{proof}
    The map $U \to X$ is continuous, hence preserves irreduciblity.
    Since it is an open embedding, the inverse image also preserves irreduciblity.
    The desired bijection is given by taking an irreducible closed subset of
    $U$ to its closure, and the inverse takes an irreducible
    closed subset of $X$ to its intersection with
    $U$. The former map is clearly a right inverse of the latter.
    To that it is also a left inverse, suppose that $Z \subseteq U = Z' \intersect U$. We wish
    to show that $Z^\text{c}=Z'$. Note that $Z^\text{c} \subseteq U^\text{c}$. Then
    $Z' - Z^\text{c}$ is open in $Z'$, but so is
    $Z$. Since $Z$ is nonempty and
    $Z'$ is irreducible, we must have $Z' - Z^\text{c}$ is
    empty.
\end{proof}

Note that the result as stated is not true when $n=\infty$. In this
case, we drop the requirement that an affine must have the same dimension as
$X$.

Now, let $n=\dim X$ and let $U_i$ be an arbitrary
open affine cover. Then if $Z_0 \subset Z_1 \subset \dots Z_j$ is a sequence of irreducible
closed subsets, $j \leq n$, then at least one
$U_i$ meets $Z_1$. By the above lemma, this
gives us a chain of irreducible closed subsets of $U_i$ of
length $j$. Hence $\sup_i \dim U_i \leq n$. In particular,
every open cover has a cover of dimension $\leq n$. If
$n$ is finite, or if we have an infinite increasing sequence
of irreducible closed sets, or if $X$ is quasicompact, or if
we have an infinite decreasing sequence of irreducible closed sets with
nonempty limit, then the above argument gives us an open with
$\dim U_i = n$.

Now, suppose $U_i$ is a cover with $\dim U_i \leq n$,
with equality on $U_j$. Then by the above, we get a chain of
length $n$ in $X$, as desired. \qed

\section{11.1.C}
Recall that a Noetherian space has a finite number of irreducible components.
\qed

\section{11.1.D}
Following the hint, we reduce to the case where $A$ is an
integral extension of $k$. In this case, we know that
$A$ is a field, since an integral extension of a field is a
field. We proved this in 7.2, but for completeness let $0 = \sum_{i=1}^n a_ix^i+a_0$
be an integral relation for $x$. Then $x\sum_{i=1}^{n-1}-a_ix^i/a_0=1$
so that $x$ is invertible. \qed

\section{11.1.E}
By the going up theorem, $\pi$ is surjective. Hence any chain
downstairs gives a chain upstairs. Now, suppose $\p_i$ is a
chain in $A$. Then $\pi(\p_i)$ is a chain in
$B$ of at most the same length. If $\pi(\p_i) = \pi(\p_{i+1})$
then $\p_i \subset \p_{i+1}$, contradicting the zero dimensionality of the fiber
(11.1.D). \qed

\section{11.1.F}
\section{11.1.G}
\begin{enumerate}[a.]
    \item Note that the reduction of a scheme has the same dimension as the scheme, since
          they are homeomorphic. Hence we may reduce to the case where
          $A$ is reduced. Now note that it suffices to show that
          dimension of each irreducible component (regarded as a closed subscheme) is
          preserved by the given pullback. Hence we may show that pullbacks of integral
          domains preserve dimension. Choose a minimal prime $\p$ of
          $A'=A\otimes_k K$. Note $A'$ is a free
          $A$-module (we proved this in 5.4.M). Let
          $a$ be in $\ker A \to A'/\p$. Then
          $a \otimes 1 \in \p$ is a zero divisor, hence since $x \mapsto ax \otimes 1$ is
          injective, we must have $a=0$. Hence $A \to A'/\p$ is
          an integral extension, so 11.1.E applies. \qed
    \item The closure of the image of a irreducible component is irreducible
\end{enumerate}
\end{document}
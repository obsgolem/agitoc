\documentclass{article}
\usepackage[utf8]{inputenc}

\usepackage{mathtools}
\usepackage{amsthm}
\usepackage{amssymb}
\usepackage{dsfont}
\usepackage{array}   % for \newcolumntype macro
\usepackage{calligra}
\usepackage{tikz-cd}
\usepackage{mathrsfs}
\usepackage{quiver}
\usepackage[shortlabels]{enumitem}

\newcolumntype{L}{>{$}l<{$}}
% math-mode version of "l" column type

\newcommand{\Rho}{\mathrm{P}}

\newcommand{\iso}{\simeq}
\newcommand{\after}{\circ}
\newcommand{\catname}[1]{\mathbf{#1}}
\newcommand{\intersect}{\cap}
\newcommand{\union}{\cup}
\newcommand{\Intersect}{\bigcap}
\newcommand{\Union}{\bigcup}
\newcommand{\wa}[1]{\langle #1 \rangle}
\newcommand{\bb}[1]{\mathbb{#1}}
\newcommand{\A}{\mathbb{A}}
\newcommand{\Z}{\mathbb{Z}}
\newcommand{\Q}{\mathbb{Q}}
\newcommand{\R}{\mathbb{R}}
\newcommand{\C}{\mathbb{C}}
\newcommand{\F}{\mathbb{F}}
\newcommand{\p}{\mathfrak{p}}
\newcommand{\q}{\mathfrak{q}}
\newcommand{\ai}{\mathfrak{a}}
\newcommand{\bi}{\mathfrak{b}}
\newcommand{\m}{\mathfrak{m}}
\newcommand{\defeq}{\vcentcolon=}

\DeclareMathOperator{\id}{id}
\DeclareMathOperator{\Hom}{\mathscr{H}\text{\kern -3pt
{\calligra\large om}}\,}
\DeclareMathOperator{\Fsh}{\mathscr{F}}
\DeclareMathOperator{\Gsh}{\mathscr{G}}
\DeclareMathOperator{\Hsh}{\mathscr{H}}
\DeclareMathOperator{\Osh}{\mathscr{O}}
% \DeclareMathOperator{\ker}{ker}
\DeclareMathOperator{\coker}{coker}
\DeclareMathOperator{\im}{im}
\DeclareMathOperator{\Spec}{Spec}
\DeclareMathOperator{\Proj}{Proj}
\DeclareMathOperator{\V}{V}
\DeclareMathOperator{\D}{D}

\newcommand*{\rationalto}[1][]{\mathbin{\tikz [baseline=-0.25ex,-latex, dashed,->,densely dashed,#1] \draw [#1] (0pt,0.5ex) -- (1.3em,0.5ex);}}%


\title{Chapter 3 Problems}
\author{Josiah Bills}
\date{July 2020}

\begin{document}

\maketitle

\section{3.2.A}
\begin{itemize}
    \item[a.] Let $(f) \supseteq (\epsilon^2)$ be prime. Then clearly $\epsilon \in (f)$.
          But $\epsilon$ is irreducible, so $(f)=(\epsilon)$. \qed
    \item[b.] \[\begin{tikzcd}
                  (0) \ar[dash, d] \\
                  (x)
              \end{tikzcd}\]
\end{itemize}

\section{3.2.C}
This is $A_{\overline{\Q}}^1$ with all roots of polynomials with irrational or
complex roots identified.

\section{3.2.E}
Over $\C(x)[y]$ the given polynomials clearly generate
$(1)$. Clearing the denominators in the linear combination
gives us the desired $h(x)$. Now proceed as described in the
book.

\section{3.2.F}
Consider a maximal ideal $\m \subset k[x_1, ... x_n] = R$. This ideal is finitely
generated (Hilbert's Basis Theorem). Thus $\frac{R}{\m}$ is a  finite
type field extension. By the Nullstellensatz, this is a finite extension, hence
algebraic. Since $k$ is algebraically closed,
$\frac{R}{\m}=k$. Let $\phi$ be the isomorphism. Then
$\phi(x_i-\phi(x_i))=0$ so that the ideal is of the required form.

\section{3.2.G}
Let $a, b \in R$ be nonzero and let $m_a(b)=ab$. Then
$m_a(b)$ is a vector space homomorphism. Now, since
$R$ is an integral domain $m_a$ is
injective. Since the domain and codomain are the same finite dimension,
$m$ is surjective, and thus an isomorphism.

\section{3.2.H}
The relevant ideal for the first case is $(x^2-2, y^2-2)$. In the second
case we have $(x^2+xy,y^2+xy)$. In the first case, we adjoin an
$x$ and a $y$ that are equal and we end
up with $\Q(\sqrt{2})$.

\section{3.2.I}
\begin{itemize}
    \item[a.] It must contain $x^2-y$
\end{itemize}

\section{3.2.J}
Let $pi_{\p} : \frac{R}{I} \ to
    \frac{\frac{R}{I}}{\p}$ be the projection onto the integral domain.
Composing this with the projection onto $\frac{R}{I}$ and we get the
desired prime ideal containing $I$.

\section{3.2.K}
Do the same thing as above, only compose with the inclusion into the
localization instead. The kernel doesn't meet $S$ since
$S$ become units, not zeroes.

\section{3.2.L}
The quotient gets rid of any mixed variable terms we might have. We also have
$xy=0$ so that $y=0/x=0$. From here the
isomorphism is obvious.

\section{3.2.M}
Same as in 3.2.J and 3.2.K.

\section{3.2.N}
Obvious from our proof of 3.2.M.

\section{3.2.O}
The map $\phi^* : \Spec\C[x] \to \Spec\C[y]$ is given by sending the prime ideal
$(x-a)$ to the ideal $(y-a^2)$. Indeed,
$f(y) \in (y-a^2)$ implies $\phi(f) \in (x^2-a^2) \subseteq (x-a)$. Thus the inverse image
of $(y-a)$ is given by $(x-\sqrt{a})$ and
$(x+\sqrt{a})$. \qed

\section{3.2.P}
Consider $\pi_I \after \phi$. Since $\phi(J) \subset I$ we have
$\ker(\pi_I \after \phi) \supseteq J$ so that $\pi_I \after \phi$ factors through
$\pi_J$.

\section{3.2.Q}
We apply Ch 9.3.B to describe the fiber. This gives us $\F_p \otimes_{\Z} \Z[x_1, \dots, x_n]$.
By 9.2.A, this is just $\F_p[x_1, \dots, x_n]$, as desired. If
$p=0$ then $\F_p=\Q$. \qed

\section{3.2.R}
Consider $\pi_{\p}$. $\p$ contains all
nilpotents, and thus it contains $I$. Since
$I$ goes to zero we must have that $\pi_{\p}$
factors through $\pi_I$. This gives us the desired
correspondence. \qed

\section{3.2.S}
Since $x$ is not nilpotent, the localization
$A_x$ is not the zero ring. Thus it has a maximal ideal. This
maximal ideal corresponds to a prime ideal not containing
$x$ by 3.2.K. \qed

\section{3.4.A}
This is just the statement that $xy \in (y,z)$ and
$yz \in (y, z)$.

\section{3.4.E}
\begin{align*}
    a^n           & \in \intersect_i I_i \implies a^n \in I_i \implies a \in \sqrt{I_i} \implies a \in \intersect_i \sqrt{I_i}           \\
    a^{\max(n_i)} & \in \intersect_i I_i \impliedby a^{n_i} \in I_i \impliedby a \in \sqrt{I_i} \impliedby a \in \intersect_i \sqrt{I_i}
\end{align*}
\qed

\section{3.4.G}
Closed sets are a finite number of points or galois conjugacy classes of
points. Every open set contains $(0)$ since no functions
vanish at $(0)$.

\section{3.4.H}
Define $\ai_{\phi}^{\text{e}}=(\phi(\ai))$ and $\bi_{\phi}^{\text{c}}=\phi^{-1}(\bi)$. If
$\phi^{-1}(\p)=\phi^{-1}(\q)=\ai$ then we have that $\ai_{\phi}^{\text{e}} \supseteq \p \intersect \q$.

We show that $\phi^{*,-1}(\V(\bi))=\V(\bi_\phi^{\text{e}})$. Let $\p \in \V(\bi) \subseteq \Spec{B}$. The inverse
image of $\p$ is a set of ideals whose inverse image under
$\phi$ is $\bi$. Thus, by the result above,
all the ideals contain $\bi_\phi^{\text{e}}$. The reverse inclusion is
trivial. \qed

We show another useful fact, namely that
\begin{equation*}
    \overline{\phi^*(\V(\bi))}=\V(\bi^{\text{c}}_\phi)
\end{equation*}

That $\phi^*(\V(\bi)) \subseteq \V(\bi^{\text{c}}_\phi)$ follows from definitions, thus
$\overline{\phi^*(\V(\bi))} \subseteq \V(\bi^{\text{c}}_\phi)$.

Let $\overline{\phi^*(\V(\bi))}=\V(\ai)$ for some radical ideal $\ai$.
Assume also WLOG that $\bi$ is radical.
\begin{align*}
    \V(\ai^\text{e}_\phi) & =\phi^{*,-1}(\V(\ai))                     \\
                          & = \phi^{*,-1}(\overline{\phi^*(\V(\bi))}) \\
                          & \supseteq \V(\bi)
\end{align*}
So we have $\ai^{\text{e}}_\phi \subseteq \bi$ so that $\ai \subseteq \ai^{\text{ec}}_\phi \subseteq
    \bi^{\text{c}}_\phi$.

\section{3.4.I}
\begin{itemize}
    \item [a.] Identify $\Spec\frac{B}{I}$ with $\V(I)$ by the obvious
          map. Identify $\Spec B_f$ with $\D(f)$ by the obvious
          map.
    \item [b.] Trivial by the above identifications.
\end{itemize}

\section{3.4.J}
We have $f \in \p$ for all $\p \in \V(I) = \V(\sqrt{I})$.

\section{3.5.B}
We have that $\sqrt{\ai}=(1) \implies \ai=(1)$. We then have that $\D(\ai)=\Spec(A)$
so that $\V(\ai)=\V((1))$ so $\ai=(1)$. The other direction
is trivial.

\section{3.5.E}
$g$ vanishes at no points of $A_f$, so it
must generate the unit ideal, i.e. it must be invertible. \qed

\section{3.6.A}
Let $\p \in \Spec(A_0)$. Then $\p \times (1)$ is prime since
$(xy,zw) \in \p \times (1)$ implies that either $(x,z)$ or
$(y,z)$ is in $\p \times (1)$. Every prime ideal is of
this form: if $I$ is a prime ideal and
$(x, 0) \in I$ and $(x,y) \not \in I$, then
$(1,0)(x, y) \in I$ but neither of those two can be in
$I$.

Thus we can map prime ideals of $\Spec(A_0)$ to prime ideals of
$A$. By our previous discussion, this is obviously an
isomorphism as well.

We show that the desired property holds of this map: let
$\p \in \D((1,0))$. Then $\p = \p_0 \times (1)$ for some
$\p_0$.

We show this map is continuous: every ideal of $A$ is of
the form $\ai_1 \times \ai_2$. A function in this set vanishes on every
prime ideal of the form $\p \times (1)$ where $\ai_1 \subset \p$.
Thus the inverse image of every closed set is closed. \qed

\section{3.6.B}
Let $U \subset X$ be a nonempty open. Then $U^{\text{c}} \union \overline{U}=X$.
The former isn't $X$ since $U$ is
nonempty, so the latter must be $X$.

Let $C, D \subseteq \overline{Z}$ be a closed cover of $\overline{Z}$. Then
they are a closed cover of $Z$ as well. So WLOG,
$C$ is disjoint from $Z$. But then
$D$ is a closed set which contains $Z$,
so it must contain $\overline{Z}$. \qed

\section{3.6.C}
\[V((0))=V(I) \union V(J)=V(IJ)\]
So since $A$ is an integral domain,
$IJ=(0)$. \qed

\section{3.6.D}
Write it as the union of two disjoint open subsets. These must be complements
of each other, so we have a union of closed sets. The result follows by
irreduciblity.

\section{3.6.E}
Consider $\C[x,y]/(xy)$ as the closed subset $\V(xy) \subset \mathds{A}^2_{\C}$.
Then it equals $\V(x) \union \V(y)$. \qed

\section{3.6.F}
\begin{enumerate}[a.]
    \item The isomorphism is given by sending $x \mapsto a^3$,
          $y \mapsto a^2b$, $z \mapsto ab^2$, and $w \mapsto b^3$.
          \qed
    \item We begin by noting that the condition on the minors follows by noting that we
          want the individual column vectors to be linearly dependent for all possible
          combinations. Consider the following equation over $k[x_0, \dots, x_n], n \geq 3$:
          \begin{equation*}
              \operatorname{rank}\begin{pmatrix*}
                  x_0 & x_1 & \dots & x_{n-1} \\ x_1 & x_2 & \dots &
                  x_{n}
              \end{pmatrix*} \leq 1
          \end{equation*}
          We will show that this is isomorphic to the subring of $k[a, b]$
          generated by the degree $n$ monomials. The
          $n=4$ case was proved in (a). The isomorphism is always given
          by sending $x_i$ to $a^{n-i}b^i$. We assume the
          result holds for all $i<n$. Given a relation of the form
          $x_ix_{j+1}-x_jx_{i+1}=0; i, j < n-1$ we may multiply by $a^2$ to leave this
          unchanged. Thus the $a*a^{n-i}b^i = ax_i$ satisfy the same relations as the
          $x_i$. Now, consider the following:
          \begin{align*}
              x_{n-1}x_{i+1}-x_nx_i & = ab^{n-1}a^{n-i-1}b^{i+1}-b^na^{n-i}b^i \\
                                    & = a^{n-i}b^{n+i}-a^{n-i}b^{n+i}          \\
                                    & =0
          \end{align*}
          This is exactly the vanishing of the determinants with the last column. \qed
\end{enumerate}

\section{3.6.G}
\begin{enumerate}[a.]
    \item The hint says it all.
    \item Take the cover ${\D(x_i)}_{i \in \mathbb{N}}$. Consider a finite union of these, WLOG
          assume it is $\union_{i=0}^n \D(x_i)$. Then $\V(x_0, \dots x_n) \setminus \m \subseteq \D(x_{n+1})$. Thus there is
          no finite subcover of this cover.
\end{enumerate}

\section{3.6.H}
\begin{itemize}
    \item [a.] Intersect every open cover with each of the elements of the finite union.
          Take the finite covers induced there. Surely the union of the corresponding
          covers covers the whole space and is finite.
    \item [b.] Take an open cover of the closed set. Each element of the cover is an open
          set in the space. Thus, unioning that with the complement of the closed set
          gives an open set. The union of these is an open cover of the space. A finite
          subcover of this is then a finite subcover of the closed set.
\end{itemize}

\section{3.6.I}
Let ${\p}=V(I)$. Then $I=\p$ and
$\p$ must be maximal. Also $V(\m)={\m}$. \qed

\section{3.6.J}
\begin{itemize}
    \item [a.] Now $A_f$ has a maximal ideal $\m$
          which corresponds to a prime ideal $\p$ in
          $A$. We get an injection $A/\p \to A_f/\m$. This
          realizes $A/\p$ as a finite dimensional
          $k$-algebra. Thus by 3.2.G $A/\p$ is a
          field. Thus $\p$ is a maximal ideal not containing
          $f$.
\end{itemize}

\section{3.6.K}
We require that $f$ and $g$ not differ
by a nilpotent instead of the condition on the nilradical. Then
$f-g$ is nonzero on some point, and thus on some open. Thus
by 3.6.J they are non-zero on some closed point.

\section{3.6.O}
For any chain $\{Z_a\}_{a \in I}$, write $\Union Z_a=C \union D$ where
$C$ and $D$ are closed. Then either
$C \intersect Z_i$ or $D \intersect Z_i$ are equal to
$Z_i$ for all $i \in I$. Then WLOG
$\Union C \intersect Z_a=\Union Z_a$. \qed

\section{3.6.P}
Follows from the general discussion later.

\section{3.6.Q}
We show that every irreducible space is connected. Indeed, irreducible is
equivalent to every pair of nonempty open sets meeting. But disconnected spaces
are the union of disjoint open sets, so must be reducible. Now, every point of
a connected component is contained in an irreducible component, so the
connected component must be the union of these.

Consider a connected component which meets a clopen. The clopen disconnects
that component if it doesn't contain it, so the clopen must be a union of
connected components.

A noetherian topological space has finitely many connected components: indeed,
by 3.6.15 it is a finite union of irreducible and thus connected sets. Taking
the connected components of each of these gives us the desired finite union.
Thus every connected component is clopen.

\section{3.6.S}
Consider the sequence $V(I_0) \subset V(I_1) \subset \dots$. Apply $I$ to
this to get a chain of ideals which necessarily stabilizes.

\section{3.6.T}
Let $U \subset X$ be open and let $U=\Union_i U_i$. Fix an
open set $U_i$. Then for all $j$,
$U_i \union U_j$ is open. Repeat this with $U_i \union U_j$
replacing $U_i$. Then we get an ascending chain of opens.
This must stabilize at some finite point, but $U$ is open,
so it must stabilize after covering $U$ with finitely many
open sets. \qed

\section{3.6.V}
Consider a finitely generated submodule $N \subset M$. Then
$N \intersect M'$ is finitely generated, as is $N/N \intersect M'$.
Choose $f \in N$. Then it can be written as a finite linear
combination of elements of $N/N \intersect M'$. Now just add a finite
linear combination of elements of $N \intersect M'$. \qed

\section{3.6.W}
Consider the projection onto the components. The image of any submodule is an
ideal and thus finitely generated. Take each generator of this ideal with zero
in the rest of the components. The set of such clearly generates the submodule.

\section{3.6.X}
$M$ is surjected upon by $A^{\oplus n}$ so clearly
every submodule is finitely generated by the image of a generating set for its
inverse image.

\end{document}

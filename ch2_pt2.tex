\documentclass{article}
\usepackage[utf8]{inputenc}

\usepackage{mathtools}
\usepackage{amsthm}
\usepackage{amssymb}
\usepackage{dsfont}
\usepackage{array}   % for \newcolumntype macro
\usepackage{calligra}
\usepackage{tikz-cd}
\usepackage{mathrsfs}
\usepackage{quiver}
\usepackage[shortlabels]{enumitem}

\newcolumntype{L}{>{$}l<{$}}
% math-mode version of "l" column type

\newcommand{\Rho}{\mathrm{P}}

\newcommand{\iso}{\simeq}
\newcommand{\after}{\circ}
\newcommand{\catname}[1]{\mathbf{#1}}
\newcommand{\intersect}{\cap}
\newcommand{\union}{\cup}
\newcommand{\Intersect}{\bigcap}
\newcommand{\Union}{\bigcup}
\newcommand{\wa}[1]{\langle #1 \rangle}
\newcommand{\bb}[1]{\mathbb{#1}}
\newcommand{\A}{\mathbb{A}}
\newcommand{\Z}{\mathbb{Z}}
\newcommand{\Q}{\mathbb{Q}}
\newcommand{\R}{\mathbb{R}}
\newcommand{\C}{\mathbb{C}}
\newcommand{\F}{\mathbb{F}}
\newcommand{\p}{\mathfrak{p}}
\newcommand{\q}{\mathfrak{q}}
\newcommand{\ai}{\mathfrak{a}}
\newcommand{\bi}{\mathfrak{b}}
\newcommand{\m}{\mathfrak{m}}
\newcommand{\defeq}{\vcentcolon=}

\DeclareMathOperator{\id}{id}
\DeclareMathOperator{\Hom}{\mathscr{H}\text{\kern -3pt
{\calligra\large om}}\,}
\DeclareMathOperator{\Fsh}{\mathscr{F}}
\DeclareMathOperator{\Gsh}{\mathscr{G}}
\DeclareMathOperator{\Hsh}{\mathscr{H}}
\DeclareMathOperator{\Osh}{\mathscr{O}}
% \DeclareMathOperator{\ker}{ker}
\DeclareMathOperator{\coker}{coker}
\DeclareMathOperator{\im}{im}
\DeclareMathOperator{\Spec}{Spec}
\DeclareMathOperator{\Proj}{Proj}
\DeclareMathOperator{\V}{V}
\DeclareMathOperator{\D}{D}

\newcommand*{\rationalto}[1][]{\mathbin{\tikz [baseline=-0.25ex,-latex, dashed,->,densely dashed,#1] \draw [#1] (0pt,0.5ex) -- (1.3em,0.5ex);}}%


\title{Chapter 2 Problems Pt 2}
\author{Josiah Bills}
\date{July 2020}

\begin{document}

\maketitle

\section*{2.4.A}
Let $\phi_U$ be the given map and let $\phi_U(f)=\phi_U(g)$. Then for all $p \in X$
there exists an open set $V_p \ni p$ such that $f|_{V_p}=g|_{V_p}$. The $V_p$ clearly
cover $X$, so identity applies, so $f=g$. \qed

\section*{2.4.B}
Let $A=\overline{supp(s)}$. Thus for all $x \in A$ there exists an open set $U$ where $s|_U \not = 0$.
Taking the union of all such open sets over all the points in $A$ gives us $A$ itself, so $A$ is open. \qed

\section*{2.4.C}
The $U_p$ cover $U$. The $\widetilde{s_p}$ agree on overlaps since they have the same germs on those overlaps
(apply 2.4.A). Thus they glue. \qed

\section*{2.4.D}
The hypotheses mean the following diagram commutes.

\[
    \begin{tikzcd}
                                                                              & \mathscr{G}(U) \ar[dr, hook, "\iota_1"] \\
        \mathscr{F}(U) \ar[r] \ar[ur, "\phi_{1U}"] \ar[dr, swap, "\phi_{2U}"] &
        \prod_{p \in U} \mathscr{F}_p \ar[r]
                                                                              & \prod_{p \in U} \mathscr{G}_p           \\
                                                                              & \mathscr{G}(U) \ar[ur, hook, "\iota_2"]
    \end{tikzcd}
\]
Note that the image of the two injective maps is identical by commutativity. Thus you can
take $(\iota_1 \after \phi_{1U})(x)$ and pull it back along
$\iota_2$ to give $\phi_{2U}(x)$. Thus $\phi_{1U}=\phi_{2U}$ for all $U$. \qed

\section*{2.4.E}
\begin{itemize}
    \item[$\rightarrow$] Taking the stalk at $p$ is a functor and so preserves isomorphisms. \qed
    \item[$\leftarrow$] The isomorphisms of stalks assemble to an isomorphism of the sets of
          compatible germs. Consider the following diagram:
          \[
              \begin{tikzcd}
                  \mathscr{F}(U) \ar[r, "\phi"] \ar[d, hook, "\iota_F"]
                   & \mathscr{G}(U) \ar[d, hook, "\iota_G"] \\
                  \prod_{p \in U} \mathscr{F}_p \ar[r, "\phi_p"]
                   & \prod_{p \in U} \mathscr{G}_p
              \end{tikzcd}
          \]
          The above statement says that $\phi_p$ is an isomorphism of the images of the two
          $\iota$ maps. Thus, let $f, g \in \mathscr{F}(U)$ such that $\phi(f)=\phi(g)$.
          Then $\iota_F(f)=\phi^{-1}_p(\iota_G(\phi(f)))=\iota_F(g)$. Also,
          if $g \in \mathscr{G}(U)$ then $g=\phi(\iota_F^{-1}(\phi_p^{-1}(\iota_G(g))))$. \qed
\end{itemize}

\section*{2.4.H}
Apply the universal property to the morphism $\text{sh} \after \phi$.

\section*{2.4.I}
2.4.A is equivalent to identity, 2.4.C is equivalent to gluability. $\mathscr{F}^{\text{sh}}$
satisfies both by construction.

\section*{2.4.J}
Send $s \in \mathscr{F}(U)$ to $(s_p)_{p \in U}$.

\section*{2.4.K}
Define $f((s_p)_{p \in U})=g(s)$. This is well defined since the conditions of 2.4.D hold for
$g$.

\section*{2.4.L}
We want to show $\text{Hom}(\mathscr{F}^\text{sh}, \mathscr{G}) \widetilde{=}\
    \text{Hom}(\mathscr{F}, F(\mathscr{G}))$
where $F$ is the forgetful functor. We do this by sending a morphism on the right to its
image under sheafification, and a morphism $\phi$ on the left to the morphism $\phi \after \
    \text{sh}$.

\section*{2.4.M}
The germ of $(s_p)_{p \in U})$ at a point $q$ is $s_q$.

\section*{2.4.N}
c $\implies$ b is clear. b $\implies$ a follows from the fact that taking the stalks at $p$
reflects monomorphisms which follows from the fact that morphisms are determined on stalks.
a $\implies$ c follows by applying the "evaluate at $U$" functor to the composition indicated
in the hints

\section*{2.4.O}
\begin{itemize}
    \item[b $\implies$ a] follows from the fact that taking the stalks at $p$
          reflects epimorphisms which follows from the fact that morphisms are determined on stalks.
    \item[a $\implies$ b] Consider the skyscraper sheaf $i_{p,*}2$ as well as the two morphisms
          $\psi, \psi'$ defined on $U \ni p$ open by $\psi_U(g)=0$ if $g_p \in im(\phi_p)$ or 1 otherwise
          and $\psi'_U(g)=0$ for all g. Then $\psi \after \phi = \psi' \after \phi$. Taking stalks at $p$
          we get $\psi_p=\psi'_p$ so that $im(\phi_p)=\mathscr{G}_p$.
\end{itemize}

\section*{2.5.A}
We can recover stalks: since the base covers any open set, any two sections that are eventually
equal are eventually equal on a basic open set. Then just use the fact that sections are determined
by compatible germs.

\section*{2.5.B}
Both $F$ and $\mathscr{F}$ have the same etale space.

\section*{2.6.A}
Filtered colimits commute with finite limits.

\section*{2.6.B}
Taking stalks and cokernels are both colimits, and thus commute.

\section*{2.6.C}
The image is the kernel of the cokernel. Sheafifying we get the same result in the category of
sheaves. $(\ker \coker f)_p=\ker (\coker f)_p = \ker \coker f_p$

\section*{2.6.D}
This is just a repeat of 2.4.N and 2.4.O.

\section*{2.6.F}
This is 2.4.N again.

\section*{2.6.G}
Use 2.7.B and the fact that right adjoints preserve limits.

\section*{2.6.H}
Use 2.4.N and functoriality on open sets.

\section*{2.6.J}
Define using the tensor product open set by open set. Use the restriction maps $\Fsh(U) \to \
    \Fsh(V)$ and $\Gsh(U) \to \Gsh(V)$ to form get a define a restriction map on the tensor product.

\section*{2.7.B}
By left adjointness of $\text{sh}$ we can do this over presheaves. Now, consider a morphism $\phi$ on the left.
For any open $V \subseteq Y$ we have an open set $U=\pi^{-1}(V)$ so that $\pi(U) \subset V$.
Now we have a map from $\Fsh(V)$ to $\pi^{-1}\Fsh(U)$ by definition of colimit.
We define $\phi_{*,V}$ to be the composition of this map followed by $\phi$, followed by the isomorphism
$\Gsh(U)=\pi_*\Gsh(V)$.

Now let $\psi: \Fsh \to \pi_*\Gsh$. Let $U \subseteq X$, $V \supseteq \
    \pi(U)$ be open, and let $W=\pi^{-1}(V)$. For any $x \in \pi^{-1}\Fsh(U)$, we take $\Fsh(V)$
to have a representative for $x$, $\widetilde{x}$. Then
$\psi^*_U(x)=\operatorname{res}_{WU}(\psi_V(\widetilde{x}))$,
where the restriction happens in  $\Gsh(W)=\pi_*\Gsh(V)$. This is well defined:
indeed, if we take two different representatives $\widetilde{x}, \widehat{x}$ for $x$ then
the images are equal on some open set whose inverse image contains $U$.

\section*{2.7.C}
Stalk is a colimit and left adjoints preserve colimits.

\section*{2.7.D}
Just examine the etale spaces of both.

\section*{2.7.E}
$\pi^{-1}$ is left adjoint and so preserves colimits, specifically cokernels.
\begin{align*}
    \i_p^{-1}\ker\pi^{-1}(f) & = \ker\pi^{-1}(i_p^{-1}f) \\
                             & = \pi^{-1}\ker(i_p^{-1}f) \\
                             & = i_p^{-1}\pi^{-1}\ker(f)
\end{align*}
Where we use the fact that $\pi^{-1}_p$ is a colimit and thus commutes with kernels.
Finally, we conclude from 2.4.N that $\pi^{-1}\ker(f)$ is the kernel of $f$ in the category of sheaves

\section*{2.7.F}
\begin{itemize}
    \item[a.] Check the etale space.
    \item[b.] Take stalks. Then we get a morphism $\phi:\Fsh_p \to (i_*i^{-1}\Fsh)_p$
          and a morphism $\psi: (i_*i^{-1}\Fsh)_p \to \Fsh_p$ arising from the direct image functor
          and the fact that stalks of inverse images are stalks of the original functor.
          We show that these are inverses. We reduce to the case where $p \in Z$ since
          otherwise both sides are one element sets.
          We show that $\psi$ is an isomorphism. For this it suffises to show
          that every sequence of open sets in $Z$ containing $p$ arises from a sequence
          of opens in $X$. This follows by definition of the subspace topology.
          Thus $\psi$ is an isomorphism. That the two morphisms compose to the identity
          is a straightforward and tedius exercise in chasing definitions.
\end{itemize}

\end{document}
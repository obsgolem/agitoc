\documentclass{article}
\usepackage[utf8]{inputenc}

\usepackage{mathtools}
\usepackage{amsthm}
\usepackage{amssymb}
\usepackage{dsfont}
\usepackage{array}   % for \newcolumntype macro
\usepackage{calligra}
\usepackage{tikz-cd}
\usepackage{mathrsfs}
\usepackage{quiver}
\usepackage[shortlabels]{enumitem}

\newcolumntype{L}{>{$}l<{$}}
% math-mode version of "l" column type

\newcommand{\Rho}{\mathrm{P}}

\newcommand{\iso}{\simeq}
\newcommand{\after}{\circ}
\newcommand{\catname}[1]{\mathbf{#1}}
\newcommand{\intersect}{\cap}
\newcommand{\union}{\cup}
\newcommand{\Intersect}{\bigcap}
\newcommand{\Union}{\bigcup}
\newcommand{\wa}[1]{\langle #1 \rangle}
\newcommand{\bb}[1]{\mathbb{#1}}
\newcommand{\A}{\mathbb{A}}
\newcommand{\Z}{\mathbb{Z}}
\newcommand{\Q}{\mathbb{Q}}
\newcommand{\R}{\mathbb{R}}
\newcommand{\C}{\mathbb{C}}
\newcommand{\F}{\mathbb{F}}
\newcommand{\p}{\mathfrak{p}}
\newcommand{\q}{\mathfrak{q}}
\newcommand{\ai}{\mathfrak{a}}
\newcommand{\bi}{\mathfrak{b}}
\newcommand{\m}{\mathfrak{m}}
\newcommand{\defeq}{\vcentcolon=}

\DeclareMathOperator{\id}{id}
\DeclareMathOperator{\Hom}{\mathscr{H}\text{\kern -3pt
{\calligra\large om}}\,}
\DeclareMathOperator{\Fsh}{\mathscr{F}}
\DeclareMathOperator{\Gsh}{\mathscr{G}}
\DeclareMathOperator{\Hsh}{\mathscr{H}}
\DeclareMathOperator{\Osh}{\mathscr{O}}
% \DeclareMathOperator{\ker}{ker}
\DeclareMathOperator{\coker}{coker}
\DeclareMathOperator{\im}{im}
\DeclareMathOperator{\Spec}{Spec}
\DeclareMathOperator{\Proj}{Proj}
\DeclareMathOperator{\V}{V}
\DeclareMathOperator{\D}{D}

\newcommand*{\rationalto}[1][]{\mathbin{\tikz [baseline=-0.25ex,-latex, dashed,->,densely dashed,#1] \draw [#1] (0pt,0.5ex) -- (1.3em,0.5ex);}}%


\title{Chapter 8 Problems}
\author{Josiah Bills}
\date{March 2021}

\begin{document}

\maketitle

\section{8.1.A}
Choose an affine cover. Injectivity clearly holds on this cover, and so holds
globally. The complement of the image on an element of the affine cover is
open, so complement of the overall image is open. Hence the image is closed.
\qed

\section{8.1.B}
Quotients are clearly finite over the base ring, and closed embeddings are
affine by definition.

\section{8.1.C}
Composition of affines morphisms is affine. Composition of surjective morphisms
of rings is surjective. \qed

\section{8.1.D}
Consider $A \to B$ surjective. Then $A_f \to B_f$ is also
surjective. The latter part of the affine communication lemma is just gluing
schemes along affine opens.

\section{8.1.E}
Apply 8.1.D. \qed

\section{8.1.F}
Let $f: Y \to X$ be a closed subscheme. There are only two closed
sets in $X$, $X$ itself and
${\m}$. For the latter, we must have a surjective morphism
$k[x]_{(x)} \to l$ where $l$ is a field. The kernel of
this must be the closed point $\m$, which is not
$mathfrak{I}(X)$. Suppose we are given $f: X \to X$ an
automorphism and closed embedding. Then $f^{-1}(\D_x)=f^{-1}({\eta})$, giving us
$f^{\sharp}: k(x) \to B$. If the condition on ideals holds then the kernel is
$k(x)$, meaning that $B$ is the empty set,
contradicting $f$ being an automorphism. \qed

\section{8.1.G}
Let $\phi: B \to B_f$ be the natural map. Then there is an induced map
$\phi': I(B)_f \to I(B_f) x/f \mapsto \phi(x)/f$. The result follows by exactness of localization.

\section{8.1.H}
It is easy to see that $\operatorname{I}(B)$ defines a sheaf on a base. The
restriction maps are the ristriction to the ideal of the ring restriction maps.
The hypothesis ensures this is well defined. It remains to conclude that
$\Osh / \mathfrak{I}$, regarded as the cokernel in the category of presheaves,
is in fact a sheaf. Consider $\to \Spec B_f / \operatorname{I}(B_f) \to \Spec B/\operatorname{I}(B)$. Regarding this as a sheaf
on $\Spec B$, we see that regular scheme gluing and identity
applies. We used the hypothesis in order to ensure open inclusions are well
defined. \qed

\section{8.1.I}
\begin{enumerate}[a.]
    \item Consider the ideal $(s)$ on some affine open cover. Let
          $\phi: A \to A_f$. Then $(\phi(s))=(s)_f$ so that we satisfy the
          conditions of 8.1.H.
    \item It is clear that on an affine open cover $(su)=(s)$. \qed
    \item Using notation as in (a), we have $(\phi((S)))=(S)_f$ and we are done.
\end{enumerate}

\section{8.1.J}
\begin{enumerate}[a.]
    \item We check that if $I_1, I_2 \subseteq A$ are ideals, then $(I_1 \intersect I_2)_f = I_{1,f} \intersect
              I_{2,f}$.
          Let $x/f^n \in (I_1 \intersect I_2)_f$. Then since $x \in I_1 \intersect I_2$,
          $x/f^n \in I_{1,f} \intersect
              I_{2,f}$. The converse holds for similar reasons. Now consider
          $(x_1+x_2)/f^n \in (I_1 + I_2)_f$ where $x_i \in I_i$. It is obvious that
          $(x_1+x_2)/f^n=x_1/f^n+x_2/f^n \in I_{1,f} +
              I_{2,f}$. Thus $(I_1 \intersect I_2)(B)$ and $(I_1 + I_2)(B)$
          define closed embeddings by 8.1.H. \qed
    \item We have $(y-x^2)+(y)=(y,x^2)$, so we have a closed subscheme
          $\Spec A[x, y]/(x^2, y)$, a parabola intersect the x-axis. Now,
          $A[x, y]/(x^2, y) \cong
              A[x]/(x^2)$. The union is given by $(y-x^2) \intersect (y)$. \qed
    \item Trivial
    \item We have $(y^2-x^2)+(y)=(y, x^2)$, we also have $(y^2-x^2)=((y+x)(y-x))=(y+x)(y-x)$. Thus the
          above ideal corresponds to $(\V(y+x)\union\V(y-x))\intersect\V(y)$. Now $\V(y+x)\intersect\V(y)=\V(x,y)$,
          but $(x,y)(x,y) \neq (y, x^2)$. So while they give the same closed set, they don't
          give the same closed subscheme.
\end{enumerate}

\section{8.1.K}
Let $\pi: X \to \Spec A$ be a morphism which induces a homeomorphism. Then
$X$ is affine. Let $B=\Osh_X(X)$. Then we have a
morphism $\phi: X \to \Spec B$. It is clear that $\phi$ is a
homeomorphism, so it remains to show that it is an isomorphism of sheaves. Let
$\Spec C \subseteq X$. Then we have a morphism $\phi': B \to C$. Now,
let $b \in B$ be such that $\phi^{-1}(\D(b)) \subseteq \Spec C$. But it is clear
that $\phi^{-1}(\D(b)) = \D(\phi'(b))$. Now, let $b$ range over
$B$ to get a cover of $\Spec B$ by affines. By
affine locallity of affineness, we are done.

The fact that closed immersions satisfy this property follows from 2.5.E. For
the other direction, we immediately reduce to the case where
$Y$ is affine, say $\Spec A$. We find that
$\pi$ corresponds to a morphism $\pi^{\sharp}: A \to \Osh_X(X)$. Taking
kernels, we find that $\pi=\psi \after \phi$ where $\psi: \Spec A/I \to \Spec A$ is a
closed embedding. Since $\psi$ induces a homeomorphism, by the
above lemma, $X$ is affine. Hence the morphism is affine.

We change our notation now and write $\pi: A \to B$,
$\psi: A \to A/I$, and $\phi: A/I \to B$, with $\phi$
inducing a homeomorphism of spaces. By 8.1.H, the closed subscheme structure of
$\psi$ is uniquely determined by the ideal
$I=\ker \psi = \ker \pi$. Now, apply 2.4.E to the stalks of
$\phi$. \qed

\section{8.1.L}
Closed embeddings are finite by 8.1.B. Finite morphisms are locally of finite
type by 7.3.P. Open embeddings are locally of finite type by 7.3.Q(a), so the
result follows by 7.3.Q(b). \qed

\section{8.1.M}
The intersection is defined as the pullback given in the hint. (i)
$\iff$ (ii) follows directly from 7.2.B. (i)
$\implies$ (iii) follows from 9.2.B.

\section{8.1.N}
Consider $f=f_2 \after g_2 \after f_1 \after g_1$ where $g_1, g_2$ are closed
embeddings and $f_1, f_2$ are open embeddings. Then by 8.1.M we
have $f=f_2 \after f_3 \after g_3 \after g_1$, where $g_3$ is a closed embedding
and $f_3$ is an open embedding. \qed

\end{document}
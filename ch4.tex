\documentclass{article}
\usepackage[utf8]{inputenc}

\usepackage{mathtools}
\usepackage{amsthm}
\usepackage{amssymb}
\usepackage{dsfont}
\usepackage{array}   % for \newcolumntype macro
\usepackage{calligra}
\usepackage{tikz-cd}
\usepackage{mathrsfs}

\newcolumntype{L}{>{$}l<{$}} % math-mode version of "l" column type

\newcommand{\Rho}{\mathrm{P}}

\newcommand{\iso}{\simeq}
\newcommand{\after}{\circ}
\newcommand{\catname}[1]{\mathbf{#1}}
\newcommand{\intersect}{\cap}
\newcommand{\union}{\cup}
\newcommand{\Intersect}{\bigcap}
\newcommand{\Union}{\bigcup}
\newcommand{\wa}[1]{\langle #1 \rangle}
\newcommand{\ds}[1]{\mathds{#1}}
\newcommand{\Z}{\mathds{Z}}
\newcommand{\Q}{\mathds{Q}}
\newcommand{\R}{\mathds{R}}
\newcommand{\C}{\mathds{C}}
\newcommand{\p}{\mathfrak{p}}
\newcommand{\q}{\mathfrak{q}}
\newcommand{\ai}{\mathfrak{a}}
\newcommand{\bi}{\mathfrak{b}}
\newcommand{\m}{\mathfrak{m}}
\newcommand{\defeq}{\vcentcolon=}

\DeclareMathOperator{\Hom}{\mathscr{H}\text{\kern -3pt {\calligra\large om}}\,}
\DeclareMathOperator{\Fsh}{\mathscr{F}}
\DeclareMathOperator{\Gsh}{\mathscr{G}}
\DeclareMathOperator{\Hsh}{\mathscr{H}}
\DeclareMathOperator{\Osh}{\mathscr{O}}
% \DeclareMathOperator{\ker}{ker}
\DeclareMathOperator{\coker}{coker}
\DeclareMathOperator{\im}{im}
\DeclareMathOperator{\Spec}{Spec}
\DeclareMathOperator{\V}{V}
\DeclareMathOperator{\D}{D}


\title{Chapter 4 Problems}
\author{Josiah Bills}
\date{August 2020}

\begin{document}

\maketitle

\section{4.1.A}
3.5.E classifies the $g$ such that $\D(f) \subseteq \D(g)$
as the invertible elements of $A_f$. \qed

\section{4.1.D}
Tensor the exact sequence 4.1.3.1 with $M$. The result
follows from exactness and adjointness of tensor.

\section{4.3.A}
Let $X=\Spec A$ and $X'=\Spec A'$.

We show that if $\pi(\p)=\q$ then $\pi^{\#,-1}(\p)=\q$. Indeed,
let $f$ be a global section of $X$ that
vanishes at $\p$. Then $f \in \p$. Now, this
pulls back to a unique function $g \in A'$. By the general theory
of sheaves we get a isomorphism of local rings $A'_{\q} \to A_{\p}$.
Projecting to these stalks we find that $g$ vanishes at
$\q$, as desired. Performing the same argument with the
inverse map shows the desired result.

The inverse image of every distinguished open is a distinguished open, and the
direct image of a distinguished open under a homeomorphism is a distinguished
open. This corresponds to the algebraic fact that the extension of a principal
ideal along a ring map is principal.

That this function is continuous with continuous inverse is trivial. To show
the final result it suffices to show that the isomorphism
$\Osh_{X'}(\D(g)) \to \pi_*\Osh_X(\D(g))$ is the isomorphism $A'_g \to A_f$ induced by
$\pi^{\#}$. Indeed, we have that $\Osh_{X'}(\D(g))=A'_g$ and
$\pi_*\Osh_X(\D(g))=A_f$. From here the rest follows from the definition of the
structure sheaves.

\section{4.3.C}
We first note that the intersection of an affine open with an arbitrary open is
covered by distinguished opens of the affine open since it is open in the
subspace topology. This gives us an open cover of $U$ by
(distinguished) affine opens.

\section{4.3.D}
Follows trivially from the previous exercise.

\section{4.3.E}
\begin{itemize}
      \item [a.] Apply 4.3.A to the scheme $\coprod_{i=0}^n \Spec A_i \simeq \Spec \prod_{i=0}^n A_i$. \qed
      \item [b.] Consider the infinite disjoint union of one point spaces. Then the one
            point open sets are an infinte cover of the space without a finite subcover.
\end{itemize}

\section{4.3.F}
This is essentially noting that localization at a prime is a colimit over the
finite subsets of the complement of that prime. See exercise 1.something.

\section{4.3.G}
\begin{itemize}
      \item [a.]
            If $f$ doesn't vanish at $\p$ then it is
            invertible at $A_{\p}$. So there is a genuine open set where it is
            invertible, say $U$. This set can't contain a point where
            $f$ vanishes, otherwise $f$ wouldn't be
            invertible. The result follows easily from this.
      \item[b.] $f$ has an inverse on every stalk. Thus the inverses at
            the level of stalks give a set of compatible germs. Thus $f$
            is invertible globally.

            Second proof: it is clearly invertible on an affine open cover. These inverses
            glue to give a global inverse. \qed
\end{itemize}

\section{4.4.A/2.5.D}
We take the basis suggested in 2.5.D. The sheaf on this base is then the
obvious one where $\Fsh(U)=\Fsh_i(U)$. The cocyle condition ensures
everything here is well defined.

The global sections are explicitly this: A tuple of sections on each affine
such that they are compatible in the obvious way.

\section{4.4.B}
Consider $\Gamma(X,\Osh_D)$ and $\Gamma(Y,\Osh_D)$ where
$D$ is the line with the doubled origin. These two rings
are isomorphic via the obvious isomorphism, and any two isomorphic sections are
identical on the overlaps. The global sections are thus isomorphic to the
sections on $X$. But then it can't be affine since
$D \not \simeq X=\Spec k[x]$.

\section{4.4.C}
The definition of the space is obvious. The two affine sets which intersect at
a non affine set are the two copies of the affine plane which intersect at the
punctured plane.

\section{4.4.D}
We denote three open subsets $U_i, U_j, U_k$. These correspond to the
rings $R_i=k[X_{n/i}]/(x_{i/i}-1),
      R_j=k[X_{n/j}]/(x_{j/j}-1),
      R_k=k[X_{n/k}]/(x_{k/k}-1)$ where we say $X_{n/i}=x_{0/i},x_{1/i}, \dots
      x_{n/i}$. The
intersections $U_{ij}$ then correspond to the rings
$R_{ix_{j/i}} \underset{\phi_{ij}}{\simeq}
      R_{jx_{i/j}}$ where $\phi_{ij}$ is the given isomorphism.
The triple intersections are then $R_{ix_{j/i}x_{k/i}}$ and the other two
variants.

We wish to show that $\phi_{ij}=\phi_{kj} \after \phi_{ik}$. Let $l \in [0,n]$. Then
\begin{align*}
      (\phi_{kj} \after \phi_{ik})(x_{l/i}) = & \phi_{kj}(x_{l/k}/x_{i/k})            \\
      =                                       & \phi_{kj}(x_{l/k})/\phi_{kj}(x_{i/k}) \\
      =                                       & (x_{l/j}/x_{k/j})/(x_{i/j}/x_{k/j})   \\
      =                                       & x_{l/j}/x_{i/j}                       \\
      =                                       & \phi_{ij}(x_{l/i})                    \\
\end{align*}
\qed

\section{4.5.A}
We consider $\Spec k[x_{0/2},x_{1/2}] /
      (x_{0/2}^2+x_{1/2}^2-1)$ and $\Spec k[x_{0/1},x_{2/1}] /
      (x_{0/1}^2+1-x_{2/1}^2)$. The gluing
morphisms are the same ones as in $4.4.D$ sending e.g.
$x_{0/1}=x_{0/2}/x_{1/2}$. We check that this is indeed an isomorphism: Let
$\phi_{12}(f/x_{2/1}^n)=0$. We have $\phi_{12}(f)/x_{1/2}^n=0$ so that
$\phi_{12}(f)=0$. Then $\phi_{12}(f)=ax_{0/2}^2+ax_{1/2}^2-a$

\section{4.5.C}
\begin{itemize}
      \item[a.] The if side is obvious. For the only if side, consider an element
            $a$. This element is written as a sum of homogenous
            elements. Since sum doesn't change the degree, the homogenous piece of degree
            $n$ must be contained in the ideal. \qed
      \item[b.] For the sum, take the union of the generating sets. We note that the
            product of two elements is homogenous if the two elements are homogenous. Since
            products of ideals are generated by products of generators, we have the result.
            For intersection, simply recall part a. For radicals, let
            $x^n \in I$. Let $x_k$ be the top degree component
            of $x$. Then the top degree component of
            $x^n$ must have degree $kn$. The product of
            components of degree lower than $k$ won't have degree
            $kn$, so $x_k^n$ must be the top degree
            component of $x^n$. Thus $x_k \in \sqrt{I}$. The result
            follows by induction on the degree of $x^n$. \qed
      \item[c.] Let $ab \in I$ for $a,b \in S_{\bullet}$. Let
            $(ab)_k$ be the degree $k$ component of
            $ab$. Now, let $k$ be the top degree of
            $a$ and $j$ the top degree of
            $b$. Then $a_kb_j \in I$, so $a_k \in I$
            or $b_j \in I$. WLOG assume $a_k \in I$. Then we may
            replace $a$ with $a-a_k$ and repeat the
            process. This may be done finitely many times, until either
            $a$ or $b$ has been replaced with 0. In
            either case we find that all the homogenous parts of $a$ are
            in $I$ or all the homogenous parts of $b$
            are in $I$, hence $a \in I$ or
            $b \in I$. \qed
\end{itemize}

\section{4.5.D}
\begin{itemize}
      \item [a.] Let $S_+=(s_0, \dots, s_n)$ where the $s_i$ are
            homogenous. Let $g$ be arbitrary. If
            $g$ is degree 0 then we are done. Otherwise,
            $g \in S_+$. WLOG assume $g$ is homogenous.
            Then $g=\sum_{i=0}^n b_is_i$ for $b_i \in S_{\bullet}$. Then the
            $b_i$ necessarily have degree smaller than
            $g$. The result follows by induction. The reverse direction
            is trivial. \qed
      \item [b.]
            \begin{itemize}
                  \item[$\implies$] Noetherian implies $A$ is
                        noetherian, and also that $S_+$ is finitely generated. \qed
                  \item[$\impliedby$] A finitely generated algebra over a noetherian ring is
                        noetherian by Hilbert's basis theorem. \qed
            \end{itemize}
\end{itemize}

\section{Sanity check}
We verify that $(k[x_0, \dots, x_n]_{x_i})_0$ is isomorphic to the specified ring. Let
$f/x_i^n$ have degree 0. Then $f$ is
homogenous and $\deg f = n*\deg x_i$. We write $f=\prod_j x_j^{n_j}$ where
$\sum_j n_j=n$. The isomorphism then sends $x_j/x_i$ to
$x_{j/i}$ as desired.

\section{4.5.E}
\begin{itemize}
      \item[a.]
            Let $a \in Q_i$. Then $(a^{\deg f}/f^i)^2 \in P_0$ so that
            $a^2 \in Q_i$. The reverse follows by the fact that
            $P_0$ is radical. $a_1+a_2 \in Q_i$ follows from this
            result. Let $a \in A$ be homogenous of degree
            $n$ and $b \in Q_i$. Then $ab$
            has degree $i+n$. We have that $(ab)^{\deg f}/f^{i+n}=(a)^{\deg f}/f^{n}*b^{\deg f}/f^{i} \in P_0$ so that
            $P$ is a homogenous ideal. Now let $ab \in Q_i$
            where $a, b$ are homogenous of degree $m$
            and $n$ respectively (so that $m+n=i$). Then
            $(ab)^{\deg f}/f^{i} \in P_0$ so that $a^{\deg f}/f^{m}*b^{\deg f}/f^{n} \in P_0$. Then
            $a^{\deg f}/f^{m}$ or $b^{\deg f}/f^{n}$ is in $\P_0$,
            so $a$ or $b$ is in
            $P$.

            It remains to show that every homogenous prime ideal arises this way. Let
            $P$ be a homogenous prime. Then $P=\bigoplus Q_i$ where
            $Q_0$ is prime in $A_0$. Let
            $a \in Q_i$. Then clearly $a^{\deg f}/f^{i} \in Q_0$. Let
            $a^{\deg f}/f^{i} \in Q_0$ for some $a$. Then
            $a^{\deg f} \in P$ so $a \in Q_i$. \qed

      \item[b.] We describe a bijection between relevant homogenous ideals not containing
            $f$ and homogenous ideals in $(S_{\bullet})_f$. Let
            $P \subset (S_{\bullet})_f$ be homogenous. We wish to show that
            $P^{\text{c}}$ is relevant, homogenous, and doesn't contain
            $f$. The latter is clear, otherwise $P=(1)$.
            This also shows that $P$ is relevant, since
            $f$ has positive degree. It remains to show that it is
            homogenous. Let $a \in P^{\text{c}}$. Then $P$ contains
            all of $a$'s homogenous components(by homomorphism). Thus
            these components are in $P^{\text{c}}$. \qed
\end{itemize}

\section{4.5.F}
This follows directly from 4.5.E(b).

\section{4.5.G}
The sum of two homogenous ideals of positive degree is surely positive degree,
so $\V(I)$ is $\V(\sum_{i \in I} (i))$, where the sum is
(possibly) infinite. \qed

\section{4.5.H}
\begin{itemize}
      \item[a.] This says that if $f \in \p$ for all $\p \in \V(I)$
            then $f \in \sqrt{I}$ (the other direction is trivial). Consider an
            arbitary prime $\q$ containing $I$. We
            wish to show that $f \in \q$. Indeed, let $\q^{\text{h}}$
            be the ideal generated by homogenous elements of $\q$.
            Clearly then, $\q^{\text{h}} \subseteq \q$. Also, it is prime since primeness can be
            checked on homogenous elements. Now, $f \in \q^{\text{h}}$, so
            $f \in \q$. \qed
      \item[b.] Take the intersection of all these homogenous primes. This is necessarily
            homogenous. We note that it is radical since the intersection of primes is
            always radical. That $\operatorname{I}(Z_1 \union Z_2) = \operatorname{I}(Z_1) \intersect
                  \operatorname{I}(Z_2)$ is clear from the definition.
      \item[c.] $\V(\operatorname{I}(Z)) \supseteq \overline{Z}$ essentially by definition. Note that
            $\overline{Z}=\V(\operatorname{I}(\overline{Z}))$. But clearly $\operatorname{I}(\overline{Z}) \subseteq \operatorname{I}(Z)$. \qed
\end{itemize}

\section{4.5.I}
a $\iff$ b is clear from the proof of 4.5.G. Suppose a. Then
every homogenous prime ideal containing $I$ containst
$S_+$. Suppose b. Every homogenous prime ideal containing
$I$ then contains $S_+$, thus there are
no relevant homogenous primes in $\V(I)$. \qed

\section{4.5.J}
Let $\V(I) \intersect \D(f)$ be closed. Since $f \notin I$ we may
pull back $I$ to get an ideal in $S_{\bullet, f, 0}$.
Then by 4.5.E we get that the prime ideals containing this ideal are in one to
one correspondence with the ideals in $\V(I) \intersect \D(f)$. \qed

\section{4.5.K}
Let $a=g^{\deg f}/f^{\deg g}$ and let $x = b/f^n \in S_{\bullet, f, 0}$. Then
$x/a^m \in S_{\bullet, f, 0, a, 0}$. We see that \[
      \frac{x}{a^m} = \frac{b}{f^na^m} =
      \frac{bf^{\deg g}}{f^ng^{\deg f}} \in
      S_{\bullet, fg, 0}
\] The reverse
direction is similar.

\section{4.5.L}
This argument should be identical to the one in 4.4, just with some added
exponents.

\section{4.5.M}
\section{4.5.N}
Note that the homogenous ideal generated by the $x_i$ is the
irrelevant ideal, then apply 4.5.I. \qed

\section{4.5.O}
We get a maximal ideal generated by $x_{i/j}-a_{i/j} \in \Spec S_{\bullet, x_j, 0}$. This yeilds a
homogenous ideal generated by $x_i-a_ix_j$.

\section{4.5.P}
We note that $\operatorname{Proj}(S/I) \simeq \V(I)$ as topological spaces. Indeed, this
follows from the same lattice isomorphism as in $\Spec S$. We
check that the inverse image of a homogenous ideal is homogenous. Indeed,
consider an arbitrary $j \in J \in \V(I)$. Then all the homogenous
components of $\pi(j)$ are in $J^{\text{e}}_{\pi}$. We note
that $(j+I)_k = j_k+I$. Since $I \subseteq J$,
$j_k \in J$. \qed

\section{4.5.Q}
We are reduced immediately to the case $k[x_0,\dots, x_n]$ where
$k$ is algebraically closed. Then by 4.5.O we get
projective coordinates say $[a_0, \dots, a_n]$. Then
$\sum_i a_ix_i$ is well defined up to scalar multiples, i.e. the
projective coordinates uniquely determine a line with respect to those
coordinates.

\end{document}
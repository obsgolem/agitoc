\documentclass{article}
\usepackage[utf8]{inputenc}

\usepackage{mathtools}
\usepackage{amsthm}
\usepackage{amssymb}
\usepackage{dsfont}
\usepackage{array}   % for \newcolumntype macro
\usepackage{calligra}
\usepackage{tikz-cd}
\usepackage{mathrsfs}
\usepackage{quiver}
\usepackage[shortlabels]{enumitem}

\newcolumntype{L}{>{$}l<{$}}
% math-mode version of "l" column type

\newcommand{\Rho}{\mathrm{P}}

\newcommand{\iso}{\simeq}
\newcommand{\after}{\circ}
\newcommand{\catname}[1]{\mathbf{#1}}
\newcommand{\intersect}{\cap}
\newcommand{\union}{\cup}
\newcommand{\Intersect}{\bigcap}
\newcommand{\Union}{\bigcup}
\newcommand{\wa}[1]{\langle #1 \rangle}
\newcommand{\bb}[1]{\mathbb{#1}}
\newcommand{\A}{\mathbb{A}}
\newcommand{\Z}{\mathbb{Z}}
\newcommand{\Q}{\mathbb{Q}}
\newcommand{\R}{\mathbb{R}}
\newcommand{\C}{\mathbb{C}}
\newcommand{\F}{\mathbb{F}}
\newcommand{\p}{\mathfrak{p}}
\newcommand{\q}{\mathfrak{q}}
\newcommand{\ai}{\mathfrak{a}}
\newcommand{\bi}{\mathfrak{b}}
\newcommand{\m}{\mathfrak{m}}
\newcommand{\defeq}{\vcentcolon=}

\DeclareMathOperator{\id}{id}
\DeclareMathOperator{\Hom}{\mathscr{H}\text{\kern -3pt
{\calligra\large om}}\,}
\DeclareMathOperator{\Fsh}{\mathscr{F}}
\DeclareMathOperator{\Gsh}{\mathscr{G}}
\DeclareMathOperator{\Hsh}{\mathscr{H}}
\DeclareMathOperator{\Osh}{\mathscr{O}}
% \DeclareMathOperator{\ker}{ker}
\DeclareMathOperator{\coker}{coker}
\DeclareMathOperator{\im}{im}
\DeclareMathOperator{\Spec}{Spec}
\DeclareMathOperator{\Proj}{Proj}
\DeclareMathOperator{\V}{V}
\DeclareMathOperator{\D}{D}

\newcommand*{\rationalto}[1][]{\mathbin{\tikz [baseline=-0.25ex,-latex, dashed,->,densely dashed,#1] \draw [#1] (0pt,0.5ex) -- (1.3em,0.5ex);}}%


\title{Chapter 9 Problems}
\author{Josiah Bills}
\date{October 2020}

\begin{document}

\maketitle

\section{7.1.A}
Trivial \qed

\section{7.1.B}
\[
    \begin{tikzcd}
        X \ar[drr, bend left] \ar[ddr, bend right] \ar[dr, dotted] &
                                                                   &
        \\
                                                                   & U \times_Z
        Y \ar[d] \ar[r, hook]                                      & Y \ar[d,
            "\pi"]
        \\
                                                                   & U \ar[r,
        hook]                                                      & Z
    \end{tikzcd}
\]

Define the dotted morphism by taking the inverse image of
$U$, noting that it goes to $\pi^{-1}(U)$ under
the top morphism. Thus it restricts to a morphism to that scheme. The product
is obviously an open subscheme. \qed

\section{7.1.C}
Recall that noetherianness descends to open subsets. Given a point in the
subset you are embedding, take a noetherian neighborhood and intersect it with
the open subset.

For the last question, consider $\D(x_1, x_2, \dots) \to A^{\infty}_k$. By 3.6.G(b) the former
is not quasicompact.

\section{7.1.D}
We need to exhibit a morphism such that on some open cover it is an open
immersion. Consider the line with the doubled origin projecting onto the affine
line. As an open cover, take the two standard open covers. Then on each open
set it is an isomorphism.

\section{7.2.A}
We first note that surjective ring morphisms are integral, since any ring is
integral over itself. Therefore, we may decompose an arbitrary morphism into an
integral surjection and potentially non-integral injection. Thus we may replace
$B$ with $\phi(B)$.

We note that if $(b_1, \dots, b_n)=(1)$ then $(b_1^m, \dots, b_n^m)=(1)$. This
follows from $\Union_i \D(b_i)=\Spec B$ and $\D(b_i)=\D(b_i^m)$. Now, let
$a \in A$. Then over $B_{b_i}$ we have
\[
    a^n+\sum_{j=1}^{n-1}
    a^jb'_j/(b_i)^{m_j}=0
\] for some $b'_j$ and
$m_j$. Clearing denominators and subtracting the sum gives us
an equation of the desired form. Now, let $n$ be the
largest power of $a$ appearing in one of these integral
dependence equations. Then for each $i$ we have an equation
of the form \[
    a^nb_i^{m_i}=\sum_{j=1}^{n-1} a^jb'_{ij}
\] Now, let $\sum_i b''_ib_i^{m_i}=1$. Then
\begin{align*}
    a^n & =a^n\sum_i b''_ib_i^{m_i}          \\
        & =\sum_i a^nb''_ib_i^{m_i}          \\
        & =\sum_i\sum_{j=1}^{n-1} a^jb'_{ij}
\end{align*}
Which is an equation of integral dependence for $a$ over
$B$. \qed

\section{7.2.B}
\begin{enumerate}[a.]
    \item Let $a/\phi(b)^m \in A_{\phi(b)}$. Then we have an integral relation for
          $a$, $\sum_{i=0}^n a^ib_i=0$. Multiply this by
          $1/\phi(b)^{nm}$ and recall that $1/\phi(b)^{nm} \in \phi(B)_{\phi(b)}$. This gives an
          integral relation for $a/\phi(b)^m$ over $B_b$.

          Quotient of $A$ is given by a finite, hence integral
          morphism, so by 7.2.C we have that the overall morphism is integral.

          Let $a \in A/\bi^{\text{e}}_{\phi}$. Then we have an integral relation
          $\sum_{i=0}^n a^ib_i=0$ in $A$. Apply the projection to this
          to get the desired integral relation.

          We note that $1/t$ is not integral over
          $k[t]$. Suppose otherwise: Then $1/t^n=\sum_{i=0}^{n-1} p_i(t)/t^i$, so that
          $1/t=\sum_{i=0}^{n-1} p_i(t)t^{n-1-i}$. But the latter is contained in $k[t]$.
          Thus, $k[t] \to
              k[t]_t$ is not integral. \qed

    \item Let $\phi : B \to A$ be an integral extension and let
          $\phi' :
              S^{-1}B \to \phi(S)^{-1}A$ be the localization. Let $\phi'(a/b^n)=0$. Then
          $0=\phi(a)/\phi(b)^n$ so that $\phi(s)\phi(a)=0$ for some
          $\phi(s)
              \in
              \phi(S)$. Thus, by injectivity of $\phi$,
          $sa=0$ so that $a/b^n=0$. Thus
          $\phi'$ is injective.

          Recall from above that localization of $A$ is not
          necessarily integral. The morphism given in the hints is clearly surjective but
          not bijective, thus it is in particular not injective (it is still integral).
          \qed
    \item Lemma: let $\phi: B \to A$ be an integral extension and let
          $\ai
              \subset A$ be an ideal. Then $\phi': B/\ai^{\text{c}}_{\phi} \to
              A/\ai$ is an integral
          extension. By 7.2.B(a) it is integral, so we merely need to show it is
          injective. Let $\phi'(b)=0$. Then $\phi(b) \in \ai$ so that
          $b \in \ai^{\text{c}}_{\phi}$. In other words, $b = 0$ in the quotient.

          We have $\pi: A/\ai^{\text{ce}}_{\phi} \to A/\ai$ given by $\pi(a)=a$. This is well
          defined: if $a-b \in \ai^{\text{ce}}_{\phi}$ then $\pi(a-b) \in \ai$. Note that
          $\ker \pi = \ai$.

          In the lemma above, $\phi' = \pi \after \phi''$ factors through
          $\phi'': B/\ai^{\text{c}}_{\phi} \to A/\ai^{\text{ce}}_{\phi}$. To see this, we note that if $b \in \ai^{\text{c}}_{\phi}$
          then $\phi(a) \in
              \ai^{\text{ce}}_{\phi}$. We recall that for any ideal,
          $\ai^{\text{cec}}_{\phi}=\ai^{\text{c}}_{\phi}$. Thus, $\ker \phi'' =
              (0)$. \qed

\end{enumerate}

\section{7.2.C}
We first show the following result: let $A$ be a finitely
generated $B$-algebra with each of its generators integral.
Then $A$ is finite over $B$. We reduce
to the 1D case immediately. Consider $B[a]$. By lemma 7.2.1,
this must be finite (it is the smallest algebra containing both
$B$ and $a$).

Now, let $a \in A$ be arbitrary. Then $a$
satisfies a relation $a^n+b_{n-1}a^{n-1}+\dots$. Then clearly
$a$ is integral over $C[b_{n-1},\dots]$. From the
above lemma, $C[a, b_{n-1},\dots]$ is finite. But finite morphisms are
integral, hence $a$ is integral over
$C$. \qed

\subsection*{Second proof}

Let $a \in A$ be arbitrary and let $\phi: C \to B, \pi: B \to A$. Now,
by lemma 7.2.1, we have a linear relation $a=\sum_i b_ia_i$. Repeat again
for the $b_i$, letting $b_i=\sum_j c_jb_{ij}$ for a
generating set dependent on $i$, $b_{ij}$.
Then $a=\sum_i \sum_j c_j(b_{ij}a_i)$. Then clearly the $b_{ij}a_i$ form a
generating set for a finite subalgeba containing $a$. \qed

\section{7.2.D}
Consider two integral elements $a, b \in A$. Then by the above lemma
these are contained in $\phi(B)[a, b]$ which is a in particular a
subalgebra. This holds for arbitrary integral elements, so the set of all
integral elements is a subalgebra.

\section{7.2.E}
The inverse image of the zero ideal under such a map is the zero ideal. Thus,
the only prime ideal of $B$ is the zero ideal. But in that
case it is maximal, hence the ring is a field. Let $b \in B$ be
nonzero. Then surely $1/b \in A$, satisfies an integral dependence
relation $1/b^n=\sum_{i=0}^{n-1} b_ib^i$. Then we have $1/b=\sum_{i=0}^{n-1} b_ib^{n-1-i} \in B$. \qed

\section{7.2.F}
Consider $\p_m$ and $\q_m \subset \q_{m+1}$. By the lemma in
7.2.B(c) we may quotient the integral extension by $\p_m$ to
reduce to the integral domain case. We then have $(0) \subset \q_{m+1}/\q_m$. Apply
the lying over theorem to get the desired ideal. The full result follows by
induction.

\section{7.2.G}
Given the following exact sequence \[
    \begin{tikzcd}
        0 \ar[r] & N \ar[r] & M \ar[r] & M/N \ar[r] & 0
    \end{tikzcd}
\] we tensor with
$A/I$ to get \[
    \begin{tikzcd}
        N/IN \ar[r] & M/IM \ar[r] & (M/N)/(IM/N) \ar[r] & 0
    \end{tikzcd}
\] The first arrow is
surjective by hypothesis and the second arrow is surjective by exactness. This
means that $(M/N)/(IM/N)=0$, or $IM/N=M/N$. Thus by
Nakayama 2, $M/N=0$ so that $M=N$. \qed

\section{7.2.H}
Let $N=(f_1, \dots, f_n) \subseteq M$. Then $N/mN \to M/mM$ is surjective since
the images of the $f_i$ generate $M/mM$.
Thus by Nakayama 3, $M=N$. \qed

\section{7.2.I}
We argue by contrapositive. Let $M$ be an
$S[r]$ module which is finitely generated as an
$S$ module. Assume that $r$ is not
integral. Then $M$ is not faithful. Indeed,
$\phi(x)=rx$ is a homomorphism. Applying Cayley-Hamilton to it we
get an equation of the form $(r^n+\sum_{i=0}^{n-1} s_ir^i)x=0$. But since
$r$ is not integral, the coefficient is not zero, as
desired. \qed

\section{7.2.J}
Trivial. \qed

\section{7.3.A}
Trivial.

\section{7.3.B}
Noetherian spaces are quasicompact, and noetherian descends to open subsets.
Thus every open subset of a Noetherian scheme is quasicompact. This proves part
(a). By the same vein, quasiseparated descends to open subsets, proving part
(b). \qed

\section{7.3.C}
\begin{enumerate}[a.]
    \item Let $U_i=\Spec A_i$. Let $f_{ij} \in A_i$ be such that
          $(f_{ij})=A$. Then if $\pi^{-1}(U_i)$ is quasicompact then so
          is $\pi^{-1}(U_{if_{ij}})$. Indeed, by quasicompactness of
          $\pi^{-1}(U_i)$ we may cover it with affine schemes
          $\Spec
              B_{ik}$, getting an induced morphism \[\pi_{ik}: \Spec B_{ik} \to
              \Spec A_i\]
          Since this comes from a morphism of affine schemes, we have
          $\pi_{ik}^{-1}(U_{if_{ij}})=\Spec B_{ik\pi^{\#}f_{ij}}$. By varying $k$ we get an open cover
          of $\pi^{-1}(U_{if_{ij}})$.

          The converse also holds, note that $\pi^{-1}(U_i)=\Union_j
              \pi^{-1}(U_{if_{ij}})$, which is a finite
          union of quasicompact spaces. \qed
    \item We first note that quasiseparatedness is inherited by open sets. Indeed, let
          $X$ be quasiseparated, let $U \subseteq X$ be open,
          and let $U_1, U_2 \subseteq U$ be quasicompact in $U$. Then
          $U_1$ and $U_2$ are quasicompact in
          $X$, hence their intersection is quasicompact in
          $X$, hence also in $U$.

          Let $U_i=\Spec A_i$. Let $f_{ij} \in A_i$ be such that
          $(f_{ij})=A$. Then $\pi^{-1}(U_i)$ is quasiseparated iff
          $\pi^{-1}(U_{if_{ij}})$ is also. The forward direction follows by the above
          note. For the converse, let $V_1, V_2 \subseteq \pi^{-1}(U_i)$ be quasicompact. Then we may
          cover their intersection with a finite number of quasicompact intersections,
          $\pi^{-1}(U_{if_{ij}}) \intersect V_1$.
\end{enumerate}

\section{7.3.D}
Recall that affine schemes are quasicompact and quasiseparated (5.1.G).

\section{7.3.E}
Consider the restriction map $\Gamma(X, \Osh_X) \to \Gamma(X_s, \Osh_X)$. On a given affine cover
of $X_s$ we have that $s$ is invertible,
thus it is invertible on $X_s$. Now, use the universal
property of localization to define a morphism as desired. \qed

\section{7.3.F}
Let $Y=\Spec A \setminus Z$. Then the open embedding $\iota: Y \to \Spec A$ is
affine. Indeed, $Y$ is covered by open sets
$Y_{f_i}$. These may in turn be covered by affine open sets
$\Spec A_{jf_i}$. Then the $\Spec A_j$ cover
$\Spec
    A$, and $\iota^{-1}(\Spec A_j)=\Spec A_{jf_i}$ which is affine. \qed

\section{7.3.G}
We first note the following property of affine morphisms: If
$\mathbf{P}$ is a property of morphisms that is local on the target,
then it suffices to check this for affine schemes.

Now, it is trivial to check that the localization of a finite morphism of rings
is finite. For the converse, we may use a similar argument to that used in
$7.2.A$. \qed

\section{7.3.H}
We note that $X$ must be affine, say
$\Spec A$. We then have a finite $k$-algebra
$A$. If $A$ is an integral domain, then
it is a field (by 3.2.G) and we are done. Recall that finiteness passes to
quotients. Hence $A/\p$ is a integral domain which is a finite
$k$-extension, hence $\p$ is maximal.
Thus every point is closed, and thus the space is cofinite. Since
$A$ is finite over $k$ it is obviously
finitely generated, and hence noetherian. Hence by 3.6.15 it may be written as
a finite union of irreducible subsets. But we just identified these as the
points of $\Spec A$. Hence $\Spec A$ is a finite,
discrete space. \qed

\section{7.3.I}
This is trivial on morphisms of affine schemes, hence for all schemes.

\section{7.3.J}
If $f \in R$ then we define $f_n \in S_n$. The
multiplication on $S_{\bullet}$ is given by $f_ng_m=(fg)_{n+m}$.
We check the distinguished open sets of $\Proj
    S_{\bullet}$. Let
$f_n \in S_n$. Then $((S_{\bullet})_{f_n})_0
    \cong R_f$ by sending
$h_{nm}/f_n^m$ to $h/f^m$. We immediately see that
$h_{lm}f_n^m=h_{nm}f_l^m=hf_{nm+lm}$. These isomorphisms assemble to the desired
isomorphism. By finiteness we have a finite generating set (as a graded
algebra) in positive degree. Now apply 4.5.D(a).

\section{7.3.K}
We first note that the arguments in 5.2.B(a) apply to finite morphisms. Let
$\phi: B \to A$ be a finite morphism, and let $\p \subseteq B$
be prime. Then we have a finite morphism $\phi_{\p}: B/\p \to A/\p^{\text{e}}_\phi$ and a
corresponding morphism of schemes $\pi_{\p}$. Then
$\pi^{-1}(\p)$ is in one to one correspondence with
$\pi_{\p}^{-1}((0))$. So we may reduce to the case where
$\Spec B$ is an integral scheme and $\p=(0)$. Now,
we localize at $(0)$ to get $\phi': B_{(0)} \to \phi(B\setminus 0)^{-1}A$ and
corresponding $\pi'$. Let $\q \in \pi^{-1}((0))$. Then
$ff'/bb' \in \q$ implies $ff' \in \q$ and
$f$ or $f'$ is in
$\q$. Thus we get a prime ideal of $A$.
This process is clearly invertible as well, giving a bijection between primes
in $A$ and primes in $\phi(B\setminus 0)^{-1}A$. Since
finiteness is preserved by quotient and localization, we get a finite morphism
over a field whose fiber is in one to one correspondence with the original
fiber.

\section{7.3.L}
The ring map is given by $\C[x] \to C[x]_{(0)}$. This is obviously not finite.

\section{7.3.M}
An arbitrary closed subscheme is locally a closed subscheme of an affine
scheme. Hence we may work over the affine case. Now, if we are given ideals as
in the hint, we may quotient $\pi^{\#}$ to get an integral
extension by 7.2.B(b). By the lying over theorem, this is surjective. Hence the
closed set cut out by $I$ maps surjectively to the closed
set cut out by $J$, and hence the map is closed. \qed

\section{7.3.N}
We note that an integral relation for $a$ will also be an
integral relation for $a \otimes c$. Since integrality is closed
under addition and multiplication, and since every tensor is a sum of pure
tensors, we have our result. \qed

\section{7.3.O}
We first note that by applying the affine communication lemma and 5.3.3(b) we
have affine-locality on the source.

Now by way of the affine communication lemma, consider $\phi: \Spec B \to \Spec
    A$
finite type. Then, if $f \in A$ we have $\phi_f: \Spec
    B_{\phi^{\sharp}f}
    \to \Spec A_f$
is finite type, corresponding to $\phi^{-1}(\Spec B_f)$. By affine-locality on
the source we have that $\phi$ is finite type over
$\Spec
    A_f$. Now suppose $\phi$ is finite type
over $\Spec A_{f_i}$ with $(f_i)=(1)$. Let
$r \in B$ be arbitrary. Then we have, in $B_{\phi^{\sharp}f_i}$,
that $r=\sum_j a'_j/f_i^{m_j}g_{ij}^{n_j}$ for some finite generating set
$g_{ij}$. I claim that $B$ is finitely
generated with generators $\Union_{i,j} g_{ij}$. Indeed, let
$m$ be the largest power of any $f_i$
appearing in the denominator of an expression for $r$. Let
$\sum_i a_if_i^m = 1$ be a partition of unity subordinate to our open cover.
We then have $a_if_i^mr=\sum_j a_ia'_jf_i^{m-m_j}g_{ij}^{n_j}$. Summing over the $f_i$
we get $\sum_i
    a_if_i^mr=r\sum_ia_if_i^m=r$ This gives us a polynomial over
$A$ in the $g_{ij}$. \qed

\section{7.3.P}
 (a) is trivial from the definitions. For (b) we note that
it is true on the level of affine schemes. We only require that the morphism is
locally of finite type. Then we note that if $\Spec A_i \subseteq \pi^{-1}(\Spec B_i)$ then it is
a finite morphism of affine schemes. Since finite type is defined by this
property holding for all open affines, we have our result.

\section{7.3.Q}
\begin{enumerate}[a.]
    \item Let $\pi: X \to Y$ be the open embedding. We check on locally of finite
          type on an affine cover of the source. Indeed, let $\Spec A_i$ be
          such a cover. Then $\pi$ restricts to a restriction to a
          distinguished subset on $\Spec A_i$, which is trivially finite. If
          $Y$ is locally noetherian, we reduce to the case where it is
          a noetherian affine. Every open subset of a noetherian space is noetherian.
          \qed
    \item Let $\pi: X \to Y, \phi: Y \to Z$ be locally of finite type. Consider an arbitrary
          affine $\Spec C \subseteq Z$. Then we may cover $\phi^{-1}(\Spec
              C)$ with
          affines. By affine-locality on the target, we get that $\phi \after
              \pi$
          is locally of finite type. \qed
    \item Let $\Spec B \subseteq Y$ be a Noetherian affine. By the Hilbert basis
          theorem, noetherianness descends to finitely generated algebras. Thus we have a
          Noetherian cover of X. \qed
\end{enumerate}

\section{7.3.R}
\begin{enumerate}[a.]
    \item We find $\phi^r(f)=f^q$ for all $f \in k[x_1, \dots, x_n]$ since
          $\phi^r|_k(a)=a^q$ (the Frobeneus endomorphism). Let
          $f \in \p$. We wish to show that $f \in \phi^{r,-1}(\p)$. Clearly
          $f^q \in \p$. The inverse image of this is $f$, so
          the result holds. Now, let $f \in \phi^{r,-1}(\p)$. Then $f^q \in \p$,
          so by primality, $f \in \p$. It is clear that this morphism isn't
          the identity since, for example, $x_1^q \neq x_1$. \qed
    \item We note that $\phi$ is not an isomorphism. Indeed,
          $x_1$ has no inverse image. On the level of sets however,
          $F$ has inverse $F^{r-1}$. \qed
    \item Let $\psi$ be given by $\psi(a)=a^p$ on
          $K$ and the identity on monomials. Then we see that
          $\psi \after \pi = \pi \after \psi = \lambda f.f^p$. This is the identity on points, by the same argument
          as in (a). Thus $F$ has a two sided inverse on sets and is
          a bijection. Consider the ideal corresponding to the point
          $p=(a_1, \dots, a_n), a_i \in K \setminus \F_{p^r}$. Then $F^{r}(p)=(a_1^{p^{r}}, \dots,
              a_n^{p^{r}})$ is not the identity since
          the exponent is too low.
\end{enumerate}

\section{7.3.S}
Choose an arbitrary affine cover, define the desired ring morphism and note
that it is well defined since we are in characteristic $p$.
These are easily seen to glue. The morphism is the identity on the level of
sets for the same reason as above.

\section{7.3.T}
Let $\pi: X \to Y$ and let $\Spec A \subseteq \pi^{-1}(\Spec B)$. We wish to show
that if $\pi^{\sharp}: B \to A$ is finitely presented than
$\pi_f^{\sharp}: B_f \to A_f$ is finitely presented. This follows since localization
commutes with quotient. Now suppose $(f_1, \dots, f_n)=(1) \subset B$ with
$\pi_{f_i}^{\sharp}: B_{f_i} \to
    A_{f_i}$ finitely presented. We immediately have finite type, so
we merely show that there is a finite number of relations to be satisfied. Let
$r_{ij}$ be the relations satisfied by $A_{f_i}$.
WLOG we may assume that they have no denominators. We may use the usual
partition of unity argument to glue together these relations into an arbitrary
element of the kernel of $B[x_1, \dots, x_n] \to A$

\section{7.3.V}
We note that an open immersion is locally the restriction to a distinguished
open. This has a single relation, the invertibility relation, taking the form
$ax-1$.

\section{7.3.W}
Apply affine locality on the source followed by affine locality on the target.

\section{7.4.A}
We first note that open and closed sets are both constructible and a finite
union of locally closed sets. Let $U$ be open and
$Z$ be closed. $(U \intersect
    Z)^{\text{c}}=U^{\text{c}} \union
    Z^{\text{c}}$ which is the union
of two locally closed sets. Induction shows that all finite unions of locally
closed sets are constructible. Lastly, let $(U \intersect Z) \intersect (U' \intersect Z')$. Regrouping
gives us a locally closed set. Induction gives us the desired result. \qed

\section{7.4.B}
We note that the closed sets are finite sets of points, and the whole space
(see 3.2.S and 3.4.G). Thus the open sets are infinite. Since the generic
point's closure is the entire space, no intersection with an open set gives
just the generic point.

\section{7.4.C}
\begin{enumerate}[a.]
    \item We prove this for the general sober noetherian space.

          The only if direction is trivial. The empty set trivially satisfies both
          conditions. Let $\coprod_{i=1}^n U_i \intersect Z_i$ be constructible and stable under
          specialization. Now, by 3.6.15 (Closed subsets of Noetherian spaces are covered
          by finitely many irreducibles) we may assume that the $Z_i$
          are irreducible. Since we are sober, each $Z_i$ has a generic
          point. By 3.6.B, each of the $U_i$ contain the generic point
          of $Z_i$. Hence, we can write the subspace as
          $\coprod_{i=1}^n Z_i$. \qed
    \item Suppose $S$ is closed under generization. Let
          $y \in S^{\text{c}}$ be such that $x \in \overline{y}$. If
          $x \in S$ then so would $y$, which is a
          contradiction. Thus $x \in S^{\text{c}}$. \qed
\end{enumerate}

\section{7.4.D}
The given morphism is quasifinite, hence finite type, hence
$X$ is Noetherian. It remains to show that
$\pi$ is integral. We first show affineness. Indeed,
$X$ is discrete. Now, WLOG let $X=\Spec A$ be a
single point and let $x \in A$ be transcendental over
$k$. Then $k[x] \to A$ is clearly finite type,
and $k[x] \to A/\p$ (where $\p$ is the generic
point) induces a finite type, dominant map of schemes. This obviously sends the
generic point to the generic point, but the Chevaley's theorem says the generic
point of $\A_k^1$ must be constructible, contradicting 7.4.A.
\qed

\section{7.4.E}
Note that by 5.3.F, the closed points are dense. Given an open set containing
every closed point, we find its complement must contain no closed points, and
so must be empty. Thus the only constructible set containing every closed point
is the $Y$. \qed

\section{7.4.F}
Lemma: If $M$ is a finitely generated
$A$-module and $S^{-1}M=0$ then there exists
$s \in S$ such that $sM=0$. If
$sm=0$, then if $s=0$ we are done.
Otherwise, take the product of $s_i$ such that they
annihilate a generating set.

Since we are integral, $(0)$ is prime, so we may localize at
it. This gives us a $B_{(0)}$-vector space
$M_{(0)}$, with an isomorphism $\phi: B_{(0)}^{\oplus n} \to M_{(0)}$. By 1.6.G,
in our case localization and Hom commute. This gives us $\phi=\phi'/f$
where $\phi': B^{\oplus n} \to M$. By the above lemma, both $\ker \phi'$
and $\coker \phi'$ are annihilated by some $f'$ and
$f''$. Thus, $M_{ff'f''}$ is free over
$A_{ff'f''}$. Geometrically: if a finitely generated module is free
at a point, then it is free on an open set around that point.

We note that the $\phi'$ arising from this argument can't
globally surjective unless $M$ is free: indeed, by
Noetherian that would imply that $\ker \phi': B^{\oplus l} \to B^{\oplus n}$ (regarded as a
morphism) was the zero morphism, which contradicts the fact that
$B$ is an integral domain. \qed

\section{7.4.G}
It suffices to prove that if $\dagger$ holds for
$B$ then it also holds for $B/I$. Indeed,
the latter is finite over the former, so we may pull back
$M$ to a finite $B$ module. \qed

\section{7.4.H}
Let $m_n \in M_n$. Then $M_{n+1} \ni
    m_{n+1}=m_n+Tm'_n=Tm'_n \mod M_n$. \qed

\section{7.4.I}
By the Hilbert Basis Theorem, $A$ is Noetherian, so
$M_1$ is Noetherian as well. Consider
$K_n=\ker \phi_n\after \dots \after \phi_1$. It is clear that $K_n \subseteq K_{n+1}$. By
Noetherianness there exists an $N$ such that
$K_n=K_{n+1}$ for some all $n \geq N$. In particular, if
$m_N \neq 0$ then $Tm_N \neq 0$ so that
$\ker \phi_N = 0$. \qed

\section{7.4.J}
By 7.4.H, $M_{i+1}/M_i$ is finitely generated over
$A$, so it becomes free over $f_i$ for
some nonzero $f_i$. Since there are only finitely many such
quotients, we may multiply them together to get an $f$ over
which they are all free.

\section{7.4.K}
Following the hint, we note that $M_1/M_0 \cong M_1/0 \cong M_1$ is free by hypothesis.
Then suppose the result is true for all $i < n-1$. Then
$\oplus_{i=0}^{n-2} M_{i+1}/M_i \cong
    M_{n-1}$. We wish to show the following exact sequence splits:
\[
    \begin{tikzcd}
        0 \ar[r] & M_{n-1} \ar[r] & M_n \ar[r] & M_n/M_{n-1} \ar[r, "\psi"] & 0
    \end{tikzcd}
\]  Now consider \[
    \begin{tikzcd}
        {M_{n-1}\oplus M_n/M_{n-1}} \\
        {M_n} & {M_n/M_{n-1}}
        \arrow["\pi'",from=1-1, to=2-2]
        \arrow[two heads, from=2-1, to=2-2]
        \arrow["\phi_n"', dashed, from=1-1, to=2-1]
    \end{tikzcd}
\] where
$\pi'$ is the projection onto the second factor. We define
$\phi_n$ as follows. Choose a generating set
$G$ for $M_{n-1}$. Then if
$g \in G$ we set $\phi_n(g)=g$. Now, choose a
generating set $G'$ for $M_n/M_{n-1}$ and let
$g' \in G'$. Then use axiom of choice to choose representatives of
$\pi^{-1}(\psi(g'))$. It is clear this is an isomorphism.

We have the following diagram:
\begin{equation*}
    \begin{tikzcd}
        M_0 \arrow[r] \arrow[d, "\phi_0"] \arrow[rddd, bend right=60] & M_1 \arrow[r] \arrow[d, "\phi_1"] \arrow[ddd, bend right] & M_2 \arrow[r] \arrow[d, "\phi_2"] \arrow[lddd, bend left=60] & \dots \\
        \oplus_{i=0}^0 M_{i+1}/M_i \arrow[rd]                         & \oplus_{i=0}^1 M_{i+1}/M_i \arrow[d]                      & \oplus_{i=0}^2 M_{i+1}/M_i \arrow[ld]                        &       \\
                                                                      & \oplus_{i=0}^\infty M_{i+1}/M_i \arrow[d, "\phi", dotted] &                                                              &       \\
                                                                      & M                                                         &                                                              &
    \end{tikzcd}
\end{equation*}
Where every unlabeled morphism is an inclusion, $\phi$ is
defined by universal property, and the $\phi_n$ are inverted. We
implicitly apply inclusion morphisms as needed in what follows. Let
$m \in M_n \subseteq M$. Then $(\phi \after \phi_n)(m)=m$ so that
$\phi$ is surjective. Let $\phi(m)=0$. Then
clearly $m=0$ in the $M_i$. Thus
$\phi$ is injective. \qed

\section{7.4.L}
We note that all open sets of irreducible schemes are dense. We reduce to the
affine case. Cover $Y$ with $\Spec B_i$ and
cover $\pi^{-1}(\Spec B_i)$ with opens $\Spec A_j$. Suppose
$\Spec A_j \to \Spec B_i$ has the desired property for every
$i, j$. The image of each of these either contains a dense
open of $\Spec B_i$ or it doesn't. If one of them does, then we
note that a dense open of $\Spec B_i$ is a dense open of
$Y$ and we are done. If none of them do, then we are also
done since a dense open subset of $\im \pi$ would give a dense
open subset of one of these.

We are thus reduced to $\pi: \Spec A \to \Spec B$ where $A, B$
are Noetherian, and $B$ is irreducible. Let
$\phi: B \to A$ be the induced morphism of rings. Let
$\psi: A \to A/\sqrt{(0)}$ be reduced ring corresponding to
$A$. Then clearly $\sqrt{(0)} \subset \ker \psi \after \phi \subset B$, so that by
3.2.P(a) we get an induced morphism $\phi': B/\sqrt{(0)} \to A/\sqrt{(0)}$. Since the topology
on both of these is identical to the original topology, we may assume WLOG that
$\phi=\phi'$ and that $A, B$ are integral domains.

Now, following the hint we get $\phi_f: B_f \to A_f$ where
$A_f$ is a free $B_f$ module. If
$A_f$ has dimension zero, then $\pi^{-1}(D(f)) \subseteq \Spec A$ is the
empty set and we are done. We wish to show that $\pi_f: \Spec A_f \to \Spec B_f$ is
surjective. Let $x \in \ker \phi_f$. Let $a_i$ be a
generating set, let $1=\sum_i b_ia_i$ where $b_i \neq 0$. Then
$xa_i=0$. By definition of a basis, $x=0$.
Thus $\phi_f$ is injective. Since it is clearly integral, it is
an integral extension. Thus by the Lying Over theorem (7.2.5),
$\pi_f$ is surjective.

\section{7.4.M}
We first reduce to the statement that the image of a closed subset is
constructible. Indeed, if $\Union_{i=1}^n (U_i \intersect Z_i)$ is constructible then
$(U_i \intersect Z_i)$ is as well. Considering $U_i$ as an
open subscheme, it suffices to show that the image of $Z_i$
is constructible. We might as well take $U_i$ to be affine,
say $\Spec A$ so that $Z_i=\V(I)$. Then it is clear
that $\Spec A/I$ comes with an inclusion into
$\Spec A$ whose image is $Z_i$. \qed

\section{7.4.N}
The above argument reduces to the case of affine $X$. Cover
$Y$ with finitely many affines. Then the union of
constructible subsets of each of these is constructible. \qed

\section{7.4.O}
We reduce to the irreducible case by choosing a irreducible component, taking
the inverse image of it, and factoring the ring morphism in the same way aas in
7.4.L. Suppose $U \subseteq \im \pi$. If $U=\im \pi$ we are done,
so we must have $U^{\text{c}} \intersect \im \pi \neq \emptyset$. In the case $U \not \subseteq \im \pi$ we
also have $U^{\text{c}} \intersect \im \pi \neq \emptyset$. Replacing $Y$ with
$U^{\text{c}}$ and $X$ with the inverse image
(using the same method as in 7.4.L), we can repeat the above argument. This
gives us a descending chain of closed sets $U^{\text{c}}$ which must
stabilize by Noetherianness. Stabilizing implies that $U=\im \pi$
eventually. Thus, on each irreducible component we have an open subset of a
closed set, which is constructible. In addition, each time we took the first
path during our induction, we left an open subset of a closed set behind.
Taking the (necessarily finite) union of all these components gets us a
constructible set. \qed

\section{7.4.P}
Let $k=\overline{k}$. Let $g=\Pi_i g_i$. The subset of
interest in the problem are $\V((f_i)) \intersect \D(g)$. This is clearly an affine
scheme given by $k[X, Y] \to k[X, Y]_g \to
    k[X, Y]_g/(f_i)$. This gives the injection of schemes
described in the hint. We define $\pi^{\sharp}$ by the composition of
$k[Y] \to k[X, Y]$ with the above ring morphism. The given diagram
obviously commutes, and the inverse image of a non-closed point is the empty
set. The result follows trivially. \qed

\end{document}
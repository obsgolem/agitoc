Proof of the Nullstellensatz

\begin{theorem}
    Let $k$ be a field. Let $R=k[x_1, \dots, x_n]/\m = k[a_1, \dots, a_n]$ be a finitely
    generated $k$-algebra. Suppose $R$ is a
    field. Then each $a_i$ is algebraic over $k$.
\end{theorem}
\begin{proof}
    Recall that the difference between an algebraic element and an integral element
    is that the former satsifies an arbitrary polynomial equation, while the latter
    satsifies a monic polynomial equation.

    Note that $k[a_1, \dots, a_j] \subseteq R$ is still a finitely generated
    $k$-algebra (though not necessarily a field). Note also that
    $R=k(a_1, \dots, a_j)[a_{j+1}, \dots, a_n]$ and that $R$ is finitely generated
    over $k(a_1, \dots, a_j)$. Finally, note that $x_n$ is
    algebraic over $k(x_1, \dots, x_{n-1})$.

    Suppose that $a_2, \dots, a_n$ are algebraic over $k(a_1)$
    and $a_1$ is not algebraic over $k$. Then
    $a_2, \dots a_n$ is algebraic over $k[a_1]$ by clearing
    denominators. Localizing at these denominators, we get that
    $a_2,\dots a_n$ are integral over $A=k[a_1]_r \subseteq R$ for some
    $r \in k[a_1]$. Note that $A$ is clearly finitely
    generated over $k$. Since $R$ is a field
    and the $a_i$ are integral over $A$, we
    must have that $A$ is a field. But clearly
    $A \subseteq k(a_1)$, so $A=k(a_1)$. But the field of fractions
    of a PID with infinitely many primes is not finitely generated (a PID is a UFD,
    so we note that the LCM of the denominators has finitely many primes). This is
    a contradiction, so $a_1$ must be algebraic.

    Suppose $x_i, \dots x_n$ are algebraic over $k(x_1, \dots x_{i-1})=k(x_1, \dots
        x_{i-2})(x_{i-1})$. By
    the above lemma, we must have $x_{i-1}, \dots, x_n$ are algebraic over
    $k(x_1, \dots x_{i-2})$. The result follows by induction. \qed
\end{proof}
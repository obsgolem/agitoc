\documentclass{article}
\usepackage[utf8]{inputenc}

\usepackage{mathtools}
\usepackage{amsthm}
\usepackage{amssymb}
\usepackage{dsfont}
\usepackage{array}   % for \newcolumntype macro
\usepackage{calligra}
\usepackage{tikz-cd}
\usepackage{mathrsfs}

\newcolumntype{L}{>{$}l<{$}} % math-mode version of "l" column type

\newcommand{\Rho}{\mathrm{P}}

\newcommand{\iso}{\simeq}
\newcommand{\after}{\circ}
\newcommand{\catname}[1]{\mathbf{#1}}
\newcommand{\intersect}{\cap}
\newcommand{\union}{\cup}
\newcommand{\Intersect}{\bigcap}
\newcommand{\Union}{\bigcup}
\newcommand{\wa}[1]{\langle #1 \rangle}
\newcommand{\ds}[1]{\mathds{#1}}
\newcommand{\Z}{\mathds{Z}}
\newcommand{\Q}{\mathds{Q}}
\newcommand{\R}{\mathds{R}}
\newcommand{\C}{\mathds{C}}
\newcommand{\p}{\mathfrak{p}}
\newcommand{\q}{\mathfrak{q}}
\newcommand{\ai}{\mathfrak{a}}
\newcommand{\bi}{\mathfrak{b}}
\newcommand{\m}{\mathfrak{m}}
\newcommand{\defeq}{\vcentcolon=}

\DeclareMathOperator{\Hom}{\mathscr{H}\text{\kern -3pt {\calligra\large om}}\,}
\DeclareMathOperator{\Fsh}{\mathscr{F}}
\DeclareMathOperator{\Gsh}{\mathscr{G}}
\DeclareMathOperator{\Hsh}{\mathscr{H}}
\DeclareMathOperator{\Osh}{\mathscr{O}}
% \DeclareMathOperator{\ker}{ker}
\DeclareMathOperator{\coker}{coker}
\DeclareMathOperator{\im}{im}
\DeclareMathOperator{\Spec}{Spec}
\DeclareMathOperator{\V}{V}
\DeclareMathOperator{\D}{D}


\title{Chapter 9 Problems}
\author{Josiah Bills}
\date{October 2020}

\begin{document}

\maketitle

\section{9.1.A}
Consider the following diagram:
\[
    \begin{tikzcd}
                          & Y \ar[d] \ar[ddr]               \\
        X \ar[r] \ar[rrd] & X \coprod Y \ar[dr, dotted]     \\
                          &                             & Z
    \end{tikzcd}
\]
We may define the dotted morphism on $Y$ as the morphism on the top and similarly for the morphism on $X$. Since morphisms glue, we get a unique morphism making the diagram commute. \qed

\section{9.1.B}
$\Spec$ is a contravariant equivalence of categories. \qed

\section{9.1.C}
Let $Z$ be arbitrary, and let $f: Z \to X$. Consider the following diagram:
\[
    \begin{tikzcd}
        h_X(X) \ar[d, "\eta_X", swap] \ar[r, "\_ \after f"] & h_X(Z)  \ar[d, "\eta_Z"] \\
        h_Y(X)                  \ar[r, "\_ \after f", swap] & h_Y(Z)
    \end{tikzcd}
    \begin{tikzcd}
        \text{id} \ar[d] \ar[r] & f \ar[d]      \\
        \phi             \ar[r] & \phi \after f
    \end{tikzcd}
\]
Obviously, the natural transformation is uniquely determined by the destination of the identity. \qed

\section{9.1.D}
Suppose we have two morphisms of schemes $\sigma_X: X \to Z, \sigma_Y: Y \to Z$. Let $W$ be arbitrary. Then $h_X \times_{h_Z} h_Y(W)$ is given by a pair of morphsims $(f_X, f_Y)$ such that $f_X \after \sigma_X=f_Y \after \sigma_Y$. Let $U_i$ be an open cover of $W$. Consider a collection of morphisms $(f_{Xi}, f_{Yi})$ such that $f_{Xi} \after \sigma_X=f_{Yi} \after \sigma_Y$ for all $i$ and such that the components individually satisfy gluing. Then they glue to a pair of morphisms that satisfy the pullback condition. Identity is likewise trivial. \qed

\section{9.1.E}
The if direction follows from the existence of pullbacks of open embeddings (7.1.B). For the other direction, take $X=Z$ and pull back along the identity. \qed

\section{9.1.F}
\begin{enumerate}[a.]
    \item Composition of pullback squares is a pullback square. \qed
    \item Pullback $h'$ along $h''$, then consider the pullback $h' \after h_X$. You once again get a composed pullback square. \qed
    \item All the pullbacks are of representable functors, so we may perform the pullback in $\catname{Scheme}$ where the result is obvious. \qed
\end{enumerate}

\section{9.1.G}
\section{9.1.H}

\section{9.1.I}
Let $h_{U_i}$ be the open cover. Then the zariski condition ensures we may glue the $U_i$ together to get a scheme $X$. We need to show that $h \iso h_X$. For this we note that by the yoneda lemma, $\hom(h_X, h) \iso h(X)$. Now, choose the natural transformation which sends the identity to the identity.

\end{document}
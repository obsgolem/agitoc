\documentclass{article}
\usepackage[utf8]{inputenc}

\usepackage{mathtools}
\usepackage{amsthm}
\usepackage{amssymb}
\usepackage{dsfont}
\usepackage{array}   % for \newcolumntype macro
\usepackage{calligra}
\usepackage{tikz-cd}
\usepackage{mathrsfs}

\newcolumntype{L}{>{$}l<{$}} % math-mode version of "l" column type

\newcommand{\Rho}{\mathrm{P}}

\newcommand{\iso}{\simeq}
\newcommand{\after}{\circ}
\newcommand{\catname}[1]{\mathbf{#1}}
\newcommand{\intersect}{\cap}
\newcommand{\union}{\cup}
\newcommand{\Intersect}{\bigcap}
\newcommand{\Union}{\bigcup}
\newcommand{\wa}[1]{\langle #1 \rangle}
\newcommand{\ds}[1]{\mathds{#1}}
\newcommand{\Z}{\mathds{Z}}
\newcommand{\Q}{\mathds{Q}}
\newcommand{\R}{\mathds{R}}
\newcommand{\C}{\mathds{C}}
\newcommand{\p}{\mathfrak{p}}
\newcommand{\q}{\mathfrak{q}}
\newcommand{\ai}{\mathfrak{a}}
\newcommand{\bi}{\mathfrak{b}}
\newcommand{\m}{\mathfrak{m}}
\newcommand{\defeq}{\vcentcolon=}

\DeclareMathOperator{\Hom}{\mathscr{H}\text{\kern -3pt {\calligra\large om}}\,}
\DeclareMathOperator{\Fsh}{\mathscr{F}}
\DeclareMathOperator{\Gsh}{\mathscr{G}}
\DeclareMathOperator{\Hsh}{\mathscr{H}}
\DeclareMathOperator{\Osh}{\mathscr{O}}
% \DeclareMathOperator{\ker}{ker}
\DeclareMathOperator{\coker}{coker}
\DeclareMathOperator{\im}{im}
\DeclareMathOperator{\Spec}{Spec}
\DeclareMathOperator{\V}{V}
\DeclareMathOperator{\D}{D}


\title{Chapter 9 Problems}
\author{Josiah Bills}
\date{October 2020}

\begin{document}

\maketitle

\section{7.1.A}
Trivial \qed

\section{7.1.B}
\[
    \begin{tikzcd}
        X \ar[drr, bend left] \ar[ddr, bend right] \ar[dr, dotted] &
                                                                   &
        \\
                                                                   & U \times_Z
        Y \ar[d] \ar[r, hook]                                      & Y \ar[d,
            "\pi"]
        \\
                                                                   & U \ar[r,
        hook]                                                      & Z
    \end{tikzcd}
\]

Define the dotted morphism by taking the inverse image of $U$, noting that it
goes to $\pi^{-1}(U)$ under the top morphism. Thus it restricts to a morphism
to that scheme. The product is obviously an open subscheme. \qed

\section{7.1.C}
Recall that noetherianness descends to open subsets. Given a point in the
subset you are embedding, take a noetherian neighborhood and intersect it with
the open subset.

For the last question, consider $\D(x_1, x_2, \dots) \to A^{\infty}_k$. By
3.6.G(b) the former is not quasicompact.

\section{7.1.D}
We need to exhibit a morphism such that on some open cover it is an open
immersion. Consider the line with the doubled origin projecting onto the affine
line. As an open cover, take the two standard open covers. Then on each open
set it is an isomorphism.

\section{7.2.A}
We first note that surjective ring morphisms are integral, since any ring is
integral over itself. Therefore, we may decompose an arbitrary morphism into an
integral surjection and potentially non-integral injection. Thus we may replace
$B$ with $\phi(B)$.

We note that if $(b_1, \dots, b_n)=(1)$ then $(b_1^m, \dots, b_n^m)=(1)$. This
follows from $\Union_i \D(b_i)=\Spec B$ and $\D(b_i)=\D(b_i^m)$. Now, let $a
    \in A$. Then over $B_{b_i}$ we have
\[
    a^n+\sum_{j=1}^{n-1} a^jb'_j/(b_i)^{m_j}=0
\]
for some $b'_j$ and $m_j$. Clearing denominators and subtracting the sum gives
us an equation of the desired form. Now, let $n$ be the largest power of $a$
appearing in one of these integral dependence equations. Then for each $i$ we
have an equation of the form
\[
    a^nb_i^{m_i}=\sum_{j=1}^{n-1} a^jb'_{ij}
\]
Now, let $\sum_i b''_ib_i^{m_i}=1$. Then
\begin{align*}
    a^n & =a^n\sum_i b''_ib_i^{m_i}          \\
        & =\sum_i a^nb''_ib_i^{m_i}          \\
        & =\sum_i\sum_{j=1}^{n-1} a^jb'_{ij}
\end{align*}
Which is an equation of integral dependence for $a$ over $B$. \qed

\section{7.2.B}
\begin{enumerate}[a.]
    \item Let $a/\phi(b)^m \in A_{\phi(b)}$. Then we have an integral relation
          for $a$, $\sum_{i=0}^n a^ib_i=0$. Multiply this by $1/\phi(b)^{nm}$
          and recall
          that $1/\phi(b)^{nm} \in \phi(B)_{\phi(b)}$. This gives an integral
          relation
          for $a/\phi(b)^m$ over $B_b$.

          Quotient of $A$ is given by a finite, hence integral morphism, so by
          7.2.C we have that the overall morphism is integral.

          Let $a \in A/\bi^{\text{e}}_{\phi}$. Then we have an integral
          relation $\sum_{i=0}^n a^ib_i=0$ in $A$. Apply the projection to this
          to get
          the desired integral relation.

          We note that $1/t$ is not integral over $k[t]$. Suppose otherwise:
          Then $1/t^n=\sum_{i=0}^{n-1} p_i(t)/t^i$, so that
          $1/t=\sum_{i=0}^{n-1}
              p_i(t)t^{n-1-i}$. But the latter is contained in $k[t]$. Thus,
          $k[t] \to
              k[t]_t$ is not integral. \qed

    \item Let $\phi : B \to A$ be an integral extension and let $\phi' :
              S^{-1}B \to \phi(S)^{-1}A$ be the localization. Let
          $\phi'(a/b^n)=0$. Then
          $0=\phi(a)/\phi(b)^n$ so that $\phi(s)\phi(a)=0$ for some $\phi(s)
              \in
              \phi(S)$. Thus, by injectivity of $\phi$, $sa=0$ so that
          $a/b^n=0$. Thus
          $\phi'$ is injective.

          Recall from above that localization of $A$ is not necessarily
          integral. The morphism given in the hints is clearly surjective but
          not
          bijective, thus it is in particular not injective (it is still
          integral). \qed
    \item Lemma: let $\phi: B \to A$ be an integral extension and let $\ai
              \subset A$ be an ideal. Then $\phi': B/\ai^{\text{c}}_{\phi} \to
              A/\ai$ is an
          integral extension. By 7.2.B(a) it is integral, so we merely need to
          show it is
          injective. Let $\phi'(b)=0$. Then $\phi(b) \in \ai$ so that $b \in
              \ai^{\text{c}}_{\phi}$. In other words, $b = 0$ in the quotient.

          We have $\pi: A/\ai^{\text{ce}}_{\phi} \to A/\ai$ given by
          $\pi(a)=a$. This is well defined: if $a-b \in \ai^{\text{ce}}_{\phi}$
          then
          $\pi(a-b) \in \ai$. Note that $\ker \pi = \ai$.

          In the lemma above, $\phi' = \pi \after \phi''$ factors through
          $\phi'': B/\ai^{\text{c}}_{\phi} \to A/\ai^{\text{ce}}_{\phi}$. To
          see this, we
          note that if $b \in \ai^{\text{c}}_{\phi}$ then $\phi(a) \in
              \ai^{\text{ce}}_{\phi}$. We recall that for any ideal,
          $\ai^{\text{cec}}_{\phi}=\ai^{\text{c}}_{\phi}$. Thus, $\ker \phi'' =
              (0)$.
          \qed

\end{enumerate}

\section{7.2.C}
We first show the following result: let $A$ be a finitely generated $B$-algebra
with each of its generators integral. Then $A$ is finite over $B$. We reduce to
the 1D case immediately. Consider $B[a]$. By lemma 7.2.1, this must be finite
(it is the smallest algebra containing both $B$ and $a$).

Now, let $a \in A$ be arbitrary. Then $a$ satisfies a relation
$a^n+b_{n-1}a^{n-1}+\dots$. Then clearly $a$ is integral over
$C[b_{n-1},\dots]$. From the above lemma, $C[a, b_{n-1},\dots]$ is finite. But
finite morphisms are integral, hence $a$ is integral over $C$. \qed

\subsection*{Second proof}

Let $a \in A$ be arbitrary and let $\phi: C \to B, \pi: B \to A$. Now, by lemma
7.2.1, we have a linear relation $a=\sum_i b_ia_i$. Repeat again for the $b_i$,
letting $b_i=\sum_j c_jb_{ij}$ for a generating set dependent on $i$, $b_{ij}$.
Then $a=\sum_i \sum_j c_j(b_{ij}a_i)$. Then clearly the $b_{ij}a_i$ form a
generating set for a finite subalgeba containing $a$. \qed

\section{7.2.D}
Consider two integral elements $a, b \in A$. Then by the above lemma these are
contained in $\phi(B)[a, b]$ which is a in particular a subalgebra. This holds
for arbitrary integral elements, so the set of all integral elements is a
subalgebra.

\section{7.2.E}
The inverse image of the zero ideal under such a map is the zero ideal. Thus,
the only prime ideal of $B$ is the zero ideal. But in that case it is maximal,
hence the ring is a field. Let $b \in B$ be nonzero. Then surely $1/b \in A$,
satisfies an integral dependence relation $1/b^n=\sum_{i=0}^{n-1} b_ib^i$. Then
we have $1/b=\sum_{i=0}^{n-1} b_ib^{n-1-i} \in B$. \qed

\section{7.2.F}
Consider $\p_m$ and $\q_m \subset \q_{m+1}$. By the lemma in 7.2.B(c) we may
quotient the integral extension by $\p_m$ to reduce to the integral domain
case. We then have $(0) \subset \q_{m+1}/\q_m$. Apply the lying over theorem to
get the desired ideal. The full result follows by induction.

\section{7.2.G}
Given the following exact sequence
\[
    \begin{tikzcd}
        0 \ar[r] & N \ar[r] & M \ar[r] & M/N \ar[r] & 0
    \end{tikzcd}
\]
we tensor with $A/I$ to get
\[
    \begin{tikzcd}
        N/IN \ar[r] & M/IM \ar[r] & (M/N)/(IM/N) \ar[r] & 0
    \end{tikzcd}
\]
The first arrow is surjective by hypothesis and the second arrow is surjective
by exactness. This means that $(M/N)/(IM/N)=0$, or $IM/N=M/N$. Thus by Nakayama
2, $M/N=0$ so that $M=N$. \qed

\section{7.2.H}
Let $N=(f_1, \dots, f_n) \subseteq M$. Then $N/mN \to M/mM$ is surjective since
the images of the $f_i$ generate $M/mM$. Thus by Nakayama 3, $M=N$. \qed

\section{7.2.I}
We argue by contrapositive. Let $M$ be an $S[r]$ module which is finitely
generated as an $S$ module. Assume that $r$ is not integral. Then $M$ is not
faithful. Indeed, $\phi(x)=rx$ is a homomorphism. Applying Cayley-Hamilton to
it we get an equation of the form $(r^n+\sum_{i=0}^{n-1} s_ir^i)x=0$. But since
$r$ is not integral, the coefficient is not zero, as desired. \qed

\section{7.2.J}
Trivial. \qed

\section{7.3.A}
Trivial.

\section{7.3.B}
Noetherian spaces are quasicompact, and noetherian descends to open subsets.
Thus every open subset of a Noetherian scheme is quasicompact. This proves part
(a). By the same vein, quasiseparated descends to open subsets, proving part
(b). \qed

\section{7.3.C}
\begin{enumerate}[a.]
    \item Let $U_i=\Spec A_i$. Let $f_{ij} \in A_i$ be such that $(f_{ij})=A$.
          Then if $\pi^{-1}(U_i)$ is quasicompact then so is
          $\pi^{-1}(U_{if_{ij}})$. Indeed, by
          quasicompactness of $\pi^{-1}(U_i)$ we may cover it with affine
          schemes $\Spec
              B_{ik}$, getting an induced morphism \[\pi_{ik}: \Spec B_{ik} \to
              \Spec A_i\]
          Since this comes from a morphism of affine schemes, we have
          $\pi_{ik}^{-1}(U_{if_{ij}})=\Spec B_{ik\pi^{\#}f_{ij}}$. By varying
          $k$ we get an open cover of
          $\pi^{-1}(U_{if_{ij}})$.

          The converse also holds, note that $\pi^{-1}(U_i)=\Union_j
              \pi^{-1}(U_{if_{ij}})$, which is a finite union of quasicompact
          spaces.
          \qed
    \item We first note that quasiseparatedness is inherited by open sets.
          Indeed, let $X$ be quasiseparated, let $U \subseteq X$ be open, and
          let $U_1,
              U_2 \subseteq U$ be quasicompact in $U$. Then $U_1$ and $U_2$ are
          quasicompact
          in $X$, hence their intersection is quasicompact in $X$, hence also
          in $U$.

          Let $U_i=\Spec A_i$. Let $f_{ij} \in A_i$ be such that $(f_{ij})=A$.
          Then $\pi^{-1}(U_i)$ is quasiseparated iff $\pi^{-1}(U_{if_{ij}})$ is
          also.
          The forward direction follows by the above note. For the converse,
          let $V_1,
              V_2 \subseteq \pi^{-1}(U_i)$ be quasicompact. Then we may cover
          their
          intersection with a finite number of quasicompact intersections,
          $\pi^{-1}(U_{if_{ij}}) \intersect V_1$.
\end{enumerate}

\section{7.3.D}
Recall that affine schemes are quasicompact and quasiseparated (5.1.G).

\section{7.3.E}
Consider the restriction map $\Gamma(X, \Osh_X) \to \Gamma(X_s, \Osh_X)$. On a
given affine cover of $X_s$ we have that $s$ is invertible, thus it is
invertible on $X_s$. Now, use the universal property of localization to define
a morphism as desired. \qed

\section{7.3.F}
Let $Y=\Spec A \setminus Z$. Then the open embedding $\iota: Y \to \Spec A$ is
affine. Indeed, $Y$ is covered by open sets $Y_{f_i}$. These may in turn be
covered by affine open sets $\Spec A_{jf_i}$. Then the $\Spec A_j$ cover $\Spec
    A$, and $\iota^{-1}(\Spec A_j)=\Spec A_{jf_i}$ which is affine. \qed

\section{7.3.G}
We first note the following property of affine morphisms: If $\mathbf{P}$ is a
property of morphisms that is local on the target, then it suffices to check
this for affine schemes.

Now, it is trivial to check that the localization of a finite morphism of rings
is finite. For the converse, we may use a similar argument to that used in
$7.2.A$. \qed

\section{7.3.H}
We note that $X$ must be affine, say $\Spec A$. We then have a finite
$k$-algebra $A$. If $A$ is an integral domain, then it is a field (by 3.2.G)
and we are done. Recall that finiteness passes to quotients. Hence $A/\p$ is a
integral domain which is a finite $k$-extension, hence $\p$ is maximal. Thus
every point is closed, and thus the space is cofinite. Since $A$ is finite over
$k$ it is obviously finitely generated, and hence noetherian. Hence by 3.6.15
it may be written as a finite union of irreducible subsets. But we just
identified these as the points of $\Spec A$. Hence $\Spec A$ is a finite,
discrete space. \qed

\section{7.3.I}
This is trivial on morphisms of affine schemes, hence for all schemes.

\section{7.3.J}

\section{7.3.K}
We first note that the arguments in 5.2.B(a) apply to finite morphisms. Let
$\phi: B \to A$ be a finite morphism, and let $\p \subseteq B$ be prime. Then
we have a finite morphism $\phi_{\p}: B/\p \to A/\p^{\text{e}}_\phi$ and a
corresponding morphism of schemes $\pi_{\p}$. Then $\pi^{-1}(\p)$ is in one to
one correspondence with $\pi_{\p}^{-1}((0))$. So we may reduce to the case
where $\Spec B$ is an integral scheme and $\p=(0)$. Now, we localize at $(0)$
to get $\phi': B_{(0)} \to \phi(B\setminus 0)^{-1}A$ and corresponding $\pi'$.
Let $\q \in \pi^{-1}((0))$. Then $ff'/bb' \in \q$ implies $ff' \in \q$ and $f$
or $f'$ is in $\q$. Thus we get a prime ideal of $A$. This process is clearly
invertible as well, giving a bijection between primes in $A$ and primes in
$\phi(B\setminus 0)^{-1}A$. Since finiteness is preserved by quotient and
localization, we get a finite morphism over a field whose fiber is in one to
one correspondence with the original fiber.

\section{7.3.L}
The ring map is given by $\C[x] \to C[x]_{(0)}$. This is obviously not finite.

\section{7.3.M}
An arbitrary closed subscheme is locally a closed subscheme of an affine
scheme. Hence we may work over the affine case. Now, if we are given ideals as
in the hint, we may quotient $\pi^{\#}$ to get an integral extension by
7.2.B(b). By the lying over theorem, this is surjective. Hence the closed set
cut out by $I$ maps surjectively to the closed set cut out by $J$, and hence
the map is closed. \qed

\section{7.3.N}
We note that an integral relation for $a$ will also be an integral relation for
$a \otimes c$. Since integrality is closed under addition and multiplication,
and since every tensor is a sum of pure tensors, we have our result. \qed

\section{7.3.O}
\section{7.3.P}
 (a) is trivial from the definitions. For (b) we note that it is true on the
level of affine schemes. We only require that the morphism is locally of finite
type. Then we note that if $\Spec A_i \subseteq \pi^{-1}(\Spec B_i)$ then it is
a finite morphism of affine schemes. Since finite type requires this to hold
for all open affines, we have our result.

\end{document}
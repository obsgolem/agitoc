\documentclass{article}
\usepackage[utf8]{inputenc}

\usepackage{mathtools}
\usepackage{amsthm}
\usepackage{amssymb}
\usepackage{dsfont}
\usepackage{array}   % for \newcolumntype macro
\usepackage{calligra}
\usepackage{tikz-cd}
\usepackage{mathrsfs}
\usepackage{quiver}
\usepackage[shortlabels]{enumitem}

\newcolumntype{L}{>{$}l<{$}}
% math-mode version of "l" column type

\newcommand{\Rho}{\mathrm{P}}

\newcommand{\iso}{\simeq}
\newcommand{\after}{\circ}
\newcommand{\catname}[1]{\mathbf{#1}}
\newcommand{\intersect}{\cap}
\newcommand{\union}{\cup}
\newcommand{\Intersect}{\bigcap}
\newcommand{\Union}{\bigcup}
\newcommand{\wa}[1]{\langle #1 \rangle}
\newcommand{\bb}[1]{\mathbb{#1}}
\newcommand{\A}{\mathbb{A}}
\newcommand{\Z}{\mathbb{Z}}
\newcommand{\Q}{\mathbb{Q}}
\newcommand{\R}{\mathbb{R}}
\newcommand{\C}{\mathbb{C}}
\newcommand{\F}{\mathbb{F}}
\newcommand{\p}{\mathfrak{p}}
\newcommand{\q}{\mathfrak{q}}
\newcommand{\ai}{\mathfrak{a}}
\newcommand{\bi}{\mathfrak{b}}
\newcommand{\m}{\mathfrak{m}}
\newcommand{\defeq}{\vcentcolon=}

\DeclareMathOperator{\id}{id}
\DeclareMathOperator{\Hom}{\mathscr{H}\text{\kern -3pt
{\calligra\large om}}\,}
\DeclareMathOperator{\Fsh}{\mathscr{F}}
\DeclareMathOperator{\Gsh}{\mathscr{G}}
\DeclareMathOperator{\Hsh}{\mathscr{H}}
\DeclareMathOperator{\Osh}{\mathscr{O}}
% \DeclareMathOperator{\ker}{ker}
\DeclareMathOperator{\coker}{coker}
\DeclareMathOperator{\im}{im}
\DeclareMathOperator{\Spec}{Spec}
\DeclareMathOperator{\Proj}{Proj}
\DeclareMathOperator{\V}{V}
\DeclareMathOperator{\D}{D}

\newcommand*{\rationalto}[1][]{\mathbin{\tikz [baseline=-0.25ex,-latex, dashed,->,densely dashed,#1] \draw [#1] (0pt,0.5ex) -- (1.3em,0.5ex);}}%


\title{Chapter 9.3 Problems}
\author{Josiah Bills}
\date{March 2021}

\begin{document}

\maketitle

\section{9.3.A}
Consider the following diagram: \[\begin{tikzcd}
        Z                                              \\
         & F & X                                       \\
         & x & Y \arrow["{\phi_x}"', from=2-2, to=3-2]
        \arrow["\iota", from=2-2, to=2-3] \arrow["\phi", from=2-3, to=3-3] \arrow["\iota"', from=3-2, to=3-3]
        \arrow["{\pi_1}"', curve={height=12pt}, from=1-1, to=3-2]
        \arrow["{\pi_2}", curve={height=-12pt}, from=1-1, to=2-3]
        \arrow["f"{description}, dashed, from=1-1, to=2-2]
    \end{tikzcd}\] where
$F=\phi^{-1}(x)$ and $\phi_x=\phi|_F$. We define the universal map
by $f(z)=\pi_2(z)$. It is obvious that this diagram commutes in
$\catname{Set}$, so we merely need to show that $f$
is continuous. Let $U \intersect F \subseteq F \subseteq X$ be open in the subspace topology.
Then by continuity of $\pi_2$, $\pi_2^{-1}(U)$ is open in
$Y$. Thus, $f^{-1}(U \intersect F)=\pi_2^{-1}(U)
    \intersect \pi_2^{-1}(F)=\pi_2^{-1}(U)$ is an open subset of
$Z$. \qed

\section{9.3.B}
We reduce immediately to the case of $f: \Spec B \to \Spec A$ with corresponding
ring morphism $\phi: A \to B$. Let $\p \subseteq B$ be prime. Then
clearly $\V(\p) \to \Spec A$ pulls back to the closed immersion
$\V(\p^{\text{e}}) \to \Spec B$. By 3.2.I(a) this is identified with the topological
fiber of that pullback. Thus, we may reduce to the case where
$A$ is an integral domain and $\p = (0)$. Then
the pullback corresponds to the ring $\phi(A \setminus 0)^{-1}B$. Let
$f(\q)=(0)$ so that $\phi^{-1}(\q)=(0)$. Since
$(0)$ is principal, we see that $\q \intersect \im \phi = 0$. Thus
$\q$ avoids $\phi(A \setminus 0)^{-1}$, so that it is a prime
ideal of $\phi(A \setminus 0)^{-1}B$. Finally, suppose $\q \subseteq \phi(A \setminus 0)^{-1}B$ is
prime. Then $\phi(A \setminus 0) \nsubseteq \q$ so that $\phi^{-1}(\q)=(0)$. \qed

\section{9.3.C}
It suffices to show that $\tau \after \iota_p$ (where $\iota_p$
is the inclusion) factors through $\iota_{\tau(p)}$. Indeed, choose an
affine open $\Spec A \subseteq Z$ and an affine open $\Spec B \subseteq \tau^{-1}(\Spec A)$.
Then we have $f \after i: A \to B \to \kappa(p)$. By 3.2.P(a), we may factor this as
$A \to A/\tau(p) \to \kappa(p)$. By universal property of localization, we may factor
this as $A \to A/\tau(p) \to \kappa(\tau(p)) \to
    \kappa(p)$. \qed

\section{9.3.D}
Let $(p) \subset \Z$ be prime. We are interested in
$\Z_{(p)}/(p)_{(p)}
    \otimes_{\Z} \Z[i]$. If $p=0$ then we have
$\Q \otimes_{\Z} \Z[i] \cong
    \Q[i]$, a degree two field extension of $\Q$.
If $p=2$ then we have $\Z_2[i] \cong \Z_2[x]/(x^2+1) \cong
    \Z_2[x]/(x+1)^2$, a double point.

Next, we consider $\Z_p[i]$ for odd primes. We first prove Euler's
criterion, $(\frac{a}{p})=a^{((p-1)/2}$. Indeed, since $\Z_p$ is a
field, $x^2=a$ has at most two solutions over it. There are
$p-1$ units of $\Z_p$. Send
$x \mapsto x^2$. The range of this map are the residues, and at most two
values can map to the same value. Thus the range has size lower bounded by
$p-1/2$. Now, by Fermat's little theorem, $a^(p-1)=1 \mod p$
for all $a$. This factors as $(a^{p-1/2}-1)(a^{p-1/2}+1)$.
Quadratic residues clearly fall into the former case. Now
$(a^{p-1/2}-1)$ has at most $p-1/2$ roots. But then these
must be precisely the quadratic residues. Hence the other factor must be the
quadratic non-residues, as desired.

Now, consider $(-1)^{p-1/2} \mod p$. If we want this to be 1 then
$p-1/2$ must be even, say $2k$. Then
$p=4k+1$ so that $p = 1 \mod 4$. In this case, we have a
number $a$ such that $a^2=-1$. Thus
$\Z_p[x]/(x^2+1) \cong \Z_p[x]/(x+a) \times
    \Z_p[x]/(x-a)$. This is another double point. In the case
$p=3 \mod 4$ we find that $x^2+1$ is irreducible,
hence maximal, hence $\Z_p[i] \cong \F_{p^2}$. This is another degree two field
extension of the field of fractions.

\section{9.3.E}
$k[y] \cong k[x, y]/(y^2-x)$. Thus
\begin{align*}
    k[y] \otimes_{k[x]} k[z] & \cong \frac{k[x,y]}{(y^2-x)} \otimes_{k[x]} \frac{k[x,z]}{(z^2-x)} \\
                             & \cong \frac{k[x,y,z]}{(y^2-x, z^2-x)}                              \\
                             & \cong \frac{k[x,y,z]}{(z^2-y^2)}                                   \\
                             & \cong \frac{k[x,y,z]}{((z-y)(z+y))}
\end{align*}
In characteristic 2 we have $-1=1$ so that we have an
irreducible but non-reduced scheme. \qed

\section{9.3.F}
Suppose we are given a point $[a : b] \in \mathbb{P}^1_k$. Then we have
$xb=ya$, which clearly shows that $(x, y)$ must
lie on a line with slope $a/b$ or $b/a$. Thus
the fibers are the lines with slope determined by the projective point.

We examine the fiber of $(a, b) \in \A^2_k$. Clearly this is the pair
$(a,b), [a : b]$. WLOG assume that $a \neq 0$. Then we may
examine $\D_{u, x} \to \D_x$. This is given by the ring
$\Spec k[x, y, v]_x/(xv-y)$. The projection gives a ring morphism
$k[x, y]_x \to k[x, y, v]_x/(xv-y)$. This is an isomorphism sending $y/x$
to $v$.

On the open set $\D_u$ we have find that the blowup is the
affine scheme $\Spec k[x, y, v]/(xv-y)$. Consider $\V(x)$. This
gives $\Spec k[x, y, v]/(xv-y,x) \cong \Spec
    k[x, y, v]/(x, y) \cong \Spec
    k[v] \cong \Spec k \otimes_{k[x, y]}
    k[x,y,v]/(xv-y)$, i.e. the fiber over $(0, 0)$. On
the open set $\D_v$, we find this subscheme is cut out by
$y$.

\section{9.3.G}
We first reduce to the case of Noetherian $Y$. Indeed, if
the result holds for a Noetherian open subset of $Y$, then
it holds for $Y$. So we assume $Y$ is
Noetherian.

Suppose the result is true for irreducible $Y$. Then for
reducible $Y$ we get a locally closed subset
$U_i \intersect Z_i$ where the result holds for each irreducible component
$Z_i$. Since $Y$ is Noetherian, the index
set is finite. Now note that $\Union_j (\Intersect_i U_i) \intersect Z_j = \Intersect_i U_i$ is open in
$Y$, and so the result holds for $\Intersect_i U_i$.
Thus we may assume $Y$ is irreducible.

Now, following the hint, we show that $A$ is integral over
$B$. We have that $A \otimes_B \operatorname{K}(B)$ is integral over
$\operatorname{K}(B)$. Writing $A=B[X]/I$ we find an integral
relation for $x_i \otimes 1$, for example $f_i(x_i)=x_i^n \otimes 1+\sum_{j=0}^{n-1} x_i^j \otimes b_j/b_j'=0$.
Replacing $B$ with the localization at all the denominators
in these relations (and also replacing $A$ with the
relevant inverse image), we find
\begin{align*}
    f_i(x_i) & =x_i^n \otimes 1+\sum_{j=0}^{n-1} x_i^j \otimes b_j  \\
             & =x_i^n \otimes 1+\sum_{j=0}^{n-1} b_jx_i^j \otimes 1 \\
             & = (x_i^n+\sum_{j=0}^{n-1} b_jx_i^j) \otimes 1        \\
             & =0
\end{align*}

We find that either $x_i^n+\sum_{j=0}^{n-1} b_jx_i^j=0$, in which case we are done, or it
is annihilated by some $b_i''$ by replacing 1 with
$b_i''/b_i''$. Replacing $B$ with the
localization at these $b_i''$, we recall that an element of the
localization is zero iff the product with a denominator is zero. Hence,
$f_i(x_i)=0$ in $A$. \qed

\section{9.3.H}
Note that the following diagram is a pushout: \[\begin{tikzcd}
        {\Z[x_1, \dots, x_N]} & B \\{\frac{\Z[x_1,\dots, x_N,y_1, \dots, y_M]}{I}}
                              & {\frac{B[y_1, \dots, y_M]}{I}}
        \arrow[from=1-2, to=2-2]
        \arrow[from=1-1, to=1-2]
        \arrow[from=2-1, to=2-2]
        \arrow[from=1-1, to=2-1]
    \end{tikzcd}\] where the
top map is described by the hint. This shows the result for affine
$X$.

Lemma: Suppose $\phi \after \pi: B \to B[X]/I \to
    B[Y]/J$ is a composition of morphism of rings,
where $X, Y$ are finite sets of variables, and
$I, J$ are finitely generated ideals. Then we may find a ring
$B[X']/I' \cong B[Y]/I$ such that $X \subset X'$ and
$I \subset I'$. Indeed, let $x \in X$. Then we may form
$B[Y, x]/(J+(x-\phi(x)))$. Repeat this for all $x \in X$ and call
this ring $B[X, Y]/I'$. I claim this ring has the desired property.
Let $\iota: B[Y]/J \to B[X, Y]/I'$ be the inclusion. Let $x \in X \subset B[X, Y]/I'$. Then
$\iota(\phi(x))=x$, so that $\iota$ is surjective. Let
$a \in \ker \iota$. Then either $a \in J$ or
$a \in (x-\phi(x))$ for some $x \in X$. But since
$x \notin R[Y]/J$, we must have $a = 0$. Now, the fact
that $I \subset I'$ follows by examining the map
$B[X] \to B[X, Y] \to
    B[X, Y]/I'$.

Now, suppose $X$ is a scheme as in the original problem,
and it is covered by finitely many affine opens, all with affine intersections.
Let $\Spec B[y_1, \dots, y_M]/I$ be such an affine open, and let
$\Spec B[y'_1, \dots, y'_{M'}]/J$ be the intersection with an arbitrary other affine
open. Then by the above lemma we have the following diagram:
\[\begin{tikzcd}
        {\Z[x_1, \dots, x_N]}                                                  & B
        \\{\frac{\Z[x_1,\dots, x_N,y_1, \dots, y_M]}{I}} &
        {\frac{B[y_1, \dots, y_M]}{I}}                                                          \\
        {\frac{\Z[x'_1, \dots x'_{N'}, y'_1, \dots, y'_{M'}]}{I'}}             & {\frac{B[y'_1,
        \dots, y'_{M'}]}{I'}}
        \arrow[from=1-2, to=2-2]
        \arrow[from=1-1, to=1-2]
        \arrow[from=2-1, to=2-2]
        \arrow[from=1-1, to=2-1]
        \arrow[from=2-2, to=3-2]
        \arrow[from=2-1, to=3-1]
        \arrow[from=3-1, to=3-2]
    \end{tikzcd}\] where the upper right map is the map of global sections
and the lower right map is the restriction. The maps on the left are defined by
using the properties of $I$ guaranteed by the lemma.
Applying $\Spec$, we see that the upper square is a pullback,
and the whole rectangle is a pullback, hence the bottom square is a pullback.
Hence, since the lower right map is an open embedding, so is the lower left
map.

TODO: Check triple intersections, show the result when intersections aren't
finite.

\section{9.3.I}
\begin{enumerate}[a.]
    \item We recall that locally closed embeddings are preserved by base change. Now, the
          given morphism $\pi$ is the pullback of a morphism
          $\pi'$ of Noetherian schemes by 9.3.H. Thus the image of
          $\pi'$ is a finite union of locally closed subsets, which may be
          realized by finitely many locally closed embeddings. Now examine the following
          diagram: \[\begin{tikzcd}
                  {H_i'}                         &   & {H_i}
                  \\
                                                 & X &       & {X'}
                  \\{Z'_i} &   & {Z_i}        \\
                                                 & Y &       & {Y'}
                  \arrow[from=2-2, to=4-2]
                  \arrow[from=2-2, to=2-4]
                  \arrow[from=2-4, to=4-4]
                  \arrow[from=4-2, to=4-4]
                  \arrow[hook, from=3-3, to=4-4]
                  \arrow[hook, from=1-3, to=2-4]
                  \arrow[from=1-3, to=3-3]
                  \arrow[hook, from=1-1, to=2-2]
                  \arrow[hook, from=3-1, to=4-2]
                  \arrow[from=3-1, to=3-3]
                  \arrow[from=1-1, to=3-1]
                  \arrow[from=1-1, to=1-3]
              \end{tikzcd}\] Where $\{Z_i\}_{i\in I} \to Y'$ is the covering by
          locally closed embeddings. Clearly the $H_i$ cover
          $X'$, so the $H'_i$ must cover
          $X$. Thus the $Z'_i$ cover the image. \qed
    \item We first note that being a locally closed embedding is stable under base
          change. Cover $Y$ with finitely many affines
          $\{\Spec B_i\}_{i \in I}$ and $X$ with finitely many affines
          $\{\Spec A_{ij}\}_{i \in I,j \in J}$ such that $B_i \to A_{ij}$ realizes
          $A_{ij}$ as finitely presented. By (a), the image of this is a
          finite union of locally closed subsets. Taking the union across all
          $i, j$ (which is finite), we get our result. \qed
\end{enumerate}

\end{document}
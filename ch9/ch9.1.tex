\documentclass{article}
\usepackage[utf8]{inputenc}

\usepackage{mathtools}
\usepackage{amsthm}
\usepackage{amssymb}
\usepackage{dsfont}
\usepackage{array}   % for \newcolumntype macro
\usepackage{calligra}
\usepackage{tikz-cd}
\usepackage{mathrsfs}

\newcolumntype{L}{>{$}l<{$}} % math-mode version of "l" column type

\newcommand{\Rho}{\mathrm{P}}

\newcommand{\iso}{\simeq}
\newcommand{\after}{\circ}
\newcommand{\catname}[1]{\mathbf{#1}}
\newcommand{\intersect}{\cap}
\newcommand{\union}{\cup}
\newcommand{\Intersect}{\bigcap}
\newcommand{\Union}{\bigcup}
\newcommand{\wa}[1]{\langle #1 \rangle}
\newcommand{\ds}[1]{\mathds{#1}}
\newcommand{\Z}{\mathds{Z}}
\newcommand{\Q}{\mathds{Q}}
\newcommand{\R}{\mathds{R}}
\newcommand{\C}{\mathds{C}}
\newcommand{\p}{\mathfrak{p}}
\newcommand{\q}{\mathfrak{q}}
\newcommand{\ai}{\mathfrak{a}}
\newcommand{\bi}{\mathfrak{b}}
\newcommand{\m}{\mathfrak{m}}
\newcommand{\defeq}{\vcentcolon=}

\DeclareMathOperator{\Hom}{\mathscr{H}\text{\kern -3pt {\calligra\large om}}\,}
\DeclareMathOperator{\Fsh}{\mathscr{F}}
\DeclareMathOperator{\Gsh}{\mathscr{G}}
\DeclareMathOperator{\Hsh}{\mathscr{H}}
\DeclareMathOperator{\Osh}{\mathscr{O}}
% \DeclareMathOperator{\ker}{ker}
\DeclareMathOperator{\coker}{coker}
\DeclareMathOperator{\im}{im}
\DeclareMathOperator{\Spec}{Spec}
\DeclareMathOperator{\V}{V}
\DeclareMathOperator{\D}{D}


\title{Chapter 9.1 Problems}
\author{Josiah Bills}
\date{March 2021}

\begin{document}

\maketitle

\section{9.1.A}
Consider the following diagram: \[
    \begin{tikzcd}
        & Y \ar[d] \ar[ddr]
        \\
        X \ar[r] \ar[rrd] & X \coprod Y
        \ar[dr, dotted]                           \\
        &                   & Z
    \end{tikzcd}
\] We may define the dotted
morphism on $Y$ as the morphism on the top and similarly for
the morphism on $X$. Since morphisms glue, we get a unique
morphism making the diagram commute. \qed

\section{9.1.B}
$\Spec$ is a contravariant equivalence of categories. \qed

\section{9.1.C}
Let $Z$ be arbitrary, and let $f: Z \to X$.
Consider the following diagram: \[
    \begin{tikzcd}
        h_X(X) \ar[d, "\eta_X", swap] \ar[r, "\_ \after f"] & h_X(Z)
        \ar[d, "\eta_Z"]                                             \\
        h_Y(X) 	     \ar[r, "\_ \after f", swap]             & h_Y(Z)
    \end{tikzcd}
    \begin{tikzcd}
        \text{id} \ar[d] \ar[r] & f
        \ar[d]
        \\
        \phi		 \ar[r]             & \phi \after f
    \end{tikzcd}
\] Obviously, the natural
transformation is uniquely determined by the destination of the identity. \qed

\section{9.1.D}
Suppose we have two morphisms of schemes $\sigma_X: X \to Z, \sigma_Y: Y \to Z$. Let
$W$ be arbitrary. Then $h_X \times_{h_Z} h_Y(W)$ is given by
a pair of morphsims $(f_X, f_Y)$ such that $f_X \after \sigma_X=f_Y \after \sigma_Y$.
Let $U_i$ be an open cover of $W$.
Consider a collection of morphisms $(f_{Xi}, f_{Yi})$ such that
$f_{Xi} \after \sigma_X=f_{Yi} \after \sigma_Y$ for all $i$ and such that the
components individually satisfy gluing. Then they glue to a pair of morphisms
that satisfy the pullback condition. Identity is likewise trivial. \qed

\section{9.1.E}
The if direction follows from the existence of pullbacks of open embeddings
(7.1.B). For the other direction, take $X=Z$ and pull back
along the identity. \qed

\section{9.1.F}
\begin{enumerate}[a.]
    \item Composition of pullback squares is a pullback square. \qed
    \item Pullback $h'$ along $h''$, then consider the
          pullback $h' \after h_X$. You once again get a composed pullback square.
          \qed
    \item All the pullbacks are of representable functors, so we may perform the pullback
          in $\catname{Scheme}$ where the result is obvious. \qed
\end{enumerate}

\section{9.1.G}
For the moment we assume that fiber products exist. Then we have
$U\times_WV = U\times_ZY \intersect X\times_ZV$. If $\phi: T \to X\times_ZY$ then
$\phi^{-1}(U\times_WV)=\phi^{-1}(U\times_ZY
    ) \intersect
    \phi^{-1}(X\times_ZV)$. Clearly also $\phi^{-1}(U\times_ZY)=(\pi_X \after
    \phi)^{-1}(U)$.

Consider the following diagram: \[\begin{tikzcd}
        {h'}
        \\
        & {h_T}
        \\{h_U\times_{h_W}h_V} &
        & {h_V}
        \\
        & {h_X \times_{h_Z} h_Y } &   &
        {h_Y}
        \\{h_U}                &                         &
        {h_W}
        \\
        & {h_X}                   &   &
        {h_Z}
        \arrow[from=6-2, to=6-4]
        \arrow[from=4-4, to=6-4]
        \arrow[from=4-2, to=6-2]
        \arrow[from=4-2, to=4-4]
        \arrow[hook, from=3-3, to=4-4]
        \arrow[hook, from=5-1, to=6-2]
        \arrow[hook, from=5-3, to=6-4]
        \arrow[from=5-1, to=5-3]
        \arrow[from=3-3, to=5-3]
        \arrow[from=3-1, to=3-3]
        \arrow[from=3-1, to=5-1]
        \arrow[dashed, hook, from=3-1, to=4-2]
        \arrow[from=2-2, to=4-2]
        \arrow[hook, from=1-1, to=2-2]
        \arrow[from=1-1, to=3-1]
    \end{tikzcd}\] We wish to show that
$h'$ is representable and that $h' \to h_T$ is
represented by an open embedding. Consider the following slice of the above
diagram: \[\begin{tikzcd}
        {h_{S_1}}
        \\{h'}                 &
        {h_T}                                        &
        {}
        \\{h_U\times_{h_W}h_V} &
        {h_X\times_{h_Z}h_Y}
        \\{h_U}                &
        {h_X}
        \arrow[from=2-2, to=3-2]
        \arrow[from=2-1, to=3-1]
        \arrow[from=2-1, to=2-2]
        \arrow[from=3-1, to=3-2]
        \arrow[from=3-1, to=4-1]
        \arrow[from=3-2, to=4-2]
        \arrow[hook, from=4-1, to=4-2]
        \arrow[hook, from=1-1, to=2-2]
        \arrow[curve={height=30pt}, from=1-1, to=4-1]
    \end{tikzcd}\] We find $h_{S_1}$ defined by
applying the definition of the open embedding $h_U \to h_X$ to the
vertical morphism. Likewise, we can find an $h_{S_2}$
corresponding to $h_V$. Now, set $S=S_1 \intersect S_2$, as
motivated by the first paragraph. We show $h'=h_S$. Let
$R$ be an arbitrary scheme and consider:
\[\begin{tikzcd}
        {h''(R)}
        \\
        & {h_S(R)}                &
        {h_T(R)} &
        {}
        \\
        & {h_U\times_{h_W}h_V(R)} &
        {h_X\times_{h_Z}h_Y(R)}
        \arrow[from=2-3, to=3-3]
        \arrow[from=2-2, to=3-2]
        \arrow[from=2-2, to=2-3]
        \arrow[from=3-2, to=3-3]
        \arrow[from=1-1, to=3-2]
        \arrow[from=1-1, to=2-3]
        \arrow[dashed, from=1-1, to=2-2]
    \end{tikzcd}\] Let $x \in h''(R)$. Then we get a pair
$(f,g) \in h_U\times_{h_W}h_V(R)$ and a morphism $j \in h_T(R)$ whose image is
$(f, g)$. We wish to show that $\im j \subseteq S$. Applying
the definition of open embedding, we get two morphisms $f': R \to S_1, g': R \to S_2$
both of which map to $j$ (and hence are equal). We made no
choices during this process, so the morphism $x \to f'$ is unique.
\qed

\section{9.1.H}
Trivial

\section{9.1.I}
We first show that open immersions are injective. Let $f_X(x_1)=f_X(x_2)$
with $x_1, x_2 \in h'(X)$. Then, abusing notation, we write
$x_1, x_2: h_X \to h'$. By yoneda, we have $f\after x_1=f\after x_2: h_X \to h$. Now,
pullback $f\after x_1$ along $f$ to get an open
embedding $h_U \to h_X$. Now, by definition, $(f \after x_1)_X(\id)=f(x_1)$.
Clearly, then $(x_1,\id) \in h_U(X)$ (regarded as the pullback). Likewise,
$(x_2, \id) \in h_U(X)$, and these both go to the same element in
$h_X(X)$. Thus they are equal, so that $x_1 = x_2$.

The identity $h_{U_i}(U_i)$ determines an element
$h(U_i)$. By the zariski condition we get an element
$h(X)$. By the yoneda lemma we get a natural transformation
$f: h_X \to h$. We wish to show that this is bijective.

Let $Y$ be an arbitrary scheme with a pullback cover
$V_i$. Let $x \in h(Y)$. Then we abuse notation to
write $x: h_Y \to h$. Now, $\id \in h_Y(Y)$ is determined by
the inclusions $V_i \to Y$. These pullback to the identity on
$h_{V_i}(V_i)$, and determine a morphism $V_i \to U_i$.
These morphisms glue to give a morphism $Y \to X$ which
corresponds to $x$.

Let $f_Y(x_1)=f_Y(x_2)$. A simple diagram chase shows that the restrictions
of $x_1, x_2$ to the open cover are equal, hence they are equal.
\qed

\end{document}
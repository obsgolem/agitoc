\documentclass{article}
\usepackage[utf8]{inputenc}

\usepackage{mathtools}
\usepackage{amsthm}
\usepackage{amssymb}
\usepackage{dsfont}
\usepackage{array}   % for \newcolumntype macro
\usepackage{calligra}
\usepackage{tikz-cd}
\usepackage{mathrsfs}
\usepackage{quiver}
\usepackage[shortlabels]{enumitem}

\newcolumntype{L}{>{$}l<{$}}
% math-mode version of "l" column type

\newcommand{\Rho}{\mathrm{P}}

\newcommand{\iso}{\simeq}
\newcommand{\after}{\circ}
\newcommand{\catname}[1]{\mathbf{#1}}
\newcommand{\intersect}{\cap}
\newcommand{\union}{\cup}
\newcommand{\Intersect}{\bigcap}
\newcommand{\Union}{\bigcup}
\newcommand{\wa}[1]{\langle #1 \rangle}
\newcommand{\bb}[1]{\mathbb{#1}}
\newcommand{\A}{\mathbb{A}}
\newcommand{\Z}{\mathbb{Z}}
\newcommand{\Q}{\mathbb{Q}}
\newcommand{\R}{\mathbb{R}}
\newcommand{\C}{\mathbb{C}}
\newcommand{\F}{\mathbb{F}}
\newcommand{\p}{\mathfrak{p}}
\newcommand{\q}{\mathfrak{q}}
\newcommand{\ai}{\mathfrak{a}}
\newcommand{\bi}{\mathfrak{b}}
\newcommand{\m}{\mathfrak{m}}
\newcommand{\defeq}{\vcentcolon=}

\DeclareMathOperator{\id}{id}
\DeclareMathOperator{\Hom}{\mathscr{H}\text{\kern -3pt
{\calligra\large om}}\,}
\DeclareMathOperator{\Fsh}{\mathscr{F}}
\DeclareMathOperator{\Gsh}{\mathscr{G}}
\DeclareMathOperator{\Hsh}{\mathscr{H}}
\DeclareMathOperator{\Osh}{\mathscr{O}}
% \DeclareMathOperator{\ker}{ker}
\DeclareMathOperator{\coker}{coker}
\DeclareMathOperator{\im}{im}
\DeclareMathOperator{\Spec}{Spec}
\DeclareMathOperator{\Proj}{Proj}
\DeclareMathOperator{\V}{V}
\DeclareMathOperator{\D}{D}

\newcommand*{\rationalto}[1][]{\mathbin{\tikz [baseline=-0.25ex,-latex, dashed,->,densely dashed,#1] \draw [#1] (0pt,0.5ex) -- (1.3em,0.5ex);}}%


\title{Chapter 9 Problems}
\author{Josiah Bills}
\date{March 2021}

\begin{document}

\maketitle

\section{9.1.A}
Consider the following diagram: \[
    \begin{etikzcd}
                          & Y \ar[d] \ar[ddr]
        \\
        X \ar[r] \ar[rrd] & X \coprod Y
        \ar[dr, dotted]                           \\
                          &                   & Z
    \end{etikzcd}
\] We may define the dotted
morphism on $Y$ as the morphism on the top and similarly for
the morphism on $X$. Since morphisms glue, we get a unique
morphism making the diagram commute. \qed

\section{9.1.B}
$\Spec$ is a contravariant equivalence of categories. \qed

\section{9.1.C}
Let $Z$ be arbitrary, and let $f: Z \to X$.
Consider the following diagram: \[
    \begin{etikzcd}
        h_X(X) \ar[d, "\eta_X", swap] \ar[r, "\_ \after f"] & h_X(Z)
        \ar[d, "\eta_Z"]                                             \\
        h_Y(X) 	     \ar[r, "\_ \after f", swap]             & h_Y(Z)
    \end{etikzcd}
    \begin{etikzcd}
        \text{id} \ar[d] \ar[r] & f
        \ar[d]
        \\
        \phi		 \ar[r]             & \phi \after f
    \end{etikzcd}
\] Obviously, the natural
transformation is uniquely determined by the destination of the identity. \qed

\section{9.1.D}
Suppose we have two morphisms of schemes $\sigma_X: X \to Z, \sigma_Y: Y \to Z$. Let
$W$ be arbitrary. Then $h_X \times_{h_Z} h_Y(W)$ is given by
a pair of morphsims $(f_X, f_Y)$ such that $f_X \after \sigma_X=f_Y \after \sigma_Y$.
Let $U_i$ be an open cover of $W$.
Consider a collection of morphisms $(f_{Xi}, f_{Yi})$ such that
$f_{Xi} \after \sigma_X=f_{Yi} \after \sigma_Y$ for all $i$ and such that the
components individually satisfy gluing. Then they glue to a pair of morphisms
that satisfy the pullback condition. Identity is likewise trivial. \qed

\section{9.1.E}
The if direction follows from the existence of pullbacks of open embeddings
(7.1.B). For the other direction, take $X=Z$ and pull back
along the identity. \qed

\section{9.1.F}
\begin{enumerate}[a.]
    \item Composition of pullback squares is a pullback square. \qed
    \item Pullback $h'$ along $h''$, then consider the
          pullback $h' \after h_X$. You once again get a composed pullback square.
          \qed
    \item All the pullbacks are of representable functors, so we may perform the pullback
          in $\catname{Scheme}$ where the result is obvious. \qed
\end{enumerate}

\section{9.1.G}
For the moment we assume that fiber products exist. Then we have
$U\times_WV = U\times_ZY \intersect X\times_ZV$. If $\phi: T \to X\times_ZY$ then
$\phi^{-1}(U\times_WV)=\phi^{-1}(U\times_ZY
    ) \intersect
    \phi^{-1}(X\times_ZV)$. Clearly also $\phi^{-1}(U\times_ZY)=(\pi_X \after
    \phi)^{-1}(U)$.

Consider the following diagram: \[\begin{etikzcd}
        {h'}
        \\
                                                     & {h_T}
        \\{h_U\times_{h_W}h_V} &
                                                     & {h_V}
        \\
                                                     & {h_X \times_{h_Z} h_Y } &   &
        {h_Y}
        \\{h_U}                &                         &
        {h_W}
        \\
                                                     & {h_X}                   &   &
        {h_Z}
        \arrow[from=6-2, to=6-4]
        \arrow[from=4-4, to=6-4]
        \arrow[from=4-2, to=6-2]
        \arrow[from=4-2, to=4-4]
        \arrow[hook, from=3-3, to=4-4]
        \arrow[hook, from=5-1, to=6-2]
        \arrow[hook, from=5-3, to=6-4]
        \arrow[from=5-1, to=5-3]
        \arrow[from=3-3, to=5-3]
        \arrow[from=3-1, to=3-3]
        \arrow[from=3-1, to=5-1]
        \arrow[dashed, hook, from=3-1, to=4-2]
        \arrow[from=2-2, to=4-2]
        \arrow[hook, from=1-1, to=2-2]
        \arrow[from=1-1, to=3-1]
    \end{etikzcd}\] We wish to show that
$h'$ is representable and that $h' \to h_T$ is
represented by an open embedding. Consider the following slice of the above
diagram: \[\begin{etikzcd}
        {h_{S_1}}
        \\{h'}                 &
        {h_T}                                        &
        {}
        \\{h_U\times_{h_W}h_V} &
        {h_X\times_{h_Z}h_Y}
        \\{h_U}                &
        {h_X}
        \arrow[from=2-2, to=3-2]
        \arrow[from=2-1, to=3-1]
        \arrow[from=2-1, to=2-2]
        \arrow[from=3-1, to=3-2]
        \arrow[from=3-1, to=4-1]
        \arrow[from=3-2, to=4-2]
        \arrow[hook, from=4-1, to=4-2]
        \arrow[hook, from=1-1, to=2-2]
        \arrow[curve={height=30pt}, from=1-1, to=4-1]
    \end{etikzcd}\] We find $h_{S_1}$ defined by
applying the definition of the open embedding $h_U \to h_X$ to the
vertical morphism. Likewise, we can find an $h_{S_2}$
corresponding to $h_V$. Now, set $S=S_1 \intersect S_2$, as
motivated by the first paragraph. We show $h'=h_S$. Let
$R$ be an arbitrary scheme and consider:
\[\begin{etikzcd}
        {h''(R)}
        \\
                 & {h_S(R)}                &
        {h_T(R)} &
        {}
        \\
                 & {h_U\times_{h_W}h_V(R)} &
        {h_X\times_{h_Z}h_Y(R)}
        \arrow[from=2-3, to=3-3]
        \arrow[from=2-2, to=3-2]
        \arrow[from=2-2, to=2-3]
        \arrow[from=3-2, to=3-3]
        \arrow[from=1-1, to=3-2]
        \arrow[from=1-1, to=2-3]
        \arrow[dashed, from=1-1, to=2-2]
    \end{etikzcd}\] Let $x \in h''(R)$. Then we get a pair
$(f,g) \in h_U\times_{h_W}h_V(R)$ and a morphism $j \in h_T(R)$ whose image is
$(f, g)$. We wish to show that $\im j \subseteq S$. Applying
the definition of open embedding, we get two morphisms $f': R \to S_1, g': R \to S_2$
both of which map to $j$ (and hence are equal). We made no
choices during this process, so the morphism $x \to f'$ is unique.
\qed

\section{9.1.H}
Trivial

\section{9.1.I}
We first show that open immersions are injective. Let $f_X(x_1)=f_X(x_2)$
with $x_1, x_2 \in h'(X)$. Then, abusing notation, we write
$x_1, x_2: h_X \to h'$. By yoneda, we have $f\after x_1=f\after x_2: h_X \to h$. Now,
pullback $f\after x_1$ along $f$ to get an open
embedding $h_U \to h_X$. Now, by definition, $(f \after x_1)_X(\id)=f(x_1)$.
Clearly, then $(x_1,\id) \in h_U(X)$ (regarded as the pullback). Likewise,
$(x_2, \id) \in h_U(X)$, and these both go to the same element in
$h_X(X)$. Thus they are equal, so that $x_1 = x_2$.

The identity $h_{U_i}(U_i)$ determines an element
$h(U_i)$. By the zariski condition we get an element
$h(X)$. By the yoneda lemma we get a natural transformation
$f: h_X \to h$. We wish to show that this is bijective.

Let $Y$ be an arbitrary scheme with a pullback cover
$V_i$. Let $x \in h(Y)$. Then we abuse notation to
write $x: h_Y \to h$. Now, $\id \in h_Y(Y)$ is determined by
the inclusions $V_i \to Y$. These pullback to the identity on
$h_{V_i}(V_i)$, and determine a morphism $V_i \to U_i$.
These morphisms glue to give a morphism $Y \to X$ which
corresponds to $x$.

Let $f_Y(x_1)=f_Y(x_2)$. A simple diagram chase shows that the restrictions
of $x_1, x_2$ to the open cover are equal, hence they are equal.
\qed

\section{9.2.A}
Trivial

\section{9.2.B}
We show that $A \otimes I \cong I^{\text{e}}$ as $A$-modules. The
morphism is given by $a \otimes i \mapsto a\phi(i)$. We note that
$a \otimes i = ai \otimes 1$. This map is obviously bijective. It is clear that the
tensored morphism $I^{\text{e}} \to A$ is injective. \qed

\section{9.2.C}
\begin{enumerate}[a.]
    \item We consider an affine open set $\Spec A \subseteq X$. By affineness, we have
          the the following commutative diagram: \[\begin{etikzcd}
                  A                             & {A/J}
                  \\{A/I} & {A/I \otimes_A A/J}
                  \arrow[from=1-1, to=2-1] \arrow[from=2-1, to=2-2] \arrow[from=1-1, to=1-2]
                  \arrow["\phi", from=1-2, to=2-2]
              \end{etikzcd}\] By 9.2.B we see
          that the tensor product is isomorphic to $(A/J)/(\phi(I))$. It is clear by
          examinination that the kernel of this map is $I+J$, so the
          tensor product is $A/(I+J)$, i.e. the closed embedding
          corresponding to the intersection of closed subschemes. \qed
    \item Consider the following diagram: \[\begin{etikzcd}
                  {Z_1\times_XZ_2}                       & {Z_2\times_VW} &
                  {Z_2}
                  \\{Z_1\times_UW} & W              & V
                  \\{Z_1}          & U              & X
                  \arrow["{\text{open}}", hook, from=2-3, to=3-3]
                  \arrow["{\text{closed}}", hook, from=1-3, to=2-3]
                  \arrow["{\text{open}}"', hook, from=3-2, to=3-3]
                  \arrow["{\text{closed}}"', hook, from=3-1, to=3-2]
                  \arrow[hook, from=2-2, to=3-2]
                  \arrow[hook, from=2-2, to=2-3]
                  \arrow[hook, from=1-2, to=1-3]
                  \arrow[hook, from=1-2, to=2-2]
                  \arrow[hook, from=2-1, to=3-1]
                  \arrow[hook, from=2-1, to=2-2]
                  \arrow[hook, from=1-1, to=2-1]
                  \arrow[hook, from=1-1, to=1-2]
                  \arrow[dashed, hook, from=1-1, to=2-2]
              \end{etikzcd}\] where the upper left is
          the pullback because it is the composition of two pullback squares in two
          different ways. The top left box is the pullback of two closed embeddings, the
          bottom right box is the pullback of two open embeddings. Hence the overall
          morphism is a locally closed embedding. \qed
    \item Apply induction to (b). \qed
\end{enumerate}

\section{9.2.D}
We first note that the pullback of a quasicompact morphism along a quasicompact
morphism is quasicompact (9.4.B(a)). This shows the latter property. Now, we
may assume that $A$ is affine, say $\Spec A$.
Let $\Spec B \subseteq X, \Spec C \subseteq Y$. It suffices to show that $A \to B \otimes_A C$
is finite type. We first note that for any $A$-algebras
$B/I \otimes_A C \cong (B \otimes_A C)/J$ for some ideal $J$. Indeed, simply
tensor $I \to B \to B/I$ with $C$ and take kernels.
Now $A[X]/I \otimes_A A[Y]/J
    \cong (A[X] \otimes_A A[Y])/K \cong A[X, Y]/J$. \qed

\section{9.2.F}
Define $b/s \to \phi(b)/\phi(s)$. Suppose we have $f: A \to C$ and
$g: S^{-1}B \to C$ making the relevant diagram commute. Then we send
$a/\phi(s) \to f(a)*g(1/s)$. This is clearly unique. \qed

\section{9.2.G}
We first characterize monomorphisms. $f$ is mono iff the
following diagram is a pullback: \[\begin{etikzcd}
        W                                                                     \\
         & X & X                                                              \\
         & X & Y \arrow["f", from=2-3, to=3-3] \arrow["f"', from=3-2, to=3-3]
        \arrow["{\text{id}}"', from=2-2, to=3-2]
        \arrow["{\text{id}}", from=2-2, to=2-3] \arrow[curve={height=6pt}, from=1-1, to=3-2]
        \arrow[curve={height=-6pt}, from=1-1, to=2-3] \arrow[dashed, from=1-1, to=2-2]
    \end{etikzcd}\]

Now, it is clear that if $f: U \to X$ is pulled back along
$f$ then the result is $U$, and we are
done.

It is clear from 9.1.C(a) that the pullback of a closed subscheme along itself
is itself again, so we are done.

Now, let $f: \Spec S^{-1}A \to \Spec A$. We pull this back along the identity, getting
$\Spec S^{-1}A \otimes_A A$. But by 9.2.F this is just $\Spec S^{-1}A$, and
we are done. \qed

\section{9.2.H}
$\A^n_A \cong \Spec A[x_1, \dots, x_n] \cong \Spec A
    \otimes_{\Z} \Z[x_1, \dots, x_n]$. Now, cover $\P^n_A$ with affine opens,
so by the first part, the result holds. \qed

\section{9.2.I}
We note that equality of morphisms can be checked on an open cover. We then
cover $X$ with affine opens such that we have
$\Spec A \otimes_k \ell \to \Spec A$ as an open cover of $X \times_k \Spec \ell$. We do the
same with $Y$, reducing to the case
$Y=\Spec B$. Consider the following diagram: \[\begin{etikzcd}
        B                & {B\otimes_k\ell}           & B \\
        A                &
        {A\otimes_k\ell} & A \arrow[from=1-2, to=2-2]
        \arrow[hook, from=1-1, to=1-2] \arrow["\phi", hook, from=2-1, to=2-2] \arrow["\pi"', from=1-1, to=2-1]
        \arrow["\rho", from=1-3, to=2-3] \arrow[hook, from=2-3, to=2-2] \arrow[hook, from=1-3, to=1-2]
    \end{etikzcd}\]
The horizontal morphisms are injections. Let $\pi(x)=y$. Then
clearly $\phi(x\otimes 1)=y\otimes 1$ so that $\rho(x)=y$. \qed

\section{9.2.J}
We repeat the above reduction to affine opens. Then  \[\begin{etikzcd}
        B & {B\otimes_k\ell} \\
        A &
        {A\otimes_k\ell} \arrow["\phi", two heads, from=1-2, to=2-2]
        \arrow[hook, from=1-1, to=1-2] \arrow[hook, from=2-1, to=2-2] \arrow["\pi"', from=1-1, to=2-1]
    \end{etikzcd}\] We
have surjective $\pi$ by assumption. Let
$y \in A$. Then there exists an $x$ such that
$\pi(x\otimes 1)=y\otimes 1$. Then by injectivity we have $\phi(x)=y$
and we are done. \qed

\section{9.3.A}
Consider the following diagram: \[\begin{etikzcd}
        Z                                              \\
         & F & X                                       \\
         & x & Y \arrow["{\phi_x}"', from=2-2, to=3-2]
        \arrow["\iota", from=2-2, to=2-3] \arrow["\phi", from=2-3, to=3-3] \arrow["\iota"', from=3-2, to=3-3]
        \arrow["{\pi_1}"', curve={height=12pt}, from=1-1, to=3-2]
        \arrow["{\pi_2}", curve={height=-12pt}, from=1-1, to=2-3]
        \arrow["f"{description}, dashed, from=1-1, to=2-2]
    \end{etikzcd}\] where
$F=\phi^{-1}(x)$ and $\phi_x=\phi|_F$. We define the universal
map by $f(z)=\pi_2(z)$. It is obvious that this diagram commutes in
$\catname{Set}$, so we merely need to show that $f$
is continuous. Let $U \intersect F \subseteq F \subseteq X$ be open in the subspace topology.
Then by continuity of $\pi_2$, $\pi_2^{-1}(U)$ is open
in $Y$. Thus, $f^{-1}(U \intersect F)=\pi_2^{-1}(U)
    \intersect \pi_2^{-1}(F)=\pi_2^{-1}(U)$ is an open subset of
$Z$. \qed

\section{9.3.B}
We reduce immediately to the case of $f: \Spec B \to \Spec A$ with corresponding
ring morphism $\phi: A \to B$. Let $\p \subseteq B$ be prime.
Then clearly $\V(\p) \to \Spec A$ pulls back to the closed immersion
$\V(\p^{\text{e}}) \to \Spec B$. By 3.2.I(a) this is identified with the topological
fiber of that pullback. Thus, we may reduce to the case where
$A$ is an integral domain and $\p = (0)$. Then
the pullback corresponds to the ring $\phi(A \setminus 0)^{-1}B$. Let
$f(\q)=(0)$ so that $\phi^{-1}(\q)=(0)$. Since
$(0)$ is principal, we see that $\q \intersect \im \phi = 0$. Thus
$\q$ avoids $\phi(A \setminus 0)^{-1}$, so that it is a prime
ideal of $\phi(A \setminus 0)^{-1}B$. Finally, suppose $\q \subseteq \phi(A \setminus 0)^{-1}B$ is
prime. Then $\phi(A \setminus 0) \nsubseteq \q$ so that $\phi^{-1}(\q)=(0)$. \qed

\section{9.3.C}
It suffices to show that $\tau \after \iota_p$ (where $\iota_p$
is the inclusion) factors through $\iota_{\tau(p)}$. Indeed, choose an
affine open $\Spec A \subseteq Z$ and an affine open $\Spec B \subseteq \tau^{-1}(\Spec A)$.
Then we have $f \after i: A \to B \to \kappa(p)$. By 3.2.P(a), we may factor this as
$A \to A/\tau(p) \to \kappa(p)$. By universal property of localization, we may factor
this as $A \to A/\tau(p) \to \kappa(\tau(p)) \to
    \kappa(p)$. \qed

\section{9.3.D}
Let $(p) \subset \Z$ be prime. We are interested in
$\Z_{(p)}/(p)_{(p)}
    \otimes_{\Z} \Z[i]$. If $p=0$ then we have
$\Q \otimes_{\Z} \Z[i] \cong
    \Q[i]$, a degree two field extension of
$\Q$. If $p=2$ then we have
$\Z_2[i] \cong \Z_2[x]/(x^2+1) \cong
    \Z_2[x]/(x+1)^2$, a double point.

Next, we consider $\Z_p[i]$ for odd primes. We first prove
Euler's criterion, $(\frac{a}{p})=a^{((p-1)/2}$. Indeed, since
$\Z_p$ is a field, $x^2=a$ has at most two
solutions over it. There are $p-1$ units of
$\Z_p$. Send $x \mapsto x^2$. The range of this map
are the residues, and at most two values can map to the same value. Thus the
range has size lower bounded by $p-1/2$. Now, by Fermat's
little theorem, $a^(p-1)=1 \mod p$ for all $a$. This
factors as $(a^{p-1/2}-1)(a^{p-1/2}+1)$. Quadratic residues clearly fall into the
former case. Now $(a^{p-1/2}-1)$ has at most $p-1/2$
roots. But then these must be precisely the quadratic residues. Hence the other
factor must be the quadratic non-residues, as desired.

Now, consider $(-1)^{p-1/2} \mod p$. If we want this to be 1 then
$p-1/2$ must be even, say $2k$. Then
$p=4k+1$ so that $p = 1 \mod 4$. In this case, we have
a number $a$ such that $a^2=-1$. Thus
$\Z_p[x]/(x^2+1) \cong \Z_p[x]/(x+a) \times
    \Z_p[x]/(x-a)$. This is another double point. In the case
$p=3 \mod 4$ we find that $x^2+1$ is irreducible,
hence maximal, hence $\Z_p[i] \cong \F_{p^2}$. This is another degree two field
extension of the field of fractions.

\section{9.3.E}
$k[y] \cong k[x, y]/(y^2-x)$. Thus
\begin{align*}
    k[y] \otimes_{k[x]} k[z] & \cong \frac{k[x,y]}{(y^2-x)} \otimes_{k[x]} \frac{k[x,z]}{(z^2-x)} \\
                             & \cong \frac{k[x,y,z]}{(y^2-x, z^2-x)}                              \\
                             & \cong \frac{k[x,y,z]}{(z^2-y^2)}                                   \\
                             & \cong \frac{k[x,y,z]}{((z-y)(z+y))}
\end{align*}
In characteristic 2 we have $-1=1$ so that we have an
irreducible but non-reduced scheme. \qed

\section{9.3.F}
Suppose we are given a point $[a : b] \in \mathbb{P}^1_k$. Then we have
$xb=ya$, which clearly shows that $(x, y)$ must
lie on a line with slope $a/b$ or $b/a$.
Thus the fibers are the lines with slope determined by the projective point.

We examine the fiber of $(a, b) \in \A^2_k$. Clearly this is the pair
$(a,b), [a : b]$. WLOG assume that $a \neq 0$. Then we may
examine $\D_{u, x} \to \D_x$. This is given by the ring
$\Spec k[x, y, v]_x/(xv-y)$. The projection gives a ring morphism
$k[x, y]_x \to k[x, y, v]_x/(xv-y)$. This is an isomorphism sending $y/x$
to $v$.

On the open set $\D_u$ we have find that the blowup is the
affine scheme $\Spec k[x, y, v]/(xv-y)$. Consider $\V(x)$. This
gives $\Spec k[x, y, v]/(xv-y,x) \cong \Spec
    k[x, y, v]/(x, y) \cong \Spec
    k[v] \cong \Spec k \otimes_{k[x, y]}
    k[x,y,v]/(xv-y)$, i.e. the fiber over $(0, 0)$. On
the open set $\D_v$, we find this subscheme is cut out by
$y$.

\section{9.3.G}
We first reduce to the case of Noetherian $Y$. Indeed, if
the result holds for a Noetherian open subset of $Y$, then
it holds for $Y$. So we assume $Y$ is
Noetherian.

Suppose the result is true for irreducible $Y$. Then for
reducible $Y$ we get a locally closed subset
$U_i \intersect Z_i$ where the result holds for each irreducible component
$Z_i$. Since $Y$ is Noetherian, the index
set is finite. Now note that $\Union_j (\Intersect_i U_i) \intersect Z_j = \Intersect_i U_i$ is open in
$Y$, and so the result holds for $\Intersect_i U_i$.
Thus we may assume $Y$ is irreducible.

Now, following the hint, we show that $A$ is integral over
$B$. We have that $A \otimes_B \operatorname{K}(B)$ is integral over
$\operatorname{K}(B)$. Writing $A=B[X]/I$ we find an integral
relation for $x_i \otimes 1$, for example $f_i(x_i)=x_i^n \otimes 1+\sum_{j=0}^{n-1} x_i^j \otimes b_j/b_j'=0$.
Replacing $B$ with the localization at all the denominators
in these relations (and also replacing $A$ with the
relevant inverse image), we find
\begin{align*}
    f_i(x_i) & =x_i^n \otimes 1+\sum_{j=0}^{n-1} x_i^j \otimes b_j  \\
             & =x_i^n \otimes 1+\sum_{j=0}^{n-1} b_jx_i^j \otimes 1 \\
             & = (x_i^n+\sum_{j=0}^{n-1} b_jx_i^j) \otimes 1        \\
             & =0
\end{align*}

We find that either $x_i^n+\sum_{j=0}^{n-1} b_jx_i^j=0$, in which case we are done, or it
is annihilated by some $b_i''$ by replacing 1 with
$b_i''/b_i''$. Replacing $B$ with the
localization at these $b_i''$, we recall that an element of the
localization is zero iff the product with a denominator is zero. Hence,
$f_i(x_i)=0$ in $A$. \qed

\section{9.3.H}
Note that the following diagram is a pushout: \[\begin{etikzcd}
        {\Z[x_1, \dots, x_N]} & B \\{\frac{\Z[x_1,\dots, x_N,y_1, \dots, y_M]}{I}}
                              & {\frac{B[y_1, \dots, y_M]}{I}}
        \arrow[from=1-2, to=2-2]
        \arrow[from=1-1, to=1-2]
        \arrow[from=2-1, to=2-2]
        \arrow[from=1-1, to=2-1]
    \end{etikzcd}\] where the
top map is described by the hint. This shows the result for affine
$X$.

Lemma: Suppose $\phi \after \pi: B \to B[X]/I \to
    B[Y]/J$ is a composition of morphism of rings,
where $X, Y$ are finite sets of variables, and
$I, J$ are finitely generated ideals. Then we may find a ring
$B[X']/I' \cong B[Y]/I$ such that $X \subset X'$ and
$I \subset I'$. Indeed, let $x \in X$. Then we may form
$B[Y, x]/(J+(x-\phi(x)))$. Repeat this for all $x \in X$ and call
this ring $B[X, Y]/I'$. I claim this ring has the desired property.
Let $\iota: B[Y]/J \to B[X, Y]/I'$ be the inclusion. Let $x \in X \subset B[X, Y]/I'$. Then
$\iota(\phi(x))=x$, so that $\iota$ is surjective. Let
$a \in \ker \iota$. Then either $a \in J$ or
$a \in (x-\phi(x))$ for some $x \in X$. But since
$x \notin R[Y]/J$, we must have $a = 0$. Now, the fact
that $I \subset I'$ follows by examining the map
$B[X] \to B[X, Y] \to
    B[X, Y]/I'$.

Now, suppose $X$ is a scheme as in the original problem,
and it is covered by finitely many affine opens, all with affine intersections.
Let $\Spec B[y_1, \dots, y_M]/I$ be such an affine open, and let
$\Spec B[y'_1, \dots, y'_{M'}]/J$ be the intersection with an arbitrary other affine
open. Then by the above lemma we have the following diagram:
\[\begin{etikzcd}
        {\Z[x_1, \dots, x_N]}                                                  & B
        \\{\frac{\Z[x_1,\dots, x_N,y_1, \dots, y_M]}{I}} &
        {\frac{B[y_1, \dots, y_M]}{I}}                                                          \\
        {\frac{\Z[x'_1, \dots x'_{N'}, y'_1, \dots, y'_{M'}]}{I'}}             & {\frac{B[y'_1,
        \dots, y'_{M'}]}{I'}}
        \arrow[from=1-2, to=2-2]
        \arrow[from=1-1, to=1-2]
        \arrow[from=2-1, to=2-2]
        \arrow[from=1-1, to=2-1]
        \arrow[from=2-2, to=3-2]
        \arrow[from=2-1, to=3-1]
        \arrow[from=3-1, to=3-2]
    \end{etikzcd}\] where the upper right map is the map of global sections
and the lower right map is the restriction. The maps on the left are defined by
using the properties of $I$ guaranteed by the lemma.
Applying $\Spec$, we see that the upper square is a pullback,
and the whole rectangle is a pullback, hence the bottom square is a pullback.
Hence, since the lower right map is an open embedding, so is the lower left
map.

TODO: Check triple intersections, show the result when intersections aren't
finite.

\section{9.3.I}
\begin{enumerate}[a.]
    \item We recall that locally closed embeddings are preserved by base change. Now, the
          given morphism $\pi$ is the pullback of a morphism
          $\pi'$ of Noetherian schemes by 9.3.H. Thus the image of
          $\pi'$ is a finite union of locally closed subsets, which may
          be realized by finitely many locally closed embeddings. Now examine the
          following diagram: \[\begin{etikzcd}
                  {H_i'}                         &   & {H_i}
                  \\
                                                 & X &       & {X'}
                  \\{Z'_i} &   & {Z_i}        \\
                                                 & Y &       & {Y'}
                  \arrow[from=2-2, to=4-2]
                  \arrow[from=2-2, to=2-4]
                  \arrow[from=2-4, to=4-4]
                  \arrow[from=4-2, to=4-4]
                  \arrow[hook, from=3-3, to=4-4]
                  \arrow[hook, from=1-3, to=2-4]
                  \arrow[from=1-3, to=3-3]
                  \arrow[hook, from=1-1, to=2-2]
                  \arrow[hook, from=3-1, to=4-2]
                  \arrow[from=3-1, to=3-3]
                  \arrow[from=1-1, to=3-1]
                  \arrow[from=1-1, to=1-3]
              \end{etikzcd}\] Where $\{Z_i\}_{i\in I} \to Y'$ is the
          covering by locally closed embeddings. Clearly the $H_i$ cover
          $X'$, so the $H'_i$ must cover
          $X$. Thus the $Z'_i$ cover the image. \qed
    \item We first note that being a locally closed embedding is stable under base
          change. Cover $Y$ with finitely many affines
          $\{\Spec B_i\}_{i \in I}$ and $X$ with finitely many affines
          $\{\Spec A_{ij}\}_{i \in I,j \in J}$ such that $B_i \to A_{ij}$ realizes
          $A_{ij}$ as finitely presented. By (a), the image of this is a
          finite union of locally closed subsets. Taking the union across all
          $i, j$ (which is finite), we get our result. \qed
\end{enumerate}

\section{9.4.A}
It is clear that being a locally principally closed subscheme is affine-local
on the target. Additionally, it has the form of part (b) of the lemma, where Q
is the property "$A \to B_i$ is surjective and realizes
$B_i$ as $A/(s)$ for some
$s \in A$". Thus the property may be checked soley for affines.
The rest of the proof follows the same form as 9.4.B(g). \qed

\section{9.4.B}
\subsection{Lemma}
\begin{enumerate}[a.]
    \item Suppose P is a property of morphisms that is affine local on the target. Then
          we may reduce to the case of affine times arbitrary over affine.
    \item Let $f: X \to Z$ be a morphism of schemes. Suppose P is an affine
          local (on the target) property of $f$ of the form "For every
          affine $\Spec A \subset Z$ there exists an affine open cover
          $\{\Spec B_i\}_{i \in I}$ of $f^{-1}(\Spec A)$ such that the morphisms
          $A \to B_i$ all have propery Q". Then we may reduce to the case of
          affine times affine over affine.
\end{enumerate}

Proof:
\begin{enumerate}[a.]
    \item Consider the following diagram: \[\begin{etikzcd}
                  {X\times_ZY} & X                                                   \\
                  Y            & Z \arrow[from=1-1, to=2-1] \arrow[from=1-1, to=1-2]
                  \arrow["g"', from=2-1, to=2-2] \arrow["f", from=1-2, to=2-2]
              \end{etikzcd}\] Cover
          $Z$ with affines $\{\Spec A_i\}_{i \in I}$. Then we may cover
          $Y$ with affines $\{\Spec B_{ij}\}_{i \in I, j \in J}$ such that
          $\Spec B_{ij} \subseteq
              g^{-1}(\Spec A_i)$. By affine locality, we may check P on the cover
          $\{\Spec B_{ij}\}_{i \in I, j \in J}$, so in particular if P holds for $Y, Z$
          affine then P holds for $Y, Z$ arbitrary.
    \item We first apply part (a) of the lemma. Let $Y=\Spec C$ and
          $Z=\Spec A$. Now, choose an affine cover $\{\Spec B_i\}_{i \in I}$ of
          $X$ such that $A \to B_i$ has property Q. Then we
          may pullback by the open immersion $\Spec B_i$ to get an affine open
          cover of $X \times_Z Y$. If Q holds for the morphism
          $C \to C \otimes_A B_i$ then we have an affine open cover of
          $X \times_ZY$ satisfying P, as desired.
\end{enumerate} \qed

We now begin the actual proof:
\begin{enumerate}[a.]
    \item Apply part (a) of the lemma by 7.3.C. We use notation as in the lemma.
          $X$ may be covered by finitely many open affines by
          quasicompactness of $X \to \Spec A$. Call these $\{\Spec B_i\}_{i \in [1, n]}$.
          Then the finitely many $\Spec C \times_{\Spec A} \Spec B_i$ are affine and cover
          $X \times_ZY$.

    \item Apply part (a) of the lemma by 7.3.C. Given the intersection of two open
          affines of $X$, we form a finite affine open cover. The
          pullback of this finite affine open cover covers the pullback of the
          intersection.
    \item Apply part (a) of the lemma by 7.3.4. Then we may take $X$
          to be affine since $f$ is affine. Then we are done as
          morphisms of affine schemes are affine.
    \item Apply (e) and (g).
    \item Apply part (b) of the lemma, using the fact that integrality is affine-local on
          the target: 7.3.10. Then apply 7.3.N, base change of integral morphism of
          affine schemes is affine.
    \item Apply part (b) of the lemma by 7.3.O. Then the result follows by 9.2.B combined
          with 9.2.A (tensor product preserves finitely generated algebra-ness).
    \item Apply (a) and (f).
    \item The same proof as (f) applies here without change, simply noting that the
          extension of a finitely generated ideal is finitely generated.
    \item Apply (a), (b), and (h).
\end{enumerate} \qed

\section{9.4.C}
Quasifiniteness obviously satisfies the conditions of part (b) of the lemma
above. Thus we may reduce to the pullback of $f: \Spec A \to \Spec B$ along
$g: \Spec C \to \Spec B$. Now, consider $\Spec \kappa(\p) \to \Spec C \to \Spec B$. This factors
through the morphism $\Spec \kappa(\p) \to \Spec \kappa(g(\p)) \to \Spec B$. Clearly $f^{-1}(g(\p)) \to \Spec \kappa(g(\p))$
is finite type (since it is the pullback of $f$). Thus we
reduce to the case of the pullback of field by affine over a field. This case
is covered by 7.4.D, making the morphism finite. Then it pulls back to a finite
morphism over the field, hence it has finite fibers. \qed

\section{9.4.D}
We consider $X\times_ZY$ where $f: Y \to Z$ is surjective
and $g: X \to Z$ is arbitrary. Let $\p \in X$. Then by
9.3.C we may reduce to $\Spec \kappa(\p) \to \Spec \kappa(g(\p))$. Now, choosing a point
$\q \in f^{-1}(g(\p))$ we are reduced to examining $\kappa(\p) \otimes_{\kappa(g(\p))}\kappa(\q)$.
Now, the tensor product of two non-zero vector spaces is non-zero, since there
is only one element of the ground field which annihilates both. Thus, the above
tensor product, being of two $\kappa(g(\p))$-vector spaces must be
non-zero. It thus has a maximal ideal, hence there is at least one point in the
fiber over $\p$.

\section{9.4.E}
We let $k=\overline{k}$. By 9.2.D $X\times_kY$ is finite
type. By 9.5.L, we may prove irreduciblity on an affine open cover. Also, by
5.2.A (and the discussion in 5.3.2), reducedness is an affine-local property.
Thus, if we show that $\Spec A \otimes_k \Spec B$ is integral for (finitely
generated over $k$) integral domains
$A, B$, then we are done. Now, $B$ is a
free $k$-module, so we may choose generators
$\{b_i\}_{i \in I}$ for some index set $I$. Now,
suppose $(\sum_i a_i \otimes b_i)(\sum_j a'_j \otimes b'_j) = 0$ where $a_i, a'_j \neq 0$ and the
$b_i, b'_j$ are generators of $B$. Now,
$a_1a'_1 \neq 0$, so $\D(a_1a'_1)$ is nonempty and thus has a
maximal ideal $\m$. Consider $A/\m \otimes_k B$. By the
weak nullstellensatz, we have $A/\m \cong k$, so this is just
$B$. Now $(\sum_i \overline{a_i} \otimes b_i)(\sum_j \overline{a'_j}
    \otimes b'_j)=0 \in B$. But also
$a_1 \neq 0$ in $A/\m$ so that
$\overline{a_1}\otimes b_1 \neq 0$ and $\sum_i \overline{a_i} \otimes b_i \neq 0$ (and likewise with
$a'_1$). But this contradicts the fact that
$B$ is an integral domain. Thus $a_i = 0$ and
$a'_j=0$ for all $i, j$. \qed

\section{9.4.F}
Consider the following diagram: \[\begin{etikzcd}
        {X\times_S X'}                             & {X'\times_SY} &
        {X'} \\{X\times_SY'} &
        {Y\times_SY'}                              &
        {Y'}                                                                                    \\
        X                                          & Y             & S \arrow[from=1-3, to=2-3]
        \arrow[from=2-3, to=3-3] \arrow[from=3-2, to=3-3] \arrow[from=3-1, to=3-2]
        \arrow[from=2-2, to=3-2] \arrow[from=2-2, to=2-3] \arrow["g"', from=2-1, to=2-2]
        \arrow[from=2-1, to=3-1] \arrow[from=1-2, to=1-3] \arrow[from=1-2, to=2-2]
        \arrow[from=1-1, to=1-2] \arrow["f", from=1-1, to=2-1] \arrow[dashed, from=1-1, to=2-2]
        \arrow[curve={height=12pt}, from=3-1, to=3-3] \arrow[curve={height=-12pt}, from=1-3, to=3-3]
    \end{etikzcd}\] All the squares in this
diagram are pullbacks, hence by hypothesis $f$ and
$g$ in particular have property $P$.
Hence their composition, the (uniquely defined) dotted morphism has
$P$. \qed

\section{9.5.A}

\section{9.5.B}
Compose the relevant pullback squares. \qed

\section{9.5.C}
\section{9.5.D}
\begin{enumerate}[a.]
    \item Let $K$ be a non-trivial purely inseparable extension of an
          imperfect field $k$. Then $\Spec K \to \Spec k$ is reduced,
          but $\Spec K \otimes_{\Spec k} \Spec K$ isn't. Indeed, let $x \in K \setminus k$ and let
          $x^q \in k$. Then $(x \otimes 1 - 1 \otimes x)^q=x^q \otimes 1 - 1 \otimes x^q=x^q\otimes 1 - x^q
              \otimes 1 = 0$. \qed
    \item The example of 9.5.C is disconnected since it is the disjoin union of two
          points. \qed
    \item The example 9.5.C suffices for this also. \qed
\end{enumerate}

\section{9.5.E}

\section{9.5.F}
\begin{enumerate}[a.]
    \item We note that free modules are flat, hence tensor product with
          $E$ is exact, hence preserves injectivity. \qed
    \item We recall that reducedness is affine-local, thus we may reduce to affine
          $X$. We then note that a subring of a reduced ring is
          reduced then apply (a).
    \item By 9.5.L we may reduce to the affine case. Note that a subring of a irreducible
          ring is irreducible then apply (a). \qed
    \item The inverse image of a disconnected space under a surjective continuous map is
          disconnected. Indeed, let $f: X \to Y$ be surjective and continuous
          and let $Y=U \union V$ with $U \intersect V = \emptyset$. By surjectivity
          $f^{-1}(Y)=X$ and $f^{-1}(U) \intersect
              f^{-1}(V)=f^{-1}(U \intersect V) = \emptyset$. To see the result, recall
          that (9.4.D) the map $X_E \to X$ is surjective. \qed
\end{enumerate}

\section{9.5.4}
We first reduce to the case of reduced affine schemes over
$k$. Then we reduce to showing $\pi: A^n_k \to A^m_k$ is
open (for $m < n$). By Chevalley's theorem the image of any
open is constructible. By 7.4.C(b) the image of an open $U$
is open iff it is closed under generization. Let $p \in U$,
$S=\pi(U)$, $\pi(p)=q$. We show
$\Spec k[x_1, \dots, x_n]_p \to \Spec
    k[x_1, \dots, x_m]_q$ is surjective. It suffices to show
\[\begin{etikzcd}
        {k[x_1, \dots, x_m]_q}                        & {\kappa(r)}
        \\{k[x_1, \dots,x_n]_p} &
        {k[x_1, \dots,x_n]_p \otimes \kappa(r)}
        \arrow["f", from=1-1, to=1-2]
        \arrow["g"', from=1-1, to=2-1]
        \arrow[from=1-2, to=2-2]
        \arrow[from=2-1, to=2-2]
    \end{etikzcd}\] is non-zero. Consider $A \otimes_C B$ Indeed,
the pure tensor $a \otimes b = 0$ iff $a=0$ or
$b=0$ or $a=cx$ where
$c \in C$ and $ca=0$. Consider
$1 \otimes 1 \in k[x_1, \dots,x_n]_p \otimes
    \kappa(r)$. Suppose $a \in x_1, \dots, x_m]_q$ be such that
$f(a)=0$. Then $a \in r$ so that
$g(a) \in p$. Hence there does not exist an $x$
such that $g(a)x \neq 1 \in k[x_1, \dots,x_n]_p$. Next we show that $g$ is
injective. Indeed, the map $k[x_1, \dots, x_m] \to k[x_1, \dots, x_n]$ is injective and
localization is exact, so $k[x_1, \dots, x_m]_q \to S^{-1}k[x_1,
    \dots, x_n]$ is injective. The latter has
no zero divisors, so $S^{-1}k[x_1, \dots, x_n] \to
    k[x_1, \dots, x_n]_p$ is also injective. Thus, if
$g(a)=0$ then $a=0$ so that
$f(a)=0$. \qed

\section{Note on 9.5.6}
The book doesn't show the following critical result: Let
$S, A$ be $k$-algebras and let
$S' \subseteq S$ be a sub-$k$-algebra. Let
$f \in S'$. Let $\p \in \Spec A$ be in the image of the map
$(\Spec S' \otimes_k A)_f \to \Spec A$. Consider the following diagram: \[\begin{etikzcd}
        {\operatorname{Spec} (S' \otimes_k A)_f\otimes_A\kappa(\mathfrak{p})}                       & {\operatorname{Spec} (S' \otimes_k A)_f} &
        {\operatorname{Spec} (S \otimes_kA)_f}                                                      &
        {}
        \\{\operatorname{Spec} (S' \otimes_k A)\otimes_A\kappa(\mathfrak{p})} & {\operatorname{Spec} S'\otimes_kA}       &
        {\operatorname{Spec} S\otimes_kA}
        \\{\operatorname{Spec} \kappa(\mathfrak{p})}                          &                                          &
        {\operatorname{Spec} A}
        \arrow[from=2-2, to=3-3]
        \arrow[from=2-3, to=3-3]
        \arrow[from=2-3, to=2-2]
        \arrow[from=1-2, to=2-2]
        \arrow[from=1-3, to=2-3]
        \arrow[from=1-3, to=1-2]
        \arrow[from=3-1, to=3-3]
        \arrow[from=1-1, to=1-2]
        \arrow[from=1-1, to=2-1]
        \arrow[from=2-1, to=2-2]
        \arrow[from=2-1, to=3-1]
    \end{etikzcd}\]
Then the image of the map $(\Spec S' \otimes_k A)_f \to \Spec A$ is the same as the image of
$(\Spec S \otimes_k A)_f \to \Spec A$. This theorem is Stacks 037F. Note first that
$S' \otimes_k \kappa(\p) \to S \otimes_k \kappa(\p)$ is injective since the component maps are injective and
$k$ is a field. Now,
\begin{align*}
    (S' \otimes_k A)_f \otimes_A \kappa(\p) & \cong ((S' \otimes_k A) \otimes_A \kappa(\p))_f \\
                                            & \cong (S' \otimes_k (A \otimes_A \kappa(\p)))_f \\
                                            & \cong (S' \otimes_k \kappa(\p))_f
\end{align*}
and similarly for $S$. Now, since $\p$
is in the image of $(\Spec S' \otimes_k A)_f \to \Spec A$, we find that
$(S' \otimes_k \kappa(\p))_f$ is non-trivial, hence $S' \otimes_k \kappa(\p)$ is
nontrival and $f$ is non-nilpotent. Thus
$S \otimes_k \kappa(\p)$ is nontrival and $f$ is
non-nilpotent, so that it has a point. \qed

\section{Note on 9.5.7}
The following result is key: Let $k \to K$ be a purely
inseparable extension of fields of characteristic $p$.
Consider the following diagram: \[\begin{etikzcd}
        k & K             \\
        A & {A\otimes_kK}
        \arrow[from=1-2, to=2-2] \arrow[from=2-1, to=2-2] \arrow[from=1-1, to=2-1]
        \arrow[from=1-1, to=1-2]
    \end{etikzcd}\] The prime ideals of
$A\otimes_kK$ inject into $A$. Indeed, let
$\p, \q \subset A\otimes_kK$ and let them inverse image into the prime ideal
$\mathfrak{r}$. Let $f \in \p$. Then we see that there
exists an $n$ such that $f^{p^n} \in A$ (take
$n$ to be the largest power of a coefficient of
$f$ such that the coefficients go to
$k$). Now, $f^{p^n} \in \mathfrak{r}$ which means that
$f^{p^n} \in \q$. But $\q$ is prime so
$f \in \q$.

\section{9.5.H}
Let $X$ be a topological space and let
$p \in X$. Consider the set of connected subsets containing
$p$. Let $\{S_i\}_{i \in I}$ be a chain in this poset
and consider $S=\Union_i S_i$. Let $S=U \union V$. Then they
intersect on some element of the chain, hence they intersect. Hence
$S$ is connected. Hence by Zorn's lemma,
$p$ is contained in a maximal connected set.

Next we show that the closure of a connected subspace is connected. Indeed, let
$S^{\text{C}} \intersect (U \union V) =
    S^{\text{C}}$ where $S$ is an arbitrary subset
and $S^{\text{C}} \intersect U \intersect V = \emptyset$. Then these can't intersect in
$S$ either. Since $S$ is connected, we
must have $S \subseteq U \intersect S^{\text{C}}$ or $S \subseteq V$. WLOG assume it is
$U$. Then $V \subseteq \overline{S^{\text{C}}}$ by definition of closure,
hence $U=S^{\text{C}}$.

\section{9.5.I}
Let $S \subseteq Y$ be a connected component. Then
$S'=\phi^{-1}(S)$ is connected. Indeed, suppose otherwise. Then we have a
disconnection $S'=U \union V$ with $U \intersect V = \emptyset$. Now,
$\phi$ is open, hence $\phi(U), \phi(V)$ are open and
cover $S$. Hence $\phi(U) \intersect \phi(V) \neq \emptyset$. Let
$x \in \phi(U) \intersect \phi(V)$. Then $U$ and
$V$ disconnect $\phi^{-1}(x)$, contradiction.

Now, suppose $S \subseteq X$ a connected component. Then
$\phi(S)$ is connected. Indeed, note that $\phi$
is surjective, hence the inverse image of a disconnected set is disconnected.
Let $x \in \phi(S)$. Then $\phi^{-1}(x) \subseteq S$, since it is
connected. Thus $\phi^{-1}(\phi(S)) = S$, hence $\phi(S)$ is
connected. \qed

\section{9.5.J}
We first note that if $X=U \coprod V$ then $\Osh_X(X)=\Osh_X(U) \times \Osh_X(V)$.
Indeed, any pair of sections over $U$ and
$V$ agree on the overlap and thus glue. Now,
$(1,0)$ will suffice as a nontrivial nilpotent. Conversely, if
$e^2=e$ then $\V(e) \intersect \V(1-e)
    =\V(1)= \emptyset$ and
$\V(e) \union
    \V(1-e)=\V(e^2-e)=\V(0)=X$. $1-e$ is also idempotent (and not
0), hence it isn't a unit, so $\V(1-e)$ is nontrivial. All this
together shows that the $X=\overline{\V(e)} \coprod \overline{\V(1-e)}$. \qed

\section{9.5.L}
\begin{itemize}
    \item[$\implies$] Let $U_1=X$. \qed
    \item[$\impliedby$] Irreducible is equivalent to the statement that every
          pair of non-empty open subsets meets. We are given a cover of non-empty opens,
          which all meet. Let $V, W$ be arbitrary non-empty opens. We
          show they intersect. Indeed, choose $i, j$ such that
          $U_i \intersect V$ and $U_j \intersect W$ are non-empty. Then
          $U_i \intersect U_j \intersect V \intersect W$ is nonempty since $U_i \intersect U_j$ is nonempty.
          \qed
\end{itemize}

\end{document}
\documentclass{article}
\usepackage[utf8]{inputenc}

\usepackage{mathtools}
\usepackage{amsthm}
\usepackage{amssymb}
\usepackage{dsfont}
\usepackage{array}   % for \newcolumntype macro
\usepackage{calligra}
\usepackage{tikz-cd}
\usepackage{mathrsfs}
\usepackage{quiver}
\usepackage[shortlabels]{enumitem}

\newcolumntype{L}{>{$}l<{$}}
% math-mode version of "l" column type

\newcommand{\Rho}{\mathrm{P}}

\newcommand{\iso}{\simeq}
\newcommand{\after}{\circ}
\newcommand{\catname}[1]{\mathbf{#1}}
\newcommand{\intersect}{\cap}
\newcommand{\union}{\cup}
\newcommand{\Intersect}{\bigcap}
\newcommand{\Union}{\bigcup}
\newcommand{\wa}[1]{\langle #1 \rangle}
\newcommand{\bb}[1]{\mathbb{#1}}
\newcommand{\A}{\mathbb{A}}
\newcommand{\Z}{\mathbb{Z}}
\newcommand{\Q}{\mathbb{Q}}
\newcommand{\R}{\mathbb{R}}
\newcommand{\C}{\mathbb{C}}
\newcommand{\F}{\mathbb{F}}
\newcommand{\p}{\mathfrak{p}}
\newcommand{\q}{\mathfrak{q}}
\newcommand{\ai}{\mathfrak{a}}
\newcommand{\bi}{\mathfrak{b}}
\newcommand{\m}{\mathfrak{m}}
\newcommand{\defeq}{\vcentcolon=}

\DeclareMathOperator{\id}{id}
\DeclareMathOperator{\Hom}{\mathscr{H}\text{\kern -3pt
{\calligra\large om}}\,}
\DeclareMathOperator{\Fsh}{\mathscr{F}}
\DeclareMathOperator{\Gsh}{\mathscr{G}}
\DeclareMathOperator{\Hsh}{\mathscr{H}}
\DeclareMathOperator{\Osh}{\mathscr{O}}
% \DeclareMathOperator{\ker}{ker}
\DeclareMathOperator{\coker}{coker}
\DeclareMathOperator{\im}{im}
\DeclareMathOperator{\Spec}{Spec}
\DeclareMathOperator{\Proj}{Proj}
\DeclareMathOperator{\V}{V}
\DeclareMathOperator{\D}{D}

\newcommand*{\rationalto}[1][]{\mathbin{\tikz [baseline=-0.25ex,-latex, dashed,->,densely dashed,#1] \draw [#1] (0pt,0.5ex) -- (1.3em,0.5ex);}}%


\title{Chapter 13 Problems}
\author{Josiah Bills}
\date{January 2022}

\begin{document}

\maketitle

\section{13.1.A}
We will drop the superscript and write $s: M \to V$. We will say that
$s_i=t_i \after s|_{U_i}$ where $t_i$ is the trivialization map
for $U_i$. Let $x \in U_i \intersect U_j$. $T_{ij}$
is defined by the following commutative diagram:  \[\begin{tikzcd}
        {(U_i \cap U_j) \times\R^n} & {\pi^{-1}(U_i \cap U_j)} & {(U_i \cap U_j) \times\R^n} \\
                                    & {U_i \cap U_j}
        \arrow["{t_i}"', from=1-2, to=1-1]
        \arrow["{t_j}", from=1-2, to=1-3]
        \arrow[from=1-3, to=2-2]
        \arrow[from=1-1, to=2-2]
        \arrow[from=1-2, to=2-2]
        \arrow["{T_{ij}}", curve={height=-18pt}, from=1-1, to=1-3]
    \end{tikzcd}\] Now,
$T_{ij}$ acts on functions $U_i \intersect U_j \to (U_i \intersect U_j) \times \R^n$ pointwise, i.e.
$(T_{ij} s_i)(x)=(T_{ij}
    )(s_i(x))$. In particular $(T_{ij} s_i)(x)=s_j(x)$ by the
commutativity of the above diagram. \qed

\section{Notes on 13.1}
In order to define a morphism of locally free sheaves, it suffices to give
morphisms on an open cover, together with gluing data that is compatible with
the morphism (see e.g. 2.5.D). Suppose $\phi: \Fsh \to \Gsh$ is a morphism on
$U$ and $\p\psi: \Fsh' \to \Gsh'$ is a morphism on
$V$. Let $f: \Fsh|_{U \intersect V} \to \Fsh'|_{U \intersect V}$ and $g: \Gsh|_{U \intersect V} \to \Gsh'|_{U \intersect V}$
be the gluing data. Then compatiblity takes the form of the following
commutative diagram: \[\begin{tikzcd}
        {\mathscr{F}} & {\mathscr{F}'} \\{\mathscr{G}} & {\mathscr{G}'}
        \arrow[from=1-1, to=2-1]
        \arrow[from=1-2, to=2-2]
        \arrow[from=1-1, to=1-2]
        \arrow[from=2-1, to=2-2]
    \end{tikzcd}\] where we have ommited the
subscripts. For the case of a locally free sheaf, we can take this cover to be
trivializing, and the gluing data can be taken to be the transition functions.
Thus, all we need is to check the compatiblity condition in order to define a
morphism. \qed

\section{13.1.B}
Note that this will only hold with $m, n$ finite.

Lemma: Suppose $M, N$ are $A$-modules. Let
$X=\Spec A$. Then morphisms $M \to N$ are in one to one
correspondence with morphisms $\widetilde{M} \to \widetilde{N}$. Indeed, it is clear that
if $\phi: M \to N$ is a morphism of modules then we get a morphism of
$\Osh_X$-modules by localization, call it $\widetilde{\phi}$.
It is clear that $\widetilde{\phi}(X)=\phi$. Now, suppose $f: \widetilde{M} \to \widetilde{N}$.
Consider $\widetilde{f(X)}$. We may use identity to show that this agrees
with $f$ on a distinguished cover, and gluing to show that
it then equals $f$.

Let $U_i$ be a trivialization of $\Fsh$ and
let $V_j$ be a trivialization of $\Gsh$. Then
\begin{align*}
    \Hom(\Fsh, \Gsh)(U_i \intersect V_i) & =\hom(\Fsh|_{U_i \intersect V_i}, \Gsh|_{U_i \intersect V_i})      \\
                                         & = \hom(\Osh^m|_{U_i \intersect V_i}, \Osh^n|_{U_i \intersect V_i}) \\
                                         & = \hom(\Osh^m(U_i \intersect V_i), \Osh^n(U_i \intersect V_i))
\end{align*}
Where the last isomorphism follows from the lemma above. Now, module theory
tells us that $\hom(A^m, A^n) \cong \sum_{i=1}^m
    \sum_{i=1}^n
    \hom(A,A) \cong A^{mn}$. Now suppose $T: A^n \to A^n$ and
$T': A^m \to A^m$ are isomorphisms of free modules. Then we have an
isomorphism $\hom(A^n, A^m) \to \hom(A^n, A^m)$ given by $T' \after \_ \after T^{-1}$. The
transition functions will be this morphism, followed by the isomorphism with
$A^{mn}$. This is easily checked to satisfy the cocyle condition.

Thus, on $U_i \intersect V_i$ we have defined an isomorphism to a free
module, and we have defined transition functions satisfing the cocyle
condition. Hence we have defined a locally free sheaf. \qed

\section{13.1.C}
Let $U_i$ be a trivialization for $\Esh$
with transition maps $T_{ij}$. Then on $U_{ij}=U_i \intersect U_j$
we see that we have $\Hom(\Esh, \Osh_X)(U_{ij})=\hom(\Esh|_{U_{ij}}, \Osh_X|_{U_{ij}})$. Evaluating on
$U_{ij}$, we find that we are essentially just applying the
contravariant hom functor. Thus, $T_{ij}$ can be transformed
into trivializations for the sheaf hom by precomposition,
$T'_{ji} = \_ \after T_{ij}$.

We see that $(\Esh^\vee)^\vee)(U_i) \cong (\Osh_X^{n,\vee})^\vee)(U_i)$. But we know that for a free module
$A^n$, $((A^n)^\vee)^\vee \cong A^n$ via a natural isomorphism
(assuming $n$ is finite). Because the isomorphism is
natural, we know the following diagram commutes: \[\begin{tikzcd}
        {A^n} &  & {A^n} \\{A^{n\vee\vee}} &  & {A^{n\vee\vee}}
        \arrow[from=1-1, to=2-1]
        \arrow[from=1-3, to=2-3]
        \arrow["{T_{ij}}", from=1-1, to=1-3]
        \arrow["{ \_ \circ \_ \circ T_{ij}}"', from=2-1, to=2-3]
    \end{tikzcd}\] But
this is exactly the compatiblity condition discussed above. \qed

\section{13.1.D}
If $T_{ij}$ and $T'_{ij}$ are transition functions
for $\Fsh$ and $\Gsh$ respectively then the
transition functions for $\Fsh \otimes \Gsh$ are simply
$T_{ij}\otimes T'_{ij}$. It is again trivial to check the cocyle condition.

It will suffice to show the compatibility condition. If $f(x)=ax$
is a transition function for $\Osh_X$ then
$(a \otimes a^{-1})x$ is the transition function for the tensor
$\Osh_X \otimes \Osh_X^\vee$. But this clearly equals the identity, which is the
transition function for a free sheaf. \qed

\section{13.1.E}
The considerations in 2.6 show that exactness may be checked on stalks.
Furthermore, the stalk of the tensor product is the tensor product of the
stalks, and the stalk of a locally free sheaf is free. Since free modules are
flat, they are exact, hence tensoring with a locally free sheaf is exact at the
level of stalks. Thus we are done. \qed

\section{13.1.F}
This will follow from the isomorphism of modules $\hom(A^n, M) \cong A^{n\vee} \otimes M$ and
the tensor hom adjunction.

If $p \in X$ and $\Fsh$ is finite locally free
of rank $n$ then $\Hom(\Fsh, \Gsh)_p \cong \hom(\Fsh_p, \Gsh_p)$. Indeed, let
$\Fsh|_U=\Osh_U^n$ where $p \in U$. Then
$\Hom(\Fsh, \Gsh)|_U \cong \Hom(\Osh_U^n, \Gsh|_U)\cong
    \Hom(\Osh_U, \Gsh|_U)^n\cong \Gsh|_U^n$. Taking stalks of both sides then gives the desired
result.

Next, we show that $\Fsh \otimes \Esh^\vee \cong \Hom(\Esh, \Fsh)$. We begin by defining the morphism
of presheaves $(\Fsh \otimes \Esh^\vee)^\text{pre} \to \Hom(\Esh, \Fsh)$. Let $U$ be
arbitrary. Then we define $f_U: \Fsh(U) \otimes \Esh^\vee(U) \to
    \hom(\Esh|_U, \Fsh|_U)$ by sending
$f \otimes e$ to the morphism of sheaves $g$
given by $g_U(x)=e|_U(x)f|_U$. It is easily checked that the following
diagram commutes: \[\begin{tikzcd}
        {\mathscr{F}(U)} &  & {\mathscr{G}(U)} \\{\mathscr{F}(V)} &  & {\mathscr{G}(V)}
        \arrow["{e|_U(u)f|_U}", from=1-1, to=1-3]
        \arrow[from=1-3, to=2-3]
        \arrow[from=1-1, to=2-1]
        \arrow["{e|_V(v)f|_V}"', from=2-1, to=2-3]
    \end{tikzcd}\] Hence we have defined the desired
morphism. By universal property of sheafification, we get a morphism
$\Fsh \otimes \Esh^\vee \to \Hom(\Esh, \Fsh)$. Now, since $\Esh$ is finite locally
free, we may localize at $p$ to check isomorphism. This
gives us a morphism of modules $\Fsh_p \otimes \Esh^\vee_p \to \hom(\Osh_p^n, \Fsh_p)$ such that
$f \otimes e \mapsto e(x)f$. By the general module theory above, this is an
isomorphism, hence we have the desired global isomorphism.

Next we need a tensor hom adjunction for sheaf hom. We show that
\[\Hom(\Fsh \otimes \Esh, \Gsh) \cong \Hom(\Fsh, \Hom(Esh, \Gsh))\]. For each $U \subset X$ we will need to
define an isomorphism $f_U: \hom(\Fsh|_U \otimes \Esh|_U, \Gsh|_U) \to \hom(\Fsh|_U, \Hom(Esh|_U, \Gsh|_U))$. By universal property of
sheafification, we know that $\hom(\Fsh|_U \otimes \Esh|_U, \Gsh|_U) \cong \hom^\text{pre}(\Fsh|_U \otimes \Esh|_U, \Gsh|_U)$, hence we replace the
domain of $f$ with this. Let $g \in \hom^\text{pre}(\Fsh|_U \otimes \Esh|_U, \Gsh|_U)$. Then
we have $g_U: \Fsh(U) \otimes \Esh(U) \to
    \Gsh(U)$. We now need to give a morphism
$\Fsh|_U \to \Hom(Esh|_U, \Gsh|_U)$. We will call this natural transformation
$\phi$. Let $W \subseteq V \subseteq U$. For all
$x \in \Fsh(V)$ and $y \in \Esh(W)$ we define
$\phi_V(x)_W(y)=g_W(x|_W \otimes y)$ (note that $\phi$ is a natural
transformation whose output is another natural transformation, hence two
subscripts and function evaluations). Naturality in both
$V$ and $W$ follows from naturality of
$g$. The inverse of this map is much easier to describe.
Suppose $\phi \in \hom(\Fsh|_U, \Hom(Esh|_U, \Gsh|_U))$. Then we define $g_V: \Fsh(V) \otimes \Esh(V) \to
    \Gsh(V)$ by
$g_V(x \otimes y)=\phi_V(x)_V(y)$. This is clearly a left inverse to the above map, so
that we have $f_U^{-1}(f_U(g))=g$. We check that it is a right inverse as
well. We have $f_U(f_U^{-1}(\phi))_V(x)_W(y)=\phi_W(x|W)_W(y)=\phi_V(x)_W(y)$, as desired. $\Osh_X$
linearity of these constructions is trivial to verify.

With all these pieces, the remainder of the proof is straightforward:
\begin{align*}
    \Hom(\Fsh \otimes \Esh^\vee, \Gsh) & \cong \Hom(\Fsh, \Hom(\Esh^\vee \Gsh))         \\
                                       & \cong \Hom(\Fsh, \Esh^{\vee\vee} \otimes \Gsh) \\
                                       & \cong \Hom(\Fsh, \Esh \otimes \Gsh)
\end{align*}
\qed

\section{13.1.G}
This is an easy consequence of 13.1.D. The identity is $\Osh_X$
and the inverse is $\_^\vee$. \qed

\section{13.1.H}
We begin by noting that if $B$ is an
$A$-algebra, then we have an obvious isomorphism
$A^n \otimes_A B \cong B^n$.

We obviously have an $\pi^{-1}\Osh_Y$-module structure on
$\pi^{-1}\Gsh$. In fact, this structure is locally free. But we also
have $\pi^{\sharp}: \pi^{-1}\Osh_Y \to \Osh_X$, realizing $\Osh_X$ as an
$\pi^{-1}\Osh_Y$. Hence, after tensoring with $\pi^{-1}\Osh_Y$,
we see that $\pi^*\Gsh = \Osh_X \otimes \pi^{-1}\Gsh$ is a locally free
$\Osh_X$-module.

Next we state some algebraic facts. Let $A \to B$ be an
$A$-algebra. Let $M, N$ be
$B$-modules. Let $L$ be an
$A$-module. Then $L \otimes_A (M \otimes_B N)$ is a
$B$-module in an obvious way. Furthermore, it is isomorphic
as a $B$-module to $(L \otimes_A M) \otimes_B N$. Also,
$L \otimes_A M \cong M \otimes_A L$ as $B$-modules.

With these, we can prove that extension of scalars distributes over tensor
products. Indeed, let $M, N$ be
$A$-modules and let $A\to B$ be an
$A$-algebra. Then
\begin{align*}
    (M \otimes_A N) \otimes_A B & \cong (M \otimes_A N) \otimes_A (B \otimes_B B) \\
                                & \cong M \otimes_A (B \otimes_B B) \otimes_A N   \\
                                & \cong (M \otimes_A B) \otimes_B (B \otimes_A N)
\end{align*}
as desired.

Finally, it will suffice to show that $\pi^*\Osh_Y=\Osh_X$ and
$\pi^*(\Fsh \otimes_{\Osh_Y} \Gsh)=\pi^*\Fsh \otimes_{\Osh_X} \pi^*\Gsh$. Indeed, the former is trivial. For the latter, we have
by definition

\begin{align*}
    \pi^*(\Fsh \otimes_{\Osh_Y} \Gsh) & =\Osh_X \otimes_{\pi^{-1}\Osh_Y} \pi^{-1}(\Fsh \otimes_{\Osh_Y} \Gsh)                      \\
                                      & \cong \Osh_X \otimes_{\pi^{-1}\Osh_Y} (\pi^{-1}\Fsh \otimes_{\pi^{-1}\Osh_Y} \pi^{-1}\Gsh) \\
                                      & \cong \pi^*\Fsh \otimes_{\Osh_X} \pi^*\Gsh
\end{align*}
Where the first isomorphism follows from the fact that tensor products preserve
colimits, and the the last isomorphism follows from the above algebraic fact.
\qed

\section{13.1.I}
Given a closed subscheme $\iota: S \to X$, we may pull back a sheaf
$\Fsh$ on $X$ to a sheaf on
$S$. Given a global section $s$ of
$\Fsh$, we thus get a global section $s'$
of $\iota^*\Fsh$. If $s'=0$, then we say that
$s$ vanishes on the closed subscheme
$S$. We define $\V(s)$ by universal
property. Given any closed subscheme $S$ on which
$s$ vanishes, this subscheme should factor uniquely through
$\V(s)$. In other words, the following diagram should commute:
\[\begin{tikzcd}
        {V(s)} & X \\
        S
        \arrow[hook, from=1-1, to=1-2]
        \arrow[hook, from=2-1, to=1-2]
        \arrow["{\exists!}", dotted, hook, from=2-1, to=1-1]
    \end{tikzcd}\]

We now construct this object explicitely. Let $U_i$ be a
trivializing cover of $\Fsh$ with transition functions
$T_{ij}$. Let $s_i=s|_{U_i}$. Then
$s_i=[s_{i1}, \dots, s_{in}]$ for some $s_{ik} \in \Osh_X(U_i)$. Define a sheaf of
ideals by $\Ish(U_i)=I_i=(s_{ik})$. Now we need to check that this definition
agrees on the overlaps. From now on we implicitely restrict everything to
$U_i \intersect U_j$. We need to check $I_i=I_j$. Let
$p=a^\text{T} s_i \in I_i$ where $a$ is a vector of
coefficients. We know that $a^\text{T} s_i=a^\text{T}
    T_{ji}T_{ij} s_i=a^\text{T}
    T_{ji} s_j$, hence
$p \in I_j$. This is clearly symmetric, hence
$I_i=I_j$. We define $\V(s)$ to be the closed
subscheme corresponding to $\Ish$.

We now need to check the universal property. We can do this on a trivializing
affine cover. On a given element of the cover we have a morphism
$\iota: \Spec A/I \to \Spec A$ and an isomorphism $\iota^*\Fsh|_{\Spec A} \cong \Osh_{\Spec A/I}^n$. From this
it is easy to see that $s$ vanishes on
$\V(s)$. It remains to show universality. Suppose
$\iota': \Spec A/J \to \Spec A$ is a closed embedding such that $s$
vanishes on $\iota'^*\Fsh|_{\Spec A}$. By definition, this means that the
components of $s$ are elements of $J$.
But then $I \subseteq J$ so that $A \to A/J$ factors
uniquely as $A \to A/I \to A/J$, as desired. \qed

\section*{13.1.M}

\section*{13.2.A}
\begin{enumerate}[a.]
    \item We have $k(t)_t=k(t) \neq 0 = \Gamma(\D(t), \Fsh)$.
    \item We know that every nonempty open set contains the generic point, hence
          $\Fsh$ is actually $\widetilde{k(t)}$. Indeed,
          $k(t)_f=k(t)=\Gamma(\D(f), \Fsh)$.
\end{enumerate}
\qed

\section*{13.2.B}
$k(t)$ is not a free $k[t]$-module. Indeed,
let $f/f', g/g' \in k(t)$ where $f,f',g,g' \in k[t]$. Then
$f'g f/f' - g'f g/g' = 0$ so that $f/f', g/g'$ are linearly dependent.
\qed

\section*{13.2.C}
Let $\Fsh$ be finite locally free of rank
$n$. Since $X$ is affine,
$\Fsh=\widetilde{M}$ for some $k[x]$-module
$M$. By 5.3.3 we know that $M$ is
finitely generated. Now, by the structure theorem for finitely generated
modules over a PID, we know that $M = \bigoplus_i k[x]/(f_i)$. Localizing at
$g$, we find that $k[x]_g/(f_i)$ must equal
$k[x]_g$, hence $f_i=0$. Since
$R^n \neq R^m$ for $n \neq m$, we find that
$M$ is free of rank $n$.\qed

\section*{13.2.D}
This follows from gluing of sheaves. We know that $\Fsh|_{\Spec A}$ is
the result of gluing $\widetilde{M_i}$ along the $\phi_{ij}$.
But if we can find a module $M$ with the described
property, then $\widetilde{M}$ is the result of gluing
$\widetilde{M_i}$ along $\phi_{ij}$. Hence by uniqueness of
gluing, we get the desired result.\qed

\section*{13.2.E}
Without loss of generality it suffices to show this for $i=1$
and $j=2$. By the preceeding discussion, we have a diagram
\[\begin{tikzcd}
        0 & {M_{f_1}} & {\prod (M_1)_{f_i}} & {\prod_{i<j} (M_1)_{f_if_j}} \\
        0 & {M_1}     & {\prod (M_1)_{f_i}} & {\prod_{i<j} (M_1)_{f_if_j}}
        \arrow[from=1-1, to=1-2]
        \arrow[from=1-2, to=1-3]
        \arrow[from=1-3, to=1-4]
        \arrow[from=2-2, to=2-3]
        \arrow[from=2-3, to=2-4]
        \arrow[from=2-1, to=2-2]
        \arrow[from=1-2, to=2-2]
        \arrow[from=1-3, to=2-3]
        \arrow[from=1-4, to=2-4]
    \end{tikzcd}\] Where the vertical maps are isomorphisms. It is easy to
check that $M_{f_1} \to M_1$ is given by projection onto the first
coordinate.Tensoring with $f_2$ gives us an isomorphism
$\psi: M_{f_1f_2} \to (M_1)_{f_2}$. By an isomorphic argument, we also get an isomorphism
$\psi': M_{f_1f_2} \to (M_2)_{f_1}$. Let $m=(m_1, \dots , m_n) \in M_{f_1}$. Then
$\psi(m|_{f_2})=m_1|_{f_2}=\phi_{21}(m_2|_{f_1})=\phi(\psi'(m|_{f_1}))$ as desired.\qed

\section*{13.3.D}
We note that the given property holds by definition for the sheaf
$\widetilde{M}$ determined by the module $M$. This
shows the forward implication. In the reverse direction, given an affine
$\Spec A$, we may form the sheaf $\widetilde{\Fsh(\Spec A)}$. By
hypothesis, this sheaf agrees with $\Fsh$ on the distinguished
affine base on $\Spec A$, hence $\Fsh|_{\Spec A} \cong \widetilde{\Fsh(\Spec A)}$. \qed

\section*{13.3.E}
The proof here is identical to the proof of the QCQS lemma.

\section*{13.3.F}
Let $U=\pi^{-1}(\Spec A)$. Then \[\pi_*\Fsh(\Spec A)_s = \Fsh(U)_s \to
    \Fsh(U_s)=\pi_*\Fsh(\Spec A_s)\] by 13.3.E, hence by
13.3.D it is quasicoherent. \qed

\section*{13.3.G}
Note that a sheaf of ideals which determines a closed subset is precisely the
data of a quasicoherent sheaf of ideals. In chapter 8 we noted that any ideal
of a ring determines a closed subset. \qed

\section*{13.3.H}
We start with some graded ring theory. Let $A$ be a ring.
Then $\bigoplus_{n \geq 0} A^{\otimes n}$ is a graded ring. Note that
$A^{\otimes 0}=A$ by definition. For any $s \in A$ and any
$i\geq 0$ we have a degree $i$ copy of
$s$ which we write $s_i$. Note that
$(s_i)^n=(s^n)_{ni}$. In a similar manner, $\bigoplus_{n \geq 0} M \otimes A^{\otimes n}$ is a
graded module which, for each $m \in M$ and
$i \geq 0$, contains an $m_i$ of degree
$i$. Note that this is isomorphic to
$\bigoplus M$ as an $A$-module, and it may be
given the obvious induced grading. More generally, $\bigoplus_{n \geq 0} M^{\otimes n}$ is a
graded ring as well.

Next we note some basic module theory. The map $M \to M_i$ of
graded modules, corresponding to taking degree $i$, is an
exact functor. Furthermore, it preserves arbitrary direct sums. These facts
follows easily from the fact that morphisms of graded modules preserve degrees.
Next, if $M$ and $N$ are graded modules
over $A$ and $s$ is a degree 1 section
then $(M \oplus N)_s \cong M_s \oplus N_s$ as graded modules. There is clearly an
isomorphism, it merely remains to show it preserves grading. Let
$(m+n)/s^i$ be homogenous of degree $j$. Then
all components of $m$ and $n$ of degree
less than $j+i$ must cancel so that $m/s^i+n/s^i$
is homogenous of degree $j$, as desired.

Now, we move to our theorem. We begin with the case of a trivial line bundle on
an affine scheme, which translates to the following: Let
$A$ be a ring and $M$ an
$A$-module. Let $s \in A$. Then the following
holds: \[
    (\bigoplus_{n \geq 0} M \otimes
    A^{\otimes n})_{s_1,0} \cong
    M_s
\] as $A$-modules. As noted
above, the left is $(\bigoplus M)_{s_1,0}$. This is generated by elements of
the form $m_i/(s^i)_i$ where $m \in M$. Then we may
send such an element to $m/s^i$. It is easily verified that
this is an isomorphism.

We define $\Gamma_*(X, \Fsh, \Lsf) = \bigoplus_{n\geq 0}\Gamma(X, \Fsh \otimes \Lsf^{\otimes n})$. By definition of a sheaf, we have an exact
sequence of the form \[\begin{tikzcd}
        0 & {\Gamma_*(X, \mathscr{F}, \mathscr{L})} & {\bigoplus \Gamma_*(U_i, \mathscr{F}, \mathscr{L})} & {\bigoplus \Gamma_*(U_{ijk}, \mathscr{F}, \mathscr{L})}
        \arrow[from=1-1, to=1-2]
        \arrow[from=1-2, to=1-3]
        \arrow[from=1-3, to=1-4]
    \end{tikzcd}\] Where the $U_i$
form a finite affine cover and where $U_{ijk}$ forms a finite
affine cover of $U_i \intersect U_j$ for every $i,j$.
Since taking degree 0 and localizing commutes with finite direct sums, we know
that   \[\begin{tikzcd}[column sep=small]
        0 & {\Gamma_*(X, \mathscr{F}, \mathscr{L})_{s,0}} & {\bigoplus \Gamma_*(U_i, \mathscr{F}, \mathscr{L})_{s_i,0}} & {\bigoplus \Gamma_*(U_{ijk}, \mathscr{F}, \mathscr{L})_{s_{ijk},0}}
        \arrow[from=1-1, to=1-2]
        \arrow[from=1-2, to=1-3]
        \arrow[from=1-3, to=1-4]
    \end{tikzcd}\] Where the $s_i$ are the
restrictions of $s$ to the $U_i$. Now, by
the result affine locally, we have the following diagram:
\[\begin{tikzcd}[column sep=small]
        0 & {\Gamma_*(X, \mathscr{F}, \mathscr{L})_{s,0}} & {\bigoplus \Gamma_*(U_i, \mathscr{F}, \mathscr{L})_{s_i,0}} & {\bigoplus \Gamma_*(U_{ijk}, \mathscr{F}, \mathscr{L})_{s_{ijk},0}} \\
        0 & {\Gamma(X_s, \mathscr{F})}                    & {\bigoplus\Gamma(U_{i,s_i}, \mathscr{F})}                   & {\bigoplus\Gamma(U_{ijk,s_{ijk}}, \mathscr{F})}
        \arrow[from=2-1, to=2-2]
        \arrow[from=2-2, to=2-3]
        \arrow[from=2-3, to=2-4]
        \arrow[from=1-1, to=1-2]
        \arrow[from=1-2, to=1-3]
        \arrow[from=1-3, to=1-4]
        \arrow[from=1-3, to=2-3]
        \arrow[from=1-4, to=2-4]
        \arrow[dashed, from=1-2, to=2-2]
    \end{tikzcd}\] It is easily checked that the square commutes, hence
the dotted morphism exists. Since the vertical maps are isomorphisms, the
dotted map must be an isomorphism as well. \qed

\section*{13.3.I}

\section*{13.4.A}
The reverse implication follows by stalk-locallness of exactness. For the
forward implication, note that exactness is preserved by restriction. Hence
$\Fsh|_{U} \to \Gsh|_{U} \to \Hsh|_{U}$ gives us an exact sequence $\widetilde{M} \to \widetilde{N} \to \widetilde{P}$.
Since morphisms of sheaves arising from modules arise from morphisms of
modules, we know that $M \to N \to P$ must be exact. \qed

\section*{13.4.B}
The forward implication follows trivially from the fact that modules have a
free presentation. For the reverse implication note $\coker \widetilde{f}=\widetilde{\coker f}$.
Since $U_i$ admits a cover by affine opens, we know that
$\Fsh$ is locally the cokernel of a map of modules. Since
quasicoherence can be checked on an arbitrary cover, we know
$\Fsh$ is quasicoherent.

Note that we used no special properties about $\Osh_{U_i}^I$ other
than quasicoherence in this proof. Hence, if $X$ is a
scheme then a $\Osh_X$-module is quasicoherent iff it is locally
the cokernel of a morphism of quasicoherent modules. \qed

\section*{13.5.A}
\begin{enumerate}[a.]
    \item We recall basic module theory stating that any short exact sequence where the
          last module is free (more generally, projective)splits. In particular
          $\Fsh|_{\Spec A}\cong\widetilde{A^{a+b}}$, as desired.
    \item It suffices to show that for any diagram of the following shape
          \[\begin{tikzcd}
                  0 & {A^n} & {A^{n+m}} & {A^m} & 0 \\
                  0 & {A^n} & {A^{n+m}} & {A^m} & 0
                  \arrow[from=1-1, to=1-2]
                  \arrow[from=1-2, to=1-3]
                  \arrow[from=1-3, to=1-4]
                  \arrow[from=1-4, to=1-5]
                  \arrow[from=2-1, to=2-2]
                  \arrow[from=2-2, to=2-3]
                  \arrow[from=2-3, to=2-4]
                  \arrow[from=2-4, to=2-5]
                  \arrow[from=1-2, to=2-2]
                  \arrow[from=1-4, to=2-4]
                  \arrow[dashed, from=1-3, to=2-3]
              \end{tikzcd}\] where the vertical maps are isomorphisms, then the
          dotted isomorphism is is block upper triangular when represented as a matrix.
\end{enumerate}
\qed

\section*{13.5.B}
This reduces to a statement for modules. Suppose  \[\begin{tikzcd}
        0 & M & {A^m} & {A^n} & 0
        \arrow[from=1-4, to=1-5]
        \arrow["\psi", from=1-3, to=1-4]
        \arrow["\phi", from=1-2, to=1-3]
        \arrow[from=1-1, to=1-2]
    \end{tikzcd}\] is an
exact sequence of $A$-modules. Then $M$
is locally free of rank $m-n$.

We first prove a lemma. Let $M$ be a finitely generated
projective module over a local ring. Then $M$ is free.
Moreover, any minimal generating set is a basis. Indeed, since
$M$ is finitely generated, we have a surjection
$\phi: A^n \to M$, corresponding to a choice of generators for
$M$. Since $M$ is projective, we may
write $A^n \cong M \oplus \ker \phi$. Tensoring $\phi$ with
$A/\m$, we note that $k^n \cong M/\m M \oplus \ker \phi/\m\ker \phi$. It is clear by
examination that $\ker \phi/\m\ker \phi = 0$, hence by Nakayama, we must have
$\ker \phi = 0$. In particular, $\phi$ is an
isomorphism and the choice of generators is a basis.

Recall that localization is exact, hence for any prime $\p$
we have \[\begin{tikzcd}
        0 & {M_\mathfrak{p}} & {A^m_\mathfrak{p}} & {A^n_\mathfrak{p}} & 0
        \arrow[from=1-4, to=1-5]
        \arrow["\psi_\p", from=1-3, to=1-4]
        \arrow["\phi_\p", from=1-2, to=1-3]
        \arrow[from=1-1, to=1-2]
    \end{tikzcd}\] $A_\p^n$ is projective, hence by
the lemma above, any set of $n$ generators chosen from the
columns of $\psi$ must be a basis. A basis is equivalent to an
automorphism, hence we have an automorphism $\zeta_\p: A_\p^n \to A_\p^n$. Now, each
column of $\zeta_\p$ arises from a section over an open set
$\D(f_i)$. Since there are finitely many columns, we may multiply
these $f_i$ together to find an open set
$\D(f)$ on which each of the columns is defined, giving us a
morphism $\zeta_f$. Isomorphisms are precisely the morphisms with
invertible determinants, hence there is some open set $\D(g) \subset \D(f)$
on which $\zeta_{g}$ is an isomorphism. Hence we find that on this
open set the columns of $\zeta$ form a basis. We may now
restrict to this open set and reorder the the columns of
$\phi$ so that the columns of $\zeta$ come
first.

We now wish to construct the dotted morphism in the following diagram:
\[\begin{tikzcd}
        0 & M         & {A^m} & {A^n} & 0 \\
        0 & {A^{m-n}} & {A^m} & {A^n} & 0
        \arrow[from=1-4, to=1-5]
        \arrow["\psi", from=1-3, to=1-4]
        \arrow["\phi", from=1-2, to=1-3]
        \arrow[from=1-1, to=1-2]
        \arrow["\iota", from=2-2, to=2-3]
        \arrow["\pi", from=2-3, to=2-4]
        \arrow[from=2-4, to=2-5]
        \arrow["{\zeta^{-1}}", from=1-4, to=2-4]
        \arrow[from=1-3, to=2-3]
        \arrow[from=2-1, to=2-2]
        \arrow[dashed, from=1-2, to=2-2]
    \end{tikzcd}\] In this diagram, $\iota$ is the
inclusion of $A^{m-n}$ as the last entries
$m-n$ entries of vectors in $A^m$. The
morphism $\pi$ is the obvious quotient morphism, with the
first $n$ entries being the identity matrix, and the last
$m-n$ being zero. The vertical maps are constructed by taking
the change of coordinates $\zeta^{-1}$ on $A^n$
which takes the first $n$ columns of
$\psi$ to the standard basis. The morphism
$A^m \to A^m$ is the identity matrix. It is clear by definition that
the square commutes. Standard theory of exact sequences then allow us to define
the morphism $M \to A^{m-n}$, and the five lemma lets us show that it
is an isomorphism.

For part (b), we are asked to show that $k[t]/(t)$ is not a free
$k[t]$-module. For this it suffices to show that 1 is linearly
dependent. Indeed, $t*1=0$, as desired. \qed

\section*{13.5.C}
This comes down to the fact that sheafification commutes with restriction to
opens, hence on an affine cover, the sheafification is trivial. \qed

\section*{13.5.D}
We begin by defining $T^n \Fsh = \Fsh^{\otimes n}$. This satisfies the obvious
extension of the tensor product's universal property. It is also obviously
quasicoherent.

We also define $\Sym^n \Fsh$ by universal property. A map of
quasicoherent sheaves $T^n \Fsh \to \Gsh$ is said to be symmetric if the
map of modules is symmetric on an open cover. Given any symmetric map
$\phi: T^n \Fsh \to \Gsh$, there is a symmetric map $T^n \Fsh \to \Sym^n \Fsh$ such
that $\phi$ factors through it. $\wedge^n \Fsh$ is
defined similarly for antisymmetric maps. To show that both these exist and are
quasicoherent, it suffices to note that localization commutes with both tensor
products and quotients.

TODO

\section{13.6.A}
Consider the following diagram. \[\begin{tikzcd}
          & {}   &          & 0     \\
        0 & K    & {A^p}    & M & 0 \\
        0 & {K'} & {A^{p'}} & M & 0 \\
          & C    & {C'}     & 0
        \arrow[Rightarrow, no head, from=2-4, to=3-4]
        \arrow[from=2-3, to=3-3]
        \arrow[from=2-2, to=3-2]
        \arrow[from=2-1, to=2-2]
        \arrow[from=3-1, to=3-2]
        \arrow[from=2-2, to=2-3]
        \arrow[from=3-2, to=3-3]
        \arrow[from=3-3, to=3-4]
        \arrow[from=2-3, to=2-4]
        \arrow[from=2-4, to=2-5]
        \arrow[from=3-4, to=3-5]
        \arrow[from=1-4, to=2-4]
        \arrow[from=3-4, to=4-4]
        \arrow[from=3-2, to=4-2]
        \arrow[from=3-3, to=4-3]
        \arrow[from=4-2, to=4-3]
        \arrow[from=4-3, to=4-4]
    \end{tikzcd}\] The map
$A^p \to A^{p'}$ exists due to the lifting property of projective
modules. This map then induces the map $K \to K'$ in the obvious
way. Not shown in the diagram is the snake lemma map $0 \to C$.
Since we have an exact sequence $0 \to C \to C' \to 0$, we know
$C \cong C'$. $C'$ is obviously finitely
generated since it is the cokernel of a map whose codomain is finite free.
Since both $K$ and $C$ are finitely
generated, $K'$ must be also. \qed

\section{13.6.B}
The reverse implication is trivial, as $A \to A \to A$ is a finite
presentation of $A$, hence it is coherent. Suppose
$A$ is a coherent $A$-module and
$M$ is finitely presented. Let $\phi: A^q \to M$ be
arbitrary. Then we have the following diagram: \[\begin{tikzcd}
          &         & {K'}      & K         \\
        0 & {A^{p}} & {A^{p+q}} & {A^q} & 0 \\
        0 & {K''}   & {A^n}     & M     & 0 \\
          & 0
        \arrow[from=3-1, to=3-2]
        \arrow[from=3-2, to=3-3]
        \arrow[from=3-3, to=3-4]
        \arrow[from=3-4, to=3-5]
        \arrow["\phi", from=2-4, to=3-4]
        \arrow[from=2-4, to=2-5]
        \arrow[dashed, from=2-3, to=3-3]
        \arrow[from=2-3, to=2-4]
        \arrow[from=2-2, to=2-3]
        \arrow[from=2-2, to=3-2]
        \arrow[from=2-1, to=2-2]
        \arrow[from=1-3, to=1-4]
        \arrow[from=1-3, to=2-3]
        \arrow[from=3-2, to=4-2]
        \arrow[from=1-4, to=2-4]
    \end{tikzcd}\] where
$A^n \to M$ arises from coherence of $M$.
$A^p \to K''$ is a surjection arising from the fact that
$K''$ is finitely generated. The dotted morphism is the one
induced by the two outer morphisms. Since $A^n$ is coherent,
its kernel $K'$ is finitely generated. Not shown in the
diagram is the snake lemma morphism which gives us an exact sequence
$K' \to K \to 0$. Hence $K$ is the quotient of a
finitely generated module, hence is finitely generated. \qed

\section{13.6.C}
Finitely generated and presented are trivial since localization is exact. Let
$M$ be coherent, and let $f \in A$. We know
$A^n$ is finitely presented, hence by 1.6.G,
$\phi_f: A_f^n \to M_f$ is the localization of a morphism
$\phi: A^n \to M$. By coherence of $M$, the kernel of
this is finitely generated, hence the kernel of $\phi_f$ is too.
\qed

\section{13.6.D}
For finitely generated, note that the argument of 7.3.O applies easily here. We
prove a more general result. Suppose $M_{f_i}=0$ for all
$f_i$. Then $M = 0$. Indeed, this follows by
the fact that $\widetilde{M}$ is a sheaf. Now, consider
$N \to M \to N/M$. If $(N/M)_{f_i}=0$ then
$N \to M$ is surjective. Similarly for injective. Finite
generation follows directly from this For finite presentation, let
$0 \to K \to A^p \to M \to 0$ be exact. Then $K_{f_i}$ is finitely
generated for all $f_i$, hence $K$ is as
well. It remains to show coherence. Let $0 \to K \to A^p \to M$ be an exact
sequence. By the same argument for f.p., we get our result. \qed

\section{13.7.A}
\begin{enumerate}[a.]
    \item Let $U_i$ be an affine open cover. Then \[\Hom(\Fsh, \Gsh)(U_i)=\hom(\Fsh|_{U_i}, \Gsh|_{U_i}) \cong
              \hom(\Fsh(U_i), \Gsh(U_i)\]
          where the last isomorphism follows by quasicoherence. By 1.6.G, we meet the
          criteria of 13.3.D.
    \item Recall that we may check this on any affine cover. Hence, we reduce to showing
          that $\hom(M, N)$ is coherent when $M, N$ and
          $A$ are all coherent. In fact, more is true. We do not
          require $A$ to be coherent, and we merely require
          $M$ to be finitely presented.

          Let $A^a \to A^b \to M \to 0$ be a finite presentation of $M$.
          By left exactness of $\hom$, we get an exact sequence
          \[\begin{tikzcd}
                  0 & {\hom(M, N)} & {\hom(A^b, N)} & {\hom(A^a, N)}
                  \arrow[from=1-1, to=1-2]
                  \arrow[from=1-2, to=1-3]
                  \arrow[from=1-3, to=1-4]
              \end{tikzcd}\] Since finite direct sums commute with
          $\hom$, we get \[\begin{tikzcd}
                  0 & {\hom(M, N)} & {N^b} & {N^a}
                  \arrow[from=1-1, to=1-2]
                  \arrow[from=1-2, to=1-3]
                  \arrow[from=1-3, to=1-4]
              \end{tikzcd}\] Since the kernel of a map
          of coherent modules is coherent, we have our desired result.
    \item First we note that we have functoriality and exactness by 2.6.H. The domains
          and codomains of the relevant functors follow directly from (a) and (b).

          Note that the result of 2.6.H is easier if $\Hom$ is
          quasicoherent since the question of exactness may be checked on an affine open
          cover, where we have morphisms of modules rather than of sheaves.
\end{enumerate}

\qed

\section*{13.7.B}
We proved the general result in 13.1.F. For the quasicoherent case, this
follows trivially from the result for modules. \qed

\section*{13.7.C}
Let $F$ be an additive functor on an abelian category. Then
$F$ preserves split exact sequences. Indeed, it preserves
the inclusion of the first factor and the projection onto the second factor.

Recall that by 13.4.A we may reduce to showing the following result: If
\[\begin{tikzcd}
        0 & M & N & {A^n} & 0
        \arrow[from=1-1, to=1-2]
        \arrow[from=1-2, to=1-3]
        \arrow[from=1-3, to=1-4]
        \arrow[from=1-4, to=1-5]
    \end{tikzcd}\] is exact, then the sequence with
$\hom$ applied is exact. $hom$ is by
definition an additive functor. All we need show now is that the sequence
splits. This follows immediately from the fact that $A^n$ is
projective. \qed

\section*{13.7.D}
Remember that a set is closed iff it is closed in an open cover. Hence this
question may be rediced to the affine case. Suppose $\p \in \operatorname{Supp} \Fsh$
Note that for any $s, t$ and any $a, b$ not
annihilating them, $\operatorname{Supp}(as+bt) \subseteq \operatorname{Supp}(s) \union
    \operatorname{Supp}(t)$. Suppose $X$ is
affine. Then given any finite set of generators $s_i$,
$\operatorname{Supp}(\Fsh) \subseteq \Union_i \operatorname{Supp}(s_i)$. The reverse inclusion is by definition. We have
written $\operatorname{Supp}(\Fsh)$ as the finite union of closed sets, hence it is
closed.

We know $\C[t]/(t-a)$ is a $\C[t]$ module which is
supported at $a$. Indeed, suppose $b \neq a$.
Then $(\C[t]/(t-a))_{(t-b)} \cong
    (\C[t]/(t-a))_{(a-b)} \cong 0$. Then $M=\oplus_{a \in \C} \C[t]/(t-a)$ is clearly
infinitely generated. $(0) \in \operatorname{Supp}(M)$ since localization distributes
over direct sums, and localizing at $(0)$ inverts
$t$. Thus if $\operatorname{Supp}(M)$ is closed, it is the
entire space. But the localization of $M$ at
$(t-a)$ is clearly just $\C[t]/(t-a)$. Hence the
generic point is the only point in $\operatorname{Supp}(M)$. \qed

\section*{13.7.E}
Note that by 13.4.A, property (i) is equivalent to property (ii). Suppose
$\Spec A\\ = U \ni \p$ is affine. Let $\Fsh|_U=\widetilde{M}$. Then
$\Fsh|_{\p}=M_{\p}/\p M_{\p}$. By Nakayama, if $a_i|_{\p}$ generate
$M_{\p}/\p M_{\p}$ then $a_{i, \p}$ generates
$M_{\p}$. Consider $\pi: A^n \to M$ defined by the
$a_i$. Now, as argued in 7.4.F there exists some
$s \in A$ such that $s \coker \pi=0$. Thus
$\pi_s$ is surjective, showing property (i). \qed

\section*{13.7.F}
We essentially proved this in 7.4.F.

\section*{13.7.G}
TODO once we do ch 12

\section*{13.7.H}
We have 3 cases. First, at the point $(t)$, the rank is 1
since $(k[t]/(t))_{(t)}
    \cong k$. At the other closed points we have rank 0 for
reasons discussed in 13.7.D. At the generic point we have
$(k[t]/(t))_{(0)}
    \cong 0$ since we are inverting $t=0$. Thus
$\operatorname{rank} \Fsh = 0$.

\section*{13.7.I}
Follows trivially from the hint.

\section*{13.7.J}
By 13.7.E if $\Fsh$ has rank $n$ at
$\p$ then it has rank $\leq n$ on an open set
around $\p$. Thus the set of points of rank
$\leq n$ is open. It is clear that the inverse image of
$(-\infty, n+1)$ is the same as the inverse image of
$(-\infty, n]$. But by the above argument, this is open, as desired.
\qed

\section*{13.7.K}
\begin{enumerate}[a.]
    \item For any point $p$, we know that there exists an epimorphism
          $\Osh_U^n \to \Fsh|_U$ for some affine $U$ (by 13.7.E). So we
          reduce to the affine case. Let $\phi: A^n \to M$ be a surjection giving a
          generating set $m_i$. Then by definition if
          $x \in \ker \phi$ we have $\sum_i x_im_i=0$. WLOG assume
          $x_1$ is nonzero. Then $x_1$ is non-nilpotent
          (by reducedness), hence $\phi_{x_1}$ is non-zero. In particular
          $A_{x_1}$ has a point $\p$. Localizing at
          $\p$, then passing to the quotient, we find that
          $M/\p M$ is generated by the $m_i$, but also
          that $\sum_i x_im_i=0$ is a linear dependence relation among the
          $m_i$. By basic linear algebra, this means that the rank is
          smaller than $n$ at $\p$. But this
          contradicts our assumption of constant rank. Hence, $\ker \phi = 0$.

          Suppose $X$ is an integral scheme. Then every open is dense.
          Let $n=\min_{p \in X} \operatorname{rank}(\Fsh)(p)$. Since $\Fsh$ is finitely generated,
          this is finite. By upper semicontinuity, $U=\operatorname{rank}(\Fsh)^{-1}((-\infty, n+1))$ is open, and by
          construction, $\Fsh|_U$ has constant rank. Hence
          $\Fsh|_U$ is locally free. \qed

    \item Define $A=k[x]/(x^2)$ and $M=A/(x)$ with the obvious
          $A$-algebra structure. $M$ is a field and
          $A$ has one point, $(x)$, so
          $M$ clearly has constant rank. $\Spec A$ also
          only has one non-empty open set, so it suffices to show that
          $M$ is not free. Indeed, $1$ is
          annihilated by $x$, so $1$ is linearly
          dependent, so we are done. \qed
\end{enumerate}

\section*{13.7.L}
This is clearly affine local, hence we reduce to the case of a finite morphism
$A \to B$ and a prime $\p \subseteq A$. Then we know that
$(B \otimes_A A_\p) \otimes_{A_\p} \kappa(\p) \cong B
    \otimes_A \kappa(\p)$ is the fiber of $B$ at
$\p$. But $\Spec B \otimes_A \kappa(\p)$ is precisely the scheme
theoretic preimage of $\p$. Hence the dimensions of the two
as $\kappa(\p)$ vector spaces agree. \qed

\end{document}
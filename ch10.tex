\documentclass{article}
\usepackage[utf8]{inputenc}

\usepackage{mathtools}
\usepackage{amsthm}
\usepackage{amssymb}
\usepackage{dsfont}
\usepackage{array}   % for \newcolumntype macro
\usepackage{calligra}
\usepackage{tikz-cd}
\usepackage{mathrsfs}
\usepackage{quiver}
\usepackage[shortlabels]{enumitem}

\newcolumntype{L}{>{$}l<{$}}
% math-mode version of "l" column type

\newcommand{\Rho}{\mathrm{P}}

\newcommand{\iso}{\simeq}
\newcommand{\after}{\circ}
\newcommand{\catname}[1]{\mathbf{#1}}
\newcommand{\intersect}{\cap}
\newcommand{\union}{\cup}
\newcommand{\Intersect}{\bigcap}
\newcommand{\Union}{\bigcup}
\newcommand{\wa}[1]{\langle #1 \rangle}
\newcommand{\bb}[1]{\mathbb{#1}}
\newcommand{\A}{\mathbb{A}}
\newcommand{\Z}{\mathbb{Z}}
\newcommand{\Q}{\mathbb{Q}}
\newcommand{\R}{\mathbb{R}}
\newcommand{\C}{\mathbb{C}}
\newcommand{\F}{\mathbb{F}}
\newcommand{\p}{\mathfrak{p}}
\newcommand{\q}{\mathfrak{q}}
\newcommand{\ai}{\mathfrak{a}}
\newcommand{\bi}{\mathfrak{b}}
\newcommand{\m}{\mathfrak{m}}
\newcommand{\defeq}{\vcentcolon=}

\DeclareMathOperator{\id}{id}
\DeclareMathOperator{\Hom}{\mathscr{H}\text{\kern -3pt
{\calligra\large om}}\,}
\DeclareMathOperator{\Fsh}{\mathscr{F}}
\DeclareMathOperator{\Gsh}{\mathscr{G}}
\DeclareMathOperator{\Hsh}{\mathscr{H}}
\DeclareMathOperator{\Osh}{\mathscr{O}}
% \DeclareMathOperator{\ker}{ker}
\DeclareMathOperator{\coker}{coker}
\DeclareMathOperator{\im}{im}
\DeclareMathOperator{\Spec}{Spec}
\DeclareMathOperator{\Proj}{Proj}
\DeclareMathOperator{\V}{V}
\DeclareMathOperator{\D}{D}

\newcommand*{\rationalto}[1][]{\mathbin{\tikz [baseline=-0.25ex,-latex, dashed,->,densely dashed,#1] \draw [#1] (0pt,0.5ex) -- (1.3em,0.5ex);}}%


\title{Chapter 10 Problems}
\author{Josiah Bills}
\date{May 2021}

\begin{document}

\maketitle

\section{10.1.A}
Suppose first that the space is Hausdorff. Let $x \in U, x' \in V$ with
$U \intersect V \neq \emptyset$. Consider $(x, x') \in X \times X$. Then
$U \times V$ is an open neighborhood of $(x, x')$ which
doesn't meet the diagonal. Hence the complement of the diagonal is a union of
open sets. Suppose the diagonal is closed. Given two distinct points
$x, x' \in X$ we may choose an open set $U \times V$ which
doesn't meet the diagonal. Then, if $x'' \in U$ we have
$x'' \notin V$, hence $U \intersect V = \emptyset$ as desired. \qed

\section{10.1.B}
This is just 1.3.W \qed

\section{10.1.C}
This corresponds to the obvious statement that $A \otimes_B A \to A$ is
surjective. \qed

\section{10.1.D}
Let $X$ be the affine line with two origins, and cover it
with $\A^1_k=U_1, U_2$. A straightforward calculation shows
$X \times_k X$ is covered by four planes glued together on the open set
with the origin removed. This is the affine plane with four origins. Now,
consider $U_1 \times_k U_2 \cong U_2 \times_k U_1$. By 10.1.6, we see that $\Delta \intersect (U_1 \times_k U_2) \cong U_1 \intersect U_2 \cong \A^1_k \setminus
    \{0\}$.
But this isn't a closed subset of $X$, hence it isn't a
closed set of $U_1 \times_k U_2$, hence it isn't closed in
$X \times_k X$. \qed

\section{10.1.E}
Apply 9.4.E and 10.1.10

\section{10.1.G}
Suppose $\pi: X \to Y$ is quasiseparated. Let $\Spec A \subseteq Y$
and let $U, V \subseteq \pi^{-1}(\Spec A)$ be affine. Now, $U \intersect V \cong \Delta \intersect (U \times_Y V)$. Now,
$\Delta$ is quasicompact, so $U \intersect V \to U \times_Y V$ is
quasicompact (9.4.B(a)). But $U \times_Y V \cong U \times_{\Spec A} V$ is affine (where the
isomorphism follows from a corollary of 1.3.R), hence quasicompact, thus so is
$U \intersect V$.

Now, suppose $\pi: X \to Y$ is such that the inverse image of an affine
open is quasiseparated. We may reduce to the case where $Y$
is affine and $X$ is quasiseparated. Indeed,
quasicompactness is affine local. Cover $X$ with affines
$U_i$. Then we may pull back $U_i \times_Y U_j \to X \times_Y X$ along
$\Delta$. By 10.1.6, we see that this pullback is just
$U_i \intersect U_j$. But this is quasicompact since the we assumed
$X$ is quasiseparated. Hence we have shown that the pullback
of an affine open cover is quasicompact, hence $\Delta$ is
quasicompact.

\section{10.1.H}
The hint says it all. Separated $\implies$ diagonal closed
$\implies$ diagonal affine $\implies$ diagonal
quasicompact. \qed

\section{10.1.I}
The only if case follows by preservation of quasiseparatedness under base
change. For the other direction, let $\rho: X\times_YX \to Y$ be the natural map.
Suppose $U_i$ is an open cover of $Y$. Then
we have a cover $\rho^{-1}(U_i) \cong \pi^{-1}(U_i) \times_{U_i}
    \pi^{-1}(U_i)$ of $X \times_Y X$. But
$U_i \to \pi^{-1}(U_i) \times_{U_i} \pi^{-1}(U_i)$ is quasicompact by hypothesis, hence by 7.3.C(a)
$X \to X\times_YX$ is also quasicompact. \qed

\section{10.1.J}
We remark that the statement assumes that $A$ is a ring,
\textbf{not} a scheme. Thus one direction follows from preservation
of separatedness under composition. For the other direction, note that the
diagonal of any morphism is itself separated, since it is a locally closed
embedding (10.1.B). Thus, the hypothesis of 10.1.19 holds for separatedness.

\section{10.1.K}
\begin{enumerate}[a.]
    \item Locally closed embeddings follows from 10.1.19(i). Separatedness follows from a
          similar argument as 10.1.J. A locally closed embedding is locally of finite
          type (8.1.L), hence the diagonal is locally of finite type.
    \item By 7.3.B(b), $\rho$ is quasiseparated, hence 10.1.19(iii)
          applies.
    \item Recall that locally closed embeddings are separated, hence quasiseparated.
          Hence the diagonal $\Delta_{\rho}$ is quasiseparated. Thus cancellation
          holds for quasiseparated morphisms. \qed
\end{enumerate}
\section{10.1.L}
Separatedness follows from 10.1.J. Quasicompactness follows by 10.1.19(iii).
Locally of finite type follows from 10.1.K(a) and quasicompactness. \qed

\section{10.1.M}
To show that $\sigma$ is locally closed we apply 10.1.19(i) with
$\tau=\id$ and $\pi=\sigma$. If $\mu$ is
separated then 10.1.19(ii) applies in the same manner. \qed

\section{10.1.N}
We note first that $p_2: X \times_Y Y^{\text{red}} \to Y^{\text{red}}$ is in P since P is stable under
pullback. Next we note that $p_1: X \times_Y Y^{\text{red}} \to X$ is a closed embedding (hence
separated) for the same reason. Since $X^{\text{red}} \to X$ is a closed
embedding, cancellation holds, so $\phi: X^{\text{red}} \to X \times_Y
    Y^{\text{red}}$ is a closed embedding.
Hence $\pi^{\text{red}}=p_2 \after \phi$ is in P. \qed

\section{10.1.O}
Being universally injective is closed under composition and base change. Also,
monomorphisms are universally injective (we prove this in 9.5.24). The diagonal
is always locally closed, hence universally injective. Hence cancellation
applies, and $\pi$ is universally injective. \qed

\section{10.1.P}
Lemma: Suppose $k \to K$ is a field extension. Then
$\operatorname{Aut}_k(K)$ is in one to one correspondence with the maximal ideals
of $K \otimes_k K$. Indeed, the morphism $\Delta_{K/k}: \Spec K \to \Spec K \times_k \Spec K$ factors
through a morphism $\Spec E \to \Spec K \times_k \Spec K$ for some field
$E$. Hence the identity morphism factors as
$\Spec K \to \Spec E \to \Spec K \times_k \Spec K \to \Spec K$, where the last morphism is the projection onto the
first factor. This yields an isomorphism $E \cong K$. Suppose now we
are given an automorphism $\phi: K \to K$. Then we may form the ideal
generated by $x \otimes 1 - 1 \otimes \phi(x)$. We check that this is a maximal ideal.
Given a pure tensor $a \otimes b$ we see that this is equal to
$1 \otimes (\phi(a)b)$ in the quotient. This process clearly yeilds a field. It
is easily checked that these two processes are inverses.

\begin{enumerate}[a.]
    \item Suppose $\Delta_{X/Y}$ is surjective. Then if $K$ is a
          point of $X$ which lies over a point $k$ of
          $Y$ then $\Delta_{K/k}$ is a single point. By the
          above lemma,
    \item As maps of topological spaces, the image of the diagonal is a closed subset of
          the image. But the image is the entire space by (a), hence the diagonal is
          closed.
    \item c
\end{enumerate}

\section{10.2.A}
Lemma: \[\begin{tikzcd}
        Z & Y           \\
        X & {Y\times Y}
        \arrow["{(\operatorname{id}, \operatorname{id})}", from=1-2, to=2-2]
        \arrow[from=1-1, to=2-1]
        \arrow["{(f, f')}"', from=2-1, to=2-2]
        \arrow[from=1-1, to=1-2]
    \end{tikzcd}\] is a pullback iff	\[\begin{tikzcd}
        Z & X           \\
        X & {Y\times X}
        \arrow["{(f', \operatorname{id})}", from=1-2, to=2-2]
        \arrow[from=1-1, to=2-1]
        \arrow["{(f, \operatorname{id})}"', from=2-1, to=2-2]
        \arrow[from=1-1, to=1-2]
    \end{tikzcd}\]
is a pullback. Let $\pi_1: Y \times X \to Y$ and $\pi_2: Y \times X \to X$ be the
projections. Suppose the top diagram is a pullback. Suppose we have maps
$g, g': W \to X$ making the bottom diagram commute. Then we see
$\pi_2 \after (f', \id) \after g' = \pi_2 \after (f, \id) \after g$ so that $g'=g$. Also,
$\pi_1 \after (f,\id) \after g = \pi_1 \after (f',\id) \after g:
    W \to Y$ so that $f \after g = f' \after g$. Denote this by
$h$. Then $\Delta_Y \after h=(h, h)$. But also
$(f, f') \after g = (h, h)$, hence we get a unique morphism $W \to Z$.
This morphism also satisfies the universal property of the bottom diagram. We
omit the other direction as it is similar.

We next prove a more general result than the problem asks for. Note first that
we are asked to prove that the slice category $\catname{Scheme}/S$ has all
equalizers. Supppose a category has all fiber products(/pullbacks) and a final
object. Then it has all equalizers. Indeed, the bottom diagram in the theorem
above realizes $Z \to X$ as the equalizer of
$f$ and $f'$. The first diagram shows that
it is locally closed, and that it is closed if $Y \to S$ is
separated, i.e. $\Delta_Y$ is closed. \qed

\section{10.2.C}
The two inclusions $\Spec k[x] \to X$ will suffice. The domain of agreement
for these is the line without an origin, which is an open subscheme. \qed

\section{10.2.D}
Reducedness is necessary. Consider the given maps of schemes. On
$\D(x)$ we have $(k[x,y]/(y^2, xy))_x \cong
    k[x,y]_x/(y^2, xy) \cong
    k[x,y]_x/(y) \cong k[x]_x$ so that
$t \mapsto x$ under both maps. But the maps are clearly distinct maps
of $k$-algebras when considered globally.

Separatedness is necessary, indeed, see 10.2.C and note that removing the two
origins leaves a dense open. \qed

\section{10.2.E}
We will use different notation than in the problem. Suppose
$\pi':U \to Y$ is a representative of $\pi$. Then we
will say $\Gamma_{\pi'}: U \to U \times Y$ is the graph of $\pi'$. We
will define $\Gamma_\pi$ to be the scheme theoretic closure of
$\iota_U \after \Gamma_{\pi'}$ where $\iota_U: U\times Y \to X \times Y$ is the inclusion.

Suppose $\xi: U \to Y$ and $\xi': V \to Y$ are representatives
of $\pi$. 8.3.A and 8.3.9 together imply that the scheme
theoretic closure of $\iota_U \after \Gamma_\xi$ is uniquely determined by the set
theoretic closure. The same obviously holds true for $\Gamma_\xi$,
so the following diagram commutes: \[\begin{tikzcd}
          & {Z'}        & Z            \\
        U & {U\times Y} & {X \times Y}
        \arrow["{\iota_U}"', from=2-2, to=2-3]
        \arrow["{\Gamma_\xi}"', from=2-1, to=2-2]
        \arrow["{\Gamma_\pi}", from=1-3, to=2-3]
        \arrow[from=1-2, to=1-3]
        \arrow[from=1-2, to=2-2]
        \arrow[from=2-1, to=1-2]
    \end{tikzcd}\] In addition, the
square is a pullback. Thus $\Gamma_\pi$ is determined by
$\Gamma_\xi$, not just $\iota_U \after \Gamma_\xi$. In particular, if
$U$ is the domain of definition of $\pi$,
then $\Gamma_\pi$ is determined by $\Gamma_\xi: U \to U \times Y$. Thus,
without loss of generality, we may replace $X$ with
$U$ and may assume that $V \subseteq U$. Now, since
$V$ is dense, $V^\text{C}=U$. But then the same
holds scheme theoretically, so we are done. \qed

\section{10.2.F}
Let $\phi: \A_k^1 \rationalto \P_k^1$ be the given map. Suppose for the sake of
contradiction that we had a morphism $\phi': \A_k^1 \to \P_k^1$ extending
$\phi$ over $(0, 0)$. Consider
$X=\Spec k[x, y]/(y-1) \subset \P_k^1$. Let $\Spec k[x, y]_f \subseteq \phi^{-1}(X)$ where
$f \notin (x, y)$. Now, we know that $k[t] \to (k[x, y]_f)_x$ must be
given by $t \mapsto y/x$. Thus there must exist some
$g \in k[x, y]_f$ such that $g$ restricts to
$y/x$ in $(k[x, y]_f)_x$. But by definition of
localization, we must have $xg-y=0$. We know
$g=g'/f^n$ for some $g' \in k[x, y]$. Since
$f \notin (x, y)$ and $k[x,y]$ is an integral domain, we
know $xg'/f^n \neq y$. But this condradicts the existence of
$\phi'$. Hence $\phi'$ cannot exist.

Next we wish to study the graph of $\phi$. Let
$U$ be the domain of $\Gamma_\phi$. Consider the
affine open subset of $\A_k^2 \times \P^1_k$ given by $\Spec k[x,y] \times \Spec
    k[t]$
where $\Spec k[t] = \D(u) \subset \Proj
    k[u, v]$. Pulling back $\Gamma_\phi$ along the
inclusion of this open subset, we obtain a closed subscheme, say
$Z \to \Spec k[x,y,t]$. We know that on $\Spec k[x,y,t]_x \subset \Spec
    k[x,y,t]$ this subscheme
corresponds to the ideal $(y/x-t)$. But this ideal pulls back to
$(y-tx)$ in $\Spec k[x,y,t]$. Hence
$Z=\V(y-tx)$. But this is exactly the equation satisfied by the
blowup of 9.3.F on $\D(u)$. \qed

\section{10.2.G}
Cover $X$ with affine opens $\Spec A_i$ such
that on each open, $D=\V(t_i)$ is the vanishing scheme of a single
non zero-divisor. If we have a morphism $X \setminus D \to Y$ that extends to
$X \to Y$ then we necessarily have morphisms
$\Spec A_i \setminus \V(t_i) \to Y$ that extends to morphisms $\Spec A_i$ (and
satisfies the cocyle condition). Suppose there is a no such extension on the
affine cover. Then by contrapositive we have no extension globally. Suppose we
have a unique extension on the affines. Then, if the cocycle condition holds,
we necessarily have a unique global extension (by gluing). Hence we have
reduced to the affine case.

Suppose we have $\Spec A \setminus \D(t) \to Y$. This is the same as a morphism
$\Spec A_t \to Y$. Suppose $\zeta: \Spec A \to Y$ and
$\zeta': \Spec A \to Y$ both restrict to $\Spec A_t \to Y$. Then we know
they agree on $\Spec A_t \to \Spec A$. By 10.2.A, the largest subscheme on
which they agree must be closed, hence it must be a superscheme of the closure
of $\Spec A_t \to \Spec A$. Since $t$ is a non
zero-divisor, we know that $A \to A_t$ must be injective. But the
closure is given by quotient by the kernel on affines, hence
$\id_A$ is the closure of $\Spec A_t \to \Spec A$. \qed

\section{10.3.A}
Consider $\A^2_\C \to \A^1_\C$ given by $(x,y) \mapsto y$. The
nontrivial closed sets of $\A^1_\C$ are finite sets of points,
but the image of $\V(x^2+y^2-1)$ is clearly both nontrivial and
non-finite.

\section{10.3.B}
Suppose $S \to Y$ is the image of $\pi$.
Suppose $Z' \to Z$ is arbitrary. Then we may form the following
diagram: \[\begin{tikzcd}
         &   &      & {S'}     \\{X'} &&& {Y'} \\
         &   & {Z'} & {}   & S \\
         & X &      &      & Y \\
         &   &      & Z
        \arrow["\pi"{pos=0.7}, from=4-2, to=4-5]
        \arrow["\tau"', from=4-2, to=5-4]
        \arrow["\rho", from=4-5, to=5-4]
        \arrow[from=3-5, to=4-5]
        \arrow[from=1-4, to=3-5]
        \arrow[from=2-4, to=4-5]
        \arrow[from=1-4, to=2-4]
        \arrow[from=2-1, to=4-2]
        \arrow[from=2-1, to=2-4]
        \arrow[from=2-1, to=3-3]
        \arrow[from=2-4, to=3-3]
        \arrow[from=3-3, to=5-4]
    \end{tikzcd}\] Under the hypotheses of the theorem, we know
by 10.3.4(e) that $\pi$ is proper, hence quasicompact and
closed. Hence $\pi(X)=S$. Then any closed subset of
$S$ comes from a closed subset of $X$.
But under $\tau$ the image is closed in
$Z$. Hence $S \to Y \to Z$ is closed. It is also
universally closed, since $X' \to Z'$ is proper and
$Y' \to Z'$ is separated and finite type, causing the pullback
diagram to be identical to the original diagram. Thus, since
$S \to Y$ is proper and $\rho$ is separated and
finite type, $S \to Z$ is separated,finite type, and universally
closed, hence proper. \qed

\end{document}
\usepackage[utf8]{inputenc}

\usepackage{mathtools}
\usepackage{amsthm}
\usepackage{amssymb}
\usepackage{dsfont}
\usepackage{array}   % for \newcolumntype macro
\usepackage{calligra}
% \usepackage{tikz-cd}
\usepackage{mathrsfs}
\usepackage{quiver}
\usepackage[shortlabels]{enumitem}

\newcolumntype{L}{>{$}l<{$}}
% math-mode version of "l" column type

\newcommand{\Rho}{\mathrm{P}}

\newcommand{\iso}{\simeq}
\newcommand{\after}{\circ}
\newcommand{\catname}[1]{\mathbf{#1}}
\newcommand{\intersect}{\cap}
\newcommand{\union}{\cup}
\newcommand{\Intersect}{\bigcap}
\newcommand{\Union}{\bigcup}
\newcommand{\wa}[1]{\langle #1 \rangle}
\newcommand{\bb}[1]{\mathbb{#1}}
\newcommand{\A}{\mathbb{A}}
\renewcommand{\P}{\mathbb{P}}
\newcommand{\Z}{\mathbb{Z}}
\newcommand{\Q}{\mathbb{Q}}
\newcommand{\R}{\mathbb{R}}
\newcommand{\C}{\mathbb{C}}
\newcommand{\F}{\mathbb{F}}
\newcommand{\p}{\mathfrak{p}}
\newcommand{\q}{\mathfrak{q}}
\newcommand{\ai}{\mathfrak{a}}
\newcommand{\bi}{\mathfrak{b}}
\newcommand{\m}{\mathfrak{m}}
\newcommand{\defeq}{\vcentcolon=}

\DeclareMathOperator{\id}{id}
\DeclareMathOperator{\Hom}{\mathscr{H}\text{\kern -3pt
{\calligra\large om}}\,}
\DeclareMathOperator{\Fsh}{\mathscr{F}}
\DeclareMathOperator{\Gsh}{\mathscr{G}}
\DeclareMathOperator{\Hsh}{\mathscr{H}}
\DeclareMathOperator{\Osh}{\mathscr{O}}
\DeclareMathOperator{\Ish}{\mathscr{I}}
\DeclareMathOperator{\Jsh}{\mathscr{J}}
% \DeclareMathOperator{\ker}{ker}
\DeclareMathOperator{\coker}{coker}
\DeclareMathOperator{\im}{im}
\DeclareMathOperator{\Spec}{Spec}
\DeclareMathOperator{\Proj}{Proj}
\DeclareMathOperator{\V}{V}
\DeclareMathOperator{\D}{D}

\newcommand*{\rationalto}[1][]{\mathbin{\tikz [baseline=-0.25ex,-latex, dashed,->,densely dashed,#1] \draw [#1] (0pt,0.5ex) -- (1.3em,0.5ex);}}%

% \usepackage{tikz, environ, etoolbox}
% \usetikzlibrary{cd, external}
% \tikzexternalize
% % activate the following such that you can check the macro expansion in
% % *-figure0.md5 manually
% %\tikzset{external/up to date check=diff}

% \def\temp{&} \catcode`&=\active \let&=\temp

% \newcommand{\mytikzcdcontext}[2]{
%     \begin{tikzpicture}[baseline=(maintikzcdnode.base)]
%         \node (maintikzcdnode) [inner sep=0, outer sep=0] {\begin{tikzcd}[#2]
%                 #1
%             \end{tikzcd}};
%     \end{tikzpicture}}

% \NewEnviron{etikzcd}[1][]{%
%     % In the following, we need \BODY to expanded before \mytikzcdcontext
%     % such that the md5 function gets the tikzcd content with \BODY expanded.
%     % Howerver, expand it only once, because the \tikz-macros aren't
%     % defined at this point yet. The same thing holds for the arguments to
%     % the tikzcd-environment.
%     \def\myargs{#1}%
%     \edef\mydiagram{\noexpand\mytikzcdcontext{\expandonce\BODY}{\expandonce\myargs}}%
%     \mydiagram%
% }

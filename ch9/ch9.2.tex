\documentclass{article}
\usepackage[utf8]{inputenc}

\usepackage{mathtools}
\usepackage{amsthm}
\usepackage{amssymb}
\usepackage{dsfont}
\usepackage{array}   % for \newcolumntype macro
\usepackage{calligra}
\usepackage{tikz-cd}
\usepackage{mathrsfs}
\usepackage{quiver}
\usepackage[shortlabels]{enumitem}

\newcolumntype{L}{>{$}l<{$}}
% math-mode version of "l" column type

\newcommand{\Rho}{\mathrm{P}}

\newcommand{\iso}{\simeq}
\newcommand{\after}{\circ}
\newcommand{\catname}[1]{\mathbf{#1}}
\newcommand{\intersect}{\cap}
\newcommand{\union}{\cup}
\newcommand{\Intersect}{\bigcap}
\newcommand{\Union}{\bigcup}
\newcommand{\wa}[1]{\langle #1 \rangle}
\newcommand{\bb}[1]{\mathbb{#1}}
\newcommand{\A}{\mathbb{A}}
\newcommand{\Z}{\mathbb{Z}}
\newcommand{\Q}{\mathbb{Q}}
\newcommand{\R}{\mathbb{R}}
\newcommand{\C}{\mathbb{C}}
\newcommand{\F}{\mathbb{F}}
\newcommand{\p}{\mathfrak{p}}
\newcommand{\q}{\mathfrak{q}}
\newcommand{\ai}{\mathfrak{a}}
\newcommand{\bi}{\mathfrak{b}}
\newcommand{\m}{\mathfrak{m}}
\newcommand{\defeq}{\vcentcolon=}

\DeclareMathOperator{\id}{id}
\DeclareMathOperator{\Hom}{\mathscr{H}\text{\kern -3pt
{\calligra\large om}}\,}
\DeclareMathOperator{\Fsh}{\mathscr{F}}
\DeclareMathOperator{\Gsh}{\mathscr{G}}
\DeclareMathOperator{\Hsh}{\mathscr{H}}
\DeclareMathOperator{\Osh}{\mathscr{O}}
% \DeclareMathOperator{\ker}{ker}
\DeclareMathOperator{\coker}{coker}
\DeclareMathOperator{\im}{im}
\DeclareMathOperator{\Spec}{Spec}
\DeclareMathOperator{\Proj}{Proj}
\DeclareMathOperator{\V}{V}
\DeclareMathOperator{\D}{D}

\newcommand*{\rationalto}[1][]{\mathbin{\tikz [baseline=-0.25ex,-latex, dashed,->,densely dashed,#1] \draw [#1] (0pt,0.5ex) -- (1.3em,0.5ex);}}%


\title{Chapter 9.2 Problems}
\author{Josiah Bills}
\date{March 2021}

\begin{document}

\maketitle

\section{9.2.A}
Trivial

\section{9.2.B}
We show that $A \otimes I \cong I^{\text{e}}$ as $A$-modules. The
morphism is given by $a \otimes i \mapsto a\phi(i)$. We note that
$a \otimes i = ai \otimes 1$. This map is obviously bijective. It is clear that the
tensored morphism $I^{\text{e}} \to A$ is injective. \qed

\section{9.2.C}
\begin{enumerate}[a.]
    \item We consider an affine open set $\Spec A \subseteq X$. By affineness, we have
          the the following commutative diagram: \[\begin{tikzcd}
                  A                             & {A/J}
                  \\{A/I} & {A/I \otimes_A A/J}
                  \arrow[from=1-1, to=2-1] \arrow[from=2-1, to=2-2] \arrow[from=1-1, to=1-2]
                  \arrow["\phi", from=1-2, to=2-2]
              \end{tikzcd}\] By 9.2.B we see
          that the tensor product is isomorphic to $(A/J)/(\phi(I))$. It is clear by
          examinination that the kernel of this map is $I+J$, so the
          tensor product is $A/(I+J)$, i.e. the closed embedding
          corresponding to the intersection of closed subschemes. \qed
    \item Consider the following diagram: \[\begin{tikzcd}
                  {Z_1\times_XZ_2}                       & {Z_2\times_VW} &
                  {Z_2}
                  \\{Z_1\times_UW} & W              & V
                  \\{Z_1}          & U              & X
                  \arrow["{\text{open}}", hook, from=2-3, to=3-3]
                  \arrow["{\text{closed}}", hook, from=1-3, to=2-3]
                  \arrow["{\text{open}}"', hook, from=3-2, to=3-3]
                  \arrow["{\text{closed}}"', hook, from=3-1, to=3-2]
                  \arrow[hook, from=2-2, to=3-2]
                  \arrow[hook, from=2-2, to=2-3]
                  \arrow[hook, from=1-2, to=1-3]
                  \arrow[hook, from=1-2, to=2-2]
                  \arrow[hook, from=2-1, to=3-1]
                  \arrow[hook, from=2-1, to=2-2]
                  \arrow[hook, from=1-1, to=2-1]
                  \arrow[hook, from=1-1, to=1-2]
                  \arrow[dashed, hook, from=1-1, to=2-2]
              \end{tikzcd}\] where the upper left is
          the pullback because it is the composition of two pullback squares in two
          different ways. The top left box is the pullback of two closed embeddings, the
          bottom right box is the pullback of two open embeddings. Hence the overall
          morphism is a locally closed embedding. \qed
    \item Apply induction to (b). \qed
\end{enumerate}

\section{9.2.D}
We first note that the pullback of a quasicompact morphism along a quasicompact
morphism is quasicompact (9.4.B(a)). This shows the latter property. Now, we
may assume that $A$ is affine, say $\Spec A$.
Let $\Spec B \subseteq X, \Spec C \subseteq Y$. It suffices to show that $A \to B \otimes_A C$ is
finite type. We first note that for any $A$-algebras
$B/I \otimes_A C \cong (B \otimes_A C)/J$ for some ideal $J$. Indeed, simply
tensor $I \to B \to B/I$ with $C$ and take kernels. Now
$A[X]/I \otimes_A A[Y]/J
    \cong (A[X] \otimes_A A[Y])/K \cong A[X, Y]/J$. \qed

\section{9.2.F}
Define $b/s \to \phi(b)/\phi(s)$. Suppose we have $f: A \to C$ and
$g: S^{-1}B \to C$ making the relevant diagram commute. Then we send
$a/\phi(s) \to f(a)*g(1/s)$. This is clearly unique. \qed

\section{9.2.G}
We first characterize monomorphisms. $f$ is mono iff the
following diagram is a pullback: \[\begin{tikzcd}
        W                                                                     \\
         & X & X                                                              \\
         & X & Y \arrow["f", from=2-3, to=3-3] \arrow["f"', from=3-2, to=3-3]
        \arrow["{\text{id}}"', from=2-2, to=3-2]
        \arrow["{\text{id}}", from=2-2, to=2-3] \arrow[curve={height=6pt}, from=1-1, to=3-2]
        \arrow[curve={height=-6pt}, from=1-1, to=2-3] \arrow[dashed, from=1-1, to=2-2]
    \end{tikzcd}\]

Now, it is clear that if $f: U \to X$ is pulled back along
$f$ then the result is $U$, and we are
done.

It is clear from 9.1.C(a) that the pullback of a closed subscheme along itself
is itself again, so we are done.

Now, let $f: \Spec S^{-1}A \to \Spec A$. We pull this back along the identity, getting
$\Spec S^{-1}A \otimes_A A$. But by 9.2.F this is just $\Spec S^{-1}A$, and
we are done. \qed

\section{9.2.H}
$\A^n_A \cong \Spec A[x_1, \dots, x_n] \cong \Spec A
    \otimes_{\Z} \Z[x_1, \dots, x_n]$. Now, cover $\P^n_A$ with affine opens,
so by the first part, the result holds. \qed

\section{9.2.I}
We note that equality of morphisms can be checked on an open cover. We then
cover $X$ with affine opens such that we have
$\Spec A \otimes_k \ell \to \Spec A$ as an open cover of $X \times_k \Spec \ell$. We do the
same with $Y$, reducing to the case $Y=\Spec B$.
Consider the following diagram: \[\begin{tikzcd}
        B                & {B\otimes_k\ell}           & B \\
        A                &
        {A\otimes_k\ell} & A \arrow[from=1-2, to=2-2]
        \arrow[hook, from=1-1, to=1-2] \arrow["\phi", hook, from=2-1, to=2-2] \arrow["\pi"', from=1-1, to=2-1]
        \arrow["\rho", from=1-3, to=2-3] \arrow[hook, from=2-3, to=2-2] \arrow[hook, from=1-3, to=1-2]
    \end{tikzcd}\] The horizontal morphisms
are injections. Let $\pi(x)=y$. Then clearly $\phi(x\otimes 1)=y\otimes 1$
so that $\rho(x)=y$. \qed

\section{9.2.J}
We repeat the above reduction to affine opens. Then  \[\begin{tikzcd}
        B & {B\otimes_k\ell} \\
        A &
        {A\otimes_k\ell} \arrow["\phi", two heads, from=1-2, to=2-2]
        \arrow[hook, from=1-1, to=1-2] \arrow[hook, from=2-1, to=2-2] \arrow["\pi"', from=1-1, to=2-1]
    \end{tikzcd}\] We
have surjective $\pi$ by assumption. Let
$y \in A$. Then there exists an $x$ such that
$\pi(x\otimes 1)=y\otimes 1$. Then by injectivity we have $\phi(x)=y$ and
we are done. \qed

\end{document}
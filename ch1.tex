\documentclass{article}
\usepackage[utf8]{inputenc}

\usepackage{mathtools}
\usepackage{amsthm}
\usepackage{amssymb}
\usepackage{dsfont}
\usepackage{array}   % for \newcolumntype macro
\usepackage{calligra}
\usepackage{tikz-cd}
\usepackage{mathrsfs}
\usepackage{quiver}
\usepackage[shortlabels]{enumitem}

\newcolumntype{L}{>{$}l<{$}}
% math-mode version of "l" column type

\newcommand{\Rho}{\mathrm{P}}

\newcommand{\iso}{\simeq}
\newcommand{\after}{\circ}
\newcommand{\catname}[1]{\mathbf{#1}}
\newcommand{\intersect}{\cap}
\newcommand{\union}{\cup}
\newcommand{\Intersect}{\bigcap}
\newcommand{\Union}{\bigcup}
\newcommand{\wa}[1]{\langle #1 \rangle}
\newcommand{\bb}[1]{\mathbb{#1}}
\newcommand{\A}{\mathbb{A}}
\newcommand{\Z}{\mathbb{Z}}
\newcommand{\Q}{\mathbb{Q}}
\newcommand{\R}{\mathbb{R}}
\newcommand{\C}{\mathbb{C}}
\newcommand{\F}{\mathbb{F}}
\newcommand{\p}{\mathfrak{p}}
\newcommand{\q}{\mathfrak{q}}
\newcommand{\ai}{\mathfrak{a}}
\newcommand{\bi}{\mathfrak{b}}
\newcommand{\m}{\mathfrak{m}}
\newcommand{\defeq}{\vcentcolon=}

\DeclareMathOperator{\id}{id}
\DeclareMathOperator{\Hom}{\mathscr{H}\text{\kern -3pt
{\calligra\large om}}\,}
\DeclareMathOperator{\Fsh}{\mathscr{F}}
\DeclareMathOperator{\Gsh}{\mathscr{G}}
\DeclareMathOperator{\Hsh}{\mathscr{H}}
\DeclareMathOperator{\Osh}{\mathscr{O}}
% \DeclareMathOperator{\ker}{ker}
\DeclareMathOperator{\coker}{coker}
\DeclareMathOperator{\im}{im}
\DeclareMathOperator{\Spec}{Spec}
\DeclareMathOperator{\Proj}{Proj}
\DeclareMathOperator{\V}{V}
\DeclareMathOperator{\D}{D}

\newcommand*{\rationalto}[1][]{\mathbin{\tikz [baseline=-0.25ex,-latex, dashed,->,densely dashed,#1] \draw [#1] (0pt,0.5ex) -- (1.3em,0.5ex);}}%


\title{Chapter 1 Problems}
\author{Josiah Bills}
\date{July 2020}

\begin{document}

\maketitle

\section{Transitivity of Localization}
\begin{theorem}
    Let $a/b \sim c/d \sim e/f \in S^{-1}A$. Then $a/b \sim e/f$.
\end{theorem}
\begin{proof}
    Let $a/b \sim c/d \sim e/f$. Then there exist $s_1, s_2$ such that
    $s_1(ad-bc)=0$ and $s_2(cf-ed)=0$. Thus $s_2(bcf-edb)=s_2(s_1daf-s_1bed)=s_2ds_1(af-eb)$.
    Since $s_2ds_1 \in S$ the result follows.
\end{proof}

\section{1.3.C}
\begin{theorem}
    $A \to S^{-1}A$ is injective iff $S$ contains no
    zerodivisors.
\end{theorem}
\begin{proof}
    \hfill
    \begin{itemize}
        \item[$\implies$] For the sake of contrapositive, suppose $ab=0$,
            $0 \neq a \in S$ and $b \neq 0$. Then $a(b-0)=0$
            so that $b/1=0/1$ so that $A \to S^{-1}A$ is not injective.
        \item[$\impliedby$] Suppose $x/1=y/1$. Then $s(x-y)=0$. Since
            $s$ is not a zero divisor, $x=y$.
    \end{itemize}
\end{proof}

\section{1.3.D}
\begin{theorem}
    $A \to S^{-1}A$ is initial amongst $A$-algebras where
    every element of $S$ is invertible.
\end{theorem}
\begin{proof}
    Let $\phi: A \to B$ be such that $\phi(S)$ contains only
    invertible elements. Define $f : S^{-1}A \to B$ by $f(a/b)=\phi(a)\phi(b)^{-1}$.
    The map is well defined, and $f(a/1)=\phi(a)$ as desired.
\end{proof}

\section{1.3.R}
\begin{theorem}
    Suppose we have morphisms $X_1 \to Y, X_2 \to Y, Y \to Z$ in some category. Then the
    fiber product induces a morphism $X_1 \times_Y X_2 \to X_1 \times_Z X_2$.
\end{theorem}
\begin{proof}
    \[
        \begin{tikzcd}
            {X_1\times_YX_2}                \\
             & {X_1\times_ZX_2} & {X_2}     \\
             & {X_1}            & Y         \\
             &                  &       & Z
            \arrow[from=3-2, to=3-3]
            \arrow[from=2-3, to=3-3]
            \arrow[from=3-3, to=4-4]
            \arrow[from=2-2, to=3-2]
            \arrow[from=2-2, to=2-3]
            \arrow[from=1-1, to=3-2]
            \arrow[from=1-1, to=2-3]
            \arrow[dashed, from=1-1, to=2-2]
        \end{tikzcd}
    \]

    We prove a further fact: Assume $Y \to Z$ is mono. Then
    $X_1\times_YX_2 \to X_1\times_ZX_2$ is iso. Indeed consider, $W \to X_1 \to Y \to Z = W \to X_2 \to Y \to Z$ so that,
    by monomorphism, $W \to X_1 \to Y = W \to X_2 \to Y$. But this means that there must be a
    unique map $W \to X_1 \times_Y X_2$, which is exactly the condition for
    $X_1 \times_Y X_2$ to be the product over $Z$.
\end{proof}

\section{1.3.S}
\begin{theorem}
    Suppose we have $X_1,X_2 \to Y$ and $Y \to Z$. Show that
    the map constructed in 1.3.R is fibered over the diagonal
    $\Delta_{Y/Z}$. In other words, the following diagram is cartesian:

    \[\begin{tikzcd}
            {X_1 \times_YX_2} & {X_1 \times_ZX_2} \\
            Y                 & {Y\times_ZY}
            \arrow[from=1-1, to=2-1]
            \arrow[from=1-1, to=1-2]
            \arrow[from=1-2, to=2-2]
            \arrow[from=2-1, to=2-2]
        \end{tikzcd}\]
\end{theorem}
\begin{proof}
    The diagonal morphism is defined by the following pullback.
    \[
        \begin{tikzcd}
            Y \ar[rdd, bend right] \ar[rrd, bend left] \ar[rd, "\exists!\Delta", dotted] &                            &          \\
                                                                                         & Y \times_Z Y \ar[d] \ar[r] & Y \ar[d] \\
                                                                                         & Y \ar[r]                   & Z
        \end{tikzcd}
    \] \[
        \begin{tikzcd}
            W \arrow[rrrddd, bend right=60] \arrow[rrdd, bend left=49] \arrow[rd, "\exists!", dotted] &                                                                                                &                                      &                        &                       \\
                                                                                                      & X_1 \times_Y X_2 \arrow[rdd, bend right] \arrow[rrd, bend left] \arrow[rd, "\exists!", dotted] &                                      &                        &                       \\
                                                                                                      &                                                                                                & X_1 \times_Z X_2 \arrow[r] \arrow[d] & X_2 \arrow[d]          &                       \\
                                                                                                      &                                                                                                & X_1 \arrow[r]                        & Y \arrow[rd] \arrow[r] & Y\times_Z Y \arrow[d] \\
                                                                                                      &                                                                                                &                                      &                        & Z
        \end{tikzcd}
    \]

    The morphism from $W$ exists by virtue of the projections of
    $X_1 \times_Z X_2$ to the components, as well as the commutativity of all
    the relvant diagrams.
\end{proof}

\end{document}

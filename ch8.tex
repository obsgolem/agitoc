\documentclass{article}
\usepackage[utf8]{inputenc}

\usepackage{mathtools}
\usepackage{amsthm}
\usepackage{amssymb}
\usepackage{dsfont}
\usepackage{array}   % for \newcolumntype macro
\usepackage{calligra}
\usepackage{tikz-cd}
\usepackage{mathrsfs}

\newcolumntype{L}{>{$}l<{$}} % math-mode version of "l" column type

\newcommand{\Rho}{\mathrm{P}}

\newcommand{\iso}{\simeq}
\newcommand{\after}{\circ}
\newcommand{\catname}[1]{\mathbf{#1}}
\newcommand{\intersect}{\cap}
\newcommand{\union}{\cup}
\newcommand{\Intersect}{\bigcap}
\newcommand{\Union}{\bigcup}
\newcommand{\wa}[1]{\langle #1 \rangle}
\newcommand{\ds}[1]{\mathds{#1}}
\newcommand{\Z}{\mathds{Z}}
\newcommand{\Q}{\mathds{Q}}
\newcommand{\R}{\mathds{R}}
\newcommand{\C}{\mathds{C}}
\newcommand{\p}{\mathfrak{p}}
\newcommand{\q}{\mathfrak{q}}
\newcommand{\ai}{\mathfrak{a}}
\newcommand{\bi}{\mathfrak{b}}
\newcommand{\m}{\mathfrak{m}}
\newcommand{\defeq}{\vcentcolon=}

\DeclareMathOperator{\Hom}{\mathscr{H}\text{\kern -3pt {\calligra\large om}}\,}
\DeclareMathOperator{\Fsh}{\mathscr{F}}
\DeclareMathOperator{\Gsh}{\mathscr{G}}
\DeclareMathOperator{\Hsh}{\mathscr{H}}
\DeclareMathOperator{\Osh}{\mathscr{O}}
% \DeclareMathOperator{\ker}{ker}
\DeclareMathOperator{\coker}{coker}
\DeclareMathOperator{\im}{im}
\DeclareMathOperator{\Spec}{Spec}
\DeclareMathOperator{\V}{V}
\DeclareMathOperator{\D}{D}


\title{Chapter 8 Problems}
\author{Josiah Bills}
\date{March 2021}

\begin{document}

\maketitle

\section{8.1.A}
Choose an affine cover. Injectivity clearly holds on this cover, and so holds
globally. The complement of the image on an element of the affine cover is
open, so complement of the overall image is open. Hence the image is closed.
\qed

\section{8.1.B}
Quotients are clearly finite over the base ring, and closed embeddings are
affine by definition.

\section{8.1.C}
Composition of affines morphisms is affine. Composition of surjective morphisms
of rings is surjective. \qed

\section{8.1.D}
Consider $A \to B$ surjective. Then $A_f \to B_f$ is also
surjective. The latter part of the affine communication lemma is just gluing
schemes along affine opens.

\section{8.1.E}
Apply 8.1.D. \qed

\section{8.1.F}
Let $f: Y \to X$ be a closed subscheme. There are only two closed sets
in $X$, $X$ itself and
${\m}$. For the latter, we must have a surjective morphism
$k[x]_{(x)} \to l$ where $l$ is a field. The kernel of
this must be the closed point $\m$, which is not
$mathfrak{I}(X)$. Suppose we are given $f: X \to X$ an
automorphism and closed embedding. Then $f^{-1}(\D_x)=f^{-1}({\eta})$, giving us
$f^{\sharp}: k(x) \to B$. If the condition on ideals holds then the kernel is
$k(x)$, meaning that $B$ is the empty set,
contradicting $f$ being an automorphism. \qed

\section{8.1.G}
Let $\phi: B \to B_f$ be the natural map. Then there is an induced map
$\phi': I(B)_f \to I(B_f) x/f \mapsto \phi(x)/f$. The result follows by exactness of localization.

\section{8.1.H}
It is easy to see that $\operatorname{I}(B)$ defines a sheaf on a base. The
restriction maps are the ristriction to the ideal of the ring restriction maps.
The hypothesis ensures this is well defined. It remains to conclude that
$\Osh / \mathfrak{I}$, regarded as the cokernel in the category of presheaves,
is in fact a sheaf. Consider $\to \Spec B_f / \operatorname{I}(B_f) \to \Spec B/\operatorname{I}(B)$. Regarding this as a sheaf
on $\Spec B$, we see that regular scheme gluing and identity
applies. We used the hypothesis in order to ensure open inclusions are well
defined. \qed

\end{document}
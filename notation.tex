\documentclass{article}
\usepackage[utf8]{inputenc}

\usepackage{mathtools}
\usepackage{amsthm}
\usepackage{amssymb}
\usepackage{dsfont}
\usepackage{array}   % for \newcolumntype macro
\usepackage{calligra}
\usepackage{tikz-cd}
\usepackage{mathrsfs}

\newcolumntype{L}{>{$}l<{$}} % math-mode version of "l" column type

\newcommand{\Rho}{\mathrm{P}}

\newcommand{\iso}{\simeq}
\newcommand{\after}{\circ}
\newcommand{\catname}[1]{\mathbf{#1}}
\newcommand{\intersect}{\cap}
\newcommand{\union}{\cup}
\newcommand{\Intersect}{\bigcap}
\newcommand{\Union}{\bigcup}
\newcommand{\wa}[1]{\langle #1 \rangle}
\newcommand{\ds}[1]{\mathds{#1}}
\newcommand{\Z}{\mathds{Z}}
\newcommand{\Q}{\mathds{Q}}
\newcommand{\R}{\mathds{R}}
\newcommand{\C}{\mathds{C}}
\newcommand{\p}{\mathfrak{p}}
\newcommand{\q}{\mathfrak{q}}
\newcommand{\ai}{\mathfrak{a}}
\newcommand{\bi}{\mathfrak{b}}
\newcommand{\m}{\mathfrak{m}}
\newcommand{\defeq}{\vcentcolon=}

\DeclareMathOperator{\Hom}{\mathscr{H}\text{\kern -3pt {\calligra\large om}}\,}
\DeclareMathOperator{\Fsh}{\mathscr{F}}
\DeclareMathOperator{\Gsh}{\mathscr{G}}
\DeclareMathOperator{\Hsh}{\mathscr{H}}
\DeclareMathOperator{\Osh}{\mathscr{O}}
% \DeclareMathOperator{\ker}{ker}
\DeclareMathOperator{\coker}{coker}
\DeclareMathOperator{\im}{im}
\DeclareMathOperator{\Spec}{Spec}
\DeclareMathOperator{\V}{V}
\DeclareMathOperator{\D}{D}


\title{Problem Solution Notation}
\author{Josiah Bills}
\date{March 2021}

\begin{document}

\maketitle
\begin{itemize}
    \item $A[X]$ with a capital letter indicates a set of variables, e.g.
          $X=\{x_1,\dots, x_n\}$.
    \item $\pi$, unless otherwise defined, refers to some kind of
          projection, usually the projection onto the quotient. $\pi_1$
          with no definition is projection onto the first component of the product.
    \item $\p$ is always a prime ideal unless otherwise noted.
    \item $I, J$ are always ideals unless otherwise noted.
    \item $(S)$ refers to the ideal generated by $S$
          unless it is being used as grouping.
    \item $I^{\text{e}}_{\phi}=(\phi(I))$. $\phi$ may be omitted when it can be
          inferred from context.
    \item $\overline{X}$ where $X$ is a set denotes the
          complement of $X$. $X^{\text{C}}$ denotes the
          closure of $X$ in its containing space.
\end{itemize}

\end{document}
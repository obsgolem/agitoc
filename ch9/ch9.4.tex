\documentclass{article}
\usepackage[utf8]{inputenc}

\usepackage{mathtools}
\usepackage{amsthm}
\usepackage{amssymb}
\usepackage{dsfont}
\usepackage{array}   % for \newcolumntype macro
\usepackage{calligra}
\usepackage{tikz-cd}
\usepackage{mathrsfs}

\newcolumntype{L}{>{$}l<{$}} % math-mode version of "l" column type

\newcommand{\Rho}{\mathrm{P}}

\newcommand{\iso}{\simeq}
\newcommand{\after}{\circ}
\newcommand{\catname}[1]{\mathbf{#1}}
\newcommand{\intersect}{\cap}
\newcommand{\union}{\cup}
\newcommand{\Intersect}{\bigcap}
\newcommand{\Union}{\bigcup}
\newcommand{\wa}[1]{\langle #1 \rangle}
\newcommand{\ds}[1]{\mathds{#1}}
\newcommand{\Z}{\mathds{Z}}
\newcommand{\Q}{\mathds{Q}}
\newcommand{\R}{\mathds{R}}
\newcommand{\C}{\mathds{C}}
\newcommand{\p}{\mathfrak{p}}
\newcommand{\q}{\mathfrak{q}}
\newcommand{\ai}{\mathfrak{a}}
\newcommand{\bi}{\mathfrak{b}}
\newcommand{\m}{\mathfrak{m}}
\newcommand{\defeq}{\vcentcolon=}

\DeclareMathOperator{\Hom}{\mathscr{H}\text{\kern -3pt {\calligra\large om}}\,}
\DeclareMathOperator{\Fsh}{\mathscr{F}}
\DeclareMathOperator{\Gsh}{\mathscr{G}}
\DeclareMathOperator{\Hsh}{\mathscr{H}}
\DeclareMathOperator{\Osh}{\mathscr{O}}
% \DeclareMathOperator{\ker}{ker}
\DeclareMathOperator{\coker}{coker}
\DeclareMathOperator{\im}{im}
\DeclareMathOperator{\Spec}{Spec}
\DeclareMathOperator{\V}{V}
\DeclareMathOperator{\D}{D}


\title{Chapter 9.4 Problems}
\author{Josiah Bills}
\date{March 2021}

\begin{document}

\maketitle

\section{9.4.A}
It is clear that being a locally principally closed subscheme is affine-local
on the target. Additionally, it has the form of part (b) of the lemma, where Q
is the property "$A \to B_i$ is surjective and realizes
$B_i$ as $A/(s)$ for some
$s \in A$". Thus the property may be checked soley for affines.
The rest of the proof follows the same form as 9.4.B(g). \qed

\section{9.4.B}
\subsection{Lemma}
\begin{enumerate}[a.]
      \item Suppose P is a property of morphisms that is affine local on the target. Then
            we may reduce to the case of affine times arbitrary over affine.
      \item Let $f: X \to Z$ be a morphism of schemes. Suppose P is an affine
            local (on the target) property of $f$ of the form "For every
            affine $\Spec A \subset Z$ there exists an affine open cover
            $\{\Spec B_i\}_{i \in I}$ of $f^{-1}(\Spec A)$ such that the morphisms
            $A \to B_i$ all have propery Q". Then we may reduce to the case of
            affine times affine over affine.
\end{enumerate}

Proof:
\begin{enumerate}[a.]
      \item Consider the following diagram: \[\begin{tikzcd}
                        {X\times_ZY} & X                                                   \\
                        Y            & Z \arrow[from=1-1, to=2-1] \arrow[from=1-1, to=1-2]
                        \arrow["g"', from=2-1, to=2-2] \arrow["f", from=1-2, to=2-2]
                  \end{tikzcd}\] Cover
            $Z$ with affines $\{\Spec A_i\}_{i \in I}$. Then we may cover
            $Y$ with affines $\{\Spec B_{ij}\}_{i \in I, j \in J}$ such that
            $\Spec B_{ij} \subseteq
                  g^{-1}(\Spec A_i)$. By affine locality, we may check P on the cover
            $\{\Spec B_{ij}\}_{i \in I, j \in J}$, so in particular if P holds for $Y, Z$
            affine then P holds for $Y, Z$ arbitrary.
      \item We first apply part (a) of the lemma. Let $Y=\Spec C$ and
            $Z=\Spec A$. Now, choose an affine cover $\{\Spec B_i\}_{i \in I}$ of
            $X$ such that $A \to B_i$ has property Q. Then we
            may pullback by the open immersion $\Spec B_i$ to get an affine open
            cover of $X \times_Z Y$. If Q holds for the morphism
            $C \to C \otimes_A B_i$ then we have an affine open cover of
            $X \times_ZY$ satisfying P, as desired.
\end{enumerate} \qed

We now begin the actual proof:
\begin{enumerate}[a.]
      \item Apply part (a) of the lemma by 7.3.C. We use notation as in the lemma.
            $X$ may be covered by finitely many open affines by
            quasicompactness of $X \to \Spec A$. Call these $\{\Spec B_i\}_{i \in [1, n]}$.
            Then the finitely many $\Spec C \times_{\Spec A} \Spec B_i$ are affine and cover
            $X \times_ZY$.

      \item Apply part (a) of the lemma by 7.3.C. Given the intersection of two open
            affines of $X$, we form a finite affine open cover. The
            pullback of this finite affine open cover covers the pullback of the
            intersection.
      \item Apply part (a) of the lemma by 7.3.4. Then we may take $X$
            to be affine since $f$ is affine. Then we are done as
            morphisms of affine schemes are affine.
      \item Apply (e) and (g).
      \item Apply part (b) of the lemma, using the fact that integrality is affine-local on
            the target: 7.3.10. Then apply 7.3.N, base change of integral morphism of
            affine schemes is affine.
      \item Apply part (b) of the lemma by 7.3.O. Then the result follows by 9.2.B combined
            with 9.2.A (tensor product preserves finitely generated algebra-ness).
      \item Apply (a) and (f).
      \item The same proof as (f) applies here without change, simply noting that the
            extension of a finitely generated ideal is finitely generated.
      \item Apply (a), (b), and (h).
\end{enumerate} \qed

\section{9.4.C}
Quasifiniteness obviously satisfies the conditions of part (b) of the lemma
above. Thus we may reduce to the pullback of $f: \Spec A \to \Spec B$ along
$g: \Spec C \to \Spec B$. Now, consider $\Spec \kappa(\p) \to \Spec C \to \Spec B$. This factors
through the morphism $\Spec \kappa(\p) \to \Spec \kappa(g(\p)) \to \Spec B$. Clearly $f^{-1}(g(\p)) \to \Spec \kappa(g(\p))$
is finite type (since it is the pullback of $f$). Thus we
reduce to the case of the pullback of field by affine over a field. This case
is covered by 7.4.D, making the morphism finite. Then it pulls back to a finite
morphism over the field, hence it has finite fibers. \qed

\section{9.4.D}
We consider $X\times_ZY$ where $f: Y \to Z$ is surjective
and $g: X \to Z$ is arbitrary. Let $\p \in X$. Then by
9.3.C we may reduce to $\Spec \kappa(\p) \to \Spec \kappa(g(\p))$. Now, choosing a point
$\q \in f^{-1}(g(\p))$ we are reduced to examining $\kappa(\p) \otimes_{\kappa(g(\p))}\kappa(\q)$.
Now, the tensor product of two non-zero vector spaces is non-zero, since there
is only one element of the ground field which annihilates both. Thus, the above
tensor product, being of two $\kappa(g(\p))$-vector spaces must be
non-zero. It thus has a maximal ideal, hence there is at least one point in the
fiber over $\p$.

\section{9.4.E}
We let $k=\overline{k}$. By 9.2.D $X\times_kY$ is finite
type. By 9.5.L, we may prove irreduciblity on an affine open cover. Also, by
5.2.A (and the discussion in 5.3.2), reducedness is an affine-local property.
Thus, if we show that $\Spec A \otimes_k \Spec B$ is integral for (finitely
generated over $k$) integral domains
$A, B$, then we are done. Now, $B$ is a
free $k$-module, so we may choose generators
$\{b_i\}_{i \in I}$ for some index set $I$. Now,
suppose $(\sum_i a_i \otimes b_i)(\sum_j a'_j \otimes b'_j) = 0$ where $a_i, a'_j \neq 0$ and the
$b_i, b'_j$ are generators of $B$. Now,
$a_1a'_1 \neq 0$, so $\D(a_1a'_1)$ is nonempty and thus has a
maximal ideal $\m$. Consider $A/\m \otimes_k B$. By the
weak nullstellensatz, we have $A/\m \cong k$, so this is just
$B$. Now $(\sum_i \overline{a_i} \otimes b_i)(\sum_j \overline{a'_j}
      \otimes b'_j)=0 \in B$. But also
$a_1 \neq 0$ in $A/\m$ so that
$\overline{a_1}\otimes b_1 \neq 0$ and $\sum_i \overline{a_i} \otimes b_i \neq 0$ (and likewise with
$a'_1$). But this contradicts the fact that
$B$ is an integral domain. Thus $a_i = 0$ and
$a'_j=0$ for all $i, j$. \qed

\section{9.4.F}
Consider the following diagram: \[\begin{tikzcd}
            {X\times_S X'}                             & {X'\times_SY} &
            {X'} \\{X\times_SY'} &
            {Y\times_SY'}                              &
            {Y'}                                                                                    \\
            X                                          & Y             & S \arrow[from=1-3, to=2-3]
            \arrow[from=2-3, to=3-3] \arrow[from=3-2, to=3-3] \arrow[from=3-1, to=3-2]
            \arrow[from=2-2, to=3-2] \arrow[from=2-2, to=2-3] \arrow["g"', from=2-1, to=2-2]
            \arrow[from=2-1, to=3-1] \arrow[from=1-2, to=1-3] \arrow[from=1-2, to=2-2]
            \arrow[from=1-1, to=1-2] \arrow["f", from=1-1, to=2-1] \arrow[dashed, from=1-1, to=2-2]
            \arrow[curve={height=12pt}, from=3-1, to=3-3] \arrow[curve={height=-12pt}, from=1-3, to=3-3]
      \end{tikzcd}\] All the squares in this
diagram are pullbacks, hence by hypothesis $f$ and
$g$ in particular have property $P$.
Hence their composition, the (uniquely defined) dotted morphism has
$P$. \qed

\end{document}